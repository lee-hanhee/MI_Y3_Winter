\subsection{ML Estimator (Frequentist Approach)}
\begin{definition}
    \begin{itemize}
        \item Assume $\theta$ is \textbf{fixed} and find the "best" $\theta$:
        \[
        \hat{\theta} = \arg\max_{\theta} P_{\underline{X}}(\underline{x}|\theta) \quad \text{or} \quad \hat{\theta} = \arg\max_{\theta} f_{\underline{X}}(\underline{x}|\theta).
        \]
        \item Every $\theta$ value specifies a \textbf{different probability space} (e.g., coin, universe, etc.) for our experiment.
    \end{itemize}
\end{definition}

\subsection{MAP Estimator (Bayesian Approach)}
\begin{definition}
    \begin{enumerate}
        \item Assume \textbf{random} parameter $\Theta$ in the \textbf{same sample space} as our experiment.
        \item Assume we have a \textbf{prior} pmf/pdf for $\Theta$: $P_\Theta(\theta)$ or $f_\Theta(\theta)$.
        \item Find the most probable $\theta$ given the observations $\underline{X} = \underline{x}$:
        \[
        \hat{\theta} = \arg\max_{\theta} P_{\Theta|\underline{X}}(\theta|\underline{x}), \quad \text{if } \theta \text{ discrete},
        \]
        \[
        \hat{\theta} = \arg\max_{\theta} f_{\Theta|\underline{X}}(\theta|\underline{x}), \quad \text{if } \theta \text{ continuous}.
        \]
        \begin{itemize}
            \item \textbf{Posterior Distribution:} $f_{\Theta|\underline{X}} (\theta|\underline{x})$.
        \end{itemize}
    \end{enumerate}
\end{definition}

\subsubsection{Four Cases of Bayes' Rule}
\begin{definition}
    \[
    P_{\Theta|\underline{X}}(\theta|\underline{x}) =
    \begin{cases}
        \frac{P_{\underline{X}|\Theta}(\underline{x}|\theta)P_\Theta(\theta)}{P_{\underline{X}}(\underline{x})}, & \text{if } \underline{X} \text{ discrete}, \\
        \frac{f_{\underline{X}|\Theta}(\underline{x}|\theta)f_\Theta(\theta)}{f_{\underline{X}}(\underline{x})}, & \text{if } \underline{X} \text{ continuous}.
    \end{cases}
    \]

    \[
    f_{\Theta|\underline{X}}(\theta|\underline{x}) =
    \begin{cases}
        \frac{P_{\underline{X}|\Theta}(\underline{x}|\theta)f_\Theta(\theta)}{P_{\underline{X}}(\underline{x})}, & \text{if } \underline{X} \text{ discrete}, \\
        \frac{f_{\underline{X}|\Theta}(\underline{x}|\theta)f_\Theta(\theta)}{f_{\underline{X}}(\underline{x})}, & \text{if } \underline{X} \text{ continuous}.
    \end{cases}
    \]
\end{definition}

\begin{notes}
    The denominator $f_{\underline{X}}(\underline{x})$ is independent of $\theta$:
    \[
    f_{\underline{X}}(\underline{x}) = \int_{-\infty}^{\infty} f_{\underline{X}|\Theta}(\underline{x}|\theta) f_\Theta(\theta) d\theta.
    \]
\end{notes}

\subsection{Comparison: ML vs MAP}
\begin{definition}
    \[
        \hat{\theta}_{ML} = \arg\max_{\theta} p_{\underline{X}|\Theta}(\underline{x}|\theta),
    \]
    \[
        \hat{\theta}_{MAP} = \arg\max_{\theta} p_{\underline{X}|\Theta}(\underline{x}|\theta)p_\Theta(\theta).
    \]
    The expressions are algebraically similar, but the philosophies differ.
\end{definition}

\subsection{Example: Unknown Voltage}
\begin{enumerate}
    \item Assume prior knowledge that $\theta$ is uniformly distributed in $[0, 1]$:
    \[
    f_\Theta(\theta) =
    \begin{cases}
        1, & 0 \leq \theta \leq 1, \\
        0, & \text{otherwise}.
    \end{cases}
    \]
    \item We measure it using a voltmeter that outputs random readings $\underline{Y} = [y_1, y_2, \dots, y_n]$, uniformly distributed in $[0, \theta]$. How to estimate $\theta$?
\end{enumerate}

\subsection*{To Warm Up: ML Estimation (Ignoring the Prior)}
\begin{align*}
f_{\underline{Y}|\Theta}(\underline{y}|\theta) &= \prod_{i=1}^n f_{Y|\Theta}(y_i|\theta) \\
&= \frac{1}{\theta^n} \prod_{i=1}^n 1(0 \leq y_i \leq \theta).
\end{align*}
\[
\hat{\theta}_{ML} = \arg\max_{\theta} \frac{1}{\theta^n} 1(\theta \geq \max_{1 \leq i \leq n} y_i) = \max_{1 \leq i \leq n} y_i.
\]

\subsection*{Solution: MAP Estimation}
\begin{align*}
f_{\Theta|\underline{Y}}(\theta|\underline{y}) &= \frac{f_{\underline{Y}|\Theta}(\underline{y}|\theta) f_\Theta(\theta)}{f_{\underline{Y}}(\underline{y})} \\
&\propto f_{\underline{Y}|\Theta}(\underline{y}|\theta) f_\Theta(\theta) \\
&= \frac{1}{\theta^n} 1(\theta \geq \max_{1 \leq i \leq n} y_i) 1(0 \leq \theta \leq 1).
\end{align*}
\[
\hat{\theta}_{MAP} = \arg\max_{\theta} \frac{1}{\theta^n} 1(\theta \geq \max_{1 \leq i \leq n} y_i) 1(0 \leq \theta \leq 1) = \max_{1 \leq i \leq n} y_i.
\]

\subsection*{What If the Prior Is Different?}
Suppose the prior is:
\[
f_\Theta(\theta) =
\begin{cases}
    2, & \frac{1}{2} \leq \theta \leq 1, \\
    0, & \text{otherwise}.
\end{cases}
\]
This means "The voltage is at least $\frac{1}{2}$ but no more than 1." Then:
\begin{align*}
f_{\Theta|\underline{Y}}(\theta|\underline{y}) &\propto \frac{1}{\theta^n} 1(\theta \geq \max_{1 \leq i \leq n} y_i) 1\left(\frac{1}{2} \leq \theta \leq 1\right),
\end{align*}
\[
\hat{\theta}_{MAP} = \max \left\{\max_{1 \leq i \leq n} y_i, \frac{1}{2}\right\}.
\]

\begin{notes}
    \begin{enumerate}
        \item $\max_{1 \leq i \leq n} y_i$ is a sufficient statistic for ML and MAP in this scenario.
        \item $\hat{\theta}_{MAP} \to \max_{1 \leq i \leq n} y_i$ as $n \to \infty$, regardless of the prior.
        \item MAP optimization typically requires more computational effort than ML estimation.
    \end{enumerate}
\end{notes}