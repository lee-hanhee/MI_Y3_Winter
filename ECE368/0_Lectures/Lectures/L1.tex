\begin{faq}
    \begin{itemize}
        \item How to study? Practice, practice.
        \item What textbooks? Use 2024 version of Murphy, Leon Garcia as main reference, Bishop, 4th textbook is intro.
        \item How is HW graded? Effort, and tutorials are used to explain soln. 
    \end{itemize}
\end{faq}

\subsection{Sample Space}
\begin{motivation}
    If you have 4 sheeps and a flea, the probability that starting from sheep 1, the flea will jump to sheep 4 in 10 steps is 0.2.
    \begin{itemize}
        \item Ambigious as there are 2 different interpretations for the sample space (i.e. space of probability is not clear):
        \begin{itemize}
            \item Set of sheeps
            \item Set of number of steps
        \end{itemize}
    \end{itemize}
\end{motivation}

\subsection{Probability Definitions}
\begin{definition}
    \begin{itemize}
        \item \textbf{Random Experiment:} An outcome (realization) for each run. 
        \item \textbf{Sample Space $\Omega$:} Set of all possible outcomes.
        \item \textbf{Events:} (measurable) subsets of $\Omega$.
        \item \textbf{Probability of Event A:} $P[A] \equiv P[\text{'outcome is in A'}]$.
    \end{itemize}
\end{definition}

\begin{example} \textbf{Roll Fair Die}
    \begin{itemize}
        \item $\Omega = \{1, 2, 3, 4, 5, 6\}$.
        \item $P[\text{'even number'}] = \frac{1}{2}$.
    \end{itemize}
\end{example}

\subsection{Axioms of Probability}
\begin{definition}
    \begin{enumerate}
        \item $P[A] \geq 0$ for all $A \in \Omega$.
        \item $P[\Omega] = 1$.
        \item If $A \cap B = \emptyset$, then $P[A \cup B] = P[A] + P[B]$ for all $A, B \in \Omega$.
        \customFigure[0.25]{../Images/L1_0.png}{3rd Axiom}
    \end{enumerate}
\end{definition}

\subsection{Conditional Probability}
\begin{definition}
    \begin{equation}
        P[A|B] = \frac{P[A \cap B]}{P[B]}
    \end{equation}
    \begin{itemize}
        \item $|$: Given event (data/obs.).
    \end{itemize}
    \customFigure[0.25]{../Images/L1_1.png}{Conditional Probability}
\end{definition}

\begin{notes}
    \begin{itemize}
        \item Changing sample space to $B$.
        \item Conditional probability satisfy the 3 axioms (i.e. are probabilities), can be viewed as probability measure on new sample space B.
    \end{itemize}
\end{notes}

\subsubsection{Consequences of Conditional Probability}
\begin{definition}
    \begin{equation}
        P[A \cap B] = P[A|B]P[B] = P[B|A]P[A]
    \end{equation}
\end{definition}

\subsubsection{Independence}
\begin{definition}
    $A$ and $B$ are independent iff 
    \begin{equation}
        P[A \cap B] = P[A]P[B] \iff P[A|B] = P[A] \iff P[B|A] = P[B]
    \end{equation}
\end{definition}

\subsubsection{Importance of Labelling}
\begin{example} \textbf{Toss 2 Fair Coins}
    \begin{enumerate}
        \item \textbf{Given:} Given that one of the coins is heads, what is the probability that the other coin is tails?
        \item \textbf{Wrong Solution:} $\frac{1}{2}$ since $\{HH, HT, TH, TT\}$, so $P[T|H] = \frac{1}{2}$, which assumes that the coins are distinguishable (i.e. coin \#1 is heads)
        \item \textbf{Correct Solution:} $\frac{2}{3}$ since $\{HH, HT, TH\}$ as we didn't specify which coin was heads, so $P[T|H] = \frac{2}{3}$, which assumes that the coins are indistinguishable.
    \end{enumerate}
    
\end{example}


