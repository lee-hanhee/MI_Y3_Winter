\begin{faq}
    \begin{itemize}
        \item Why can we say that it is consistent for the last example?
    \end{itemize}
\end{faq}

\subsection{Maximum Likelihood Estimation}
\begin{motivation}
    Choose parameter $\theta$ that is most likely to generate the observation $x_1,x_2,\ldots,x_n$.
    \customFigure[0.5]{../Images/L5_0.png}{}
\end{motivation}

\begin{definition}
    \begin{equation}
        \hat{\Theta} = \arg\max_\theta \, P_{\underline{X}}(\underline{x}|\theta), \text{ discrete } X.
    \end{equation}
        
    \begin{equation}
        \hat{\Theta} = \arg\max_\theta \, f_{\underline{X}}(\underline{x}|\theta), \text{ continuous } X.
    \end{equation}
        
\end{definition}

\subsubsection{Log-Likelihood}
\begin{definition}
    \begin{equation}
        \hat{\theta} = \arg\max_\theta \sum_{i=1}^n \log P_X(x_i|\theta) 
    \end{equation}

    \begin{equation}
        \hat{\theta} = \arg\max_\theta \sum_{i=1}^n \log f_X(x_i|\theta).
    \end{equation}    
\end{definition}

\begin{warning}
    Can only go from argmax of fcn to argmax of log fcn, if it is i.i.d. 
\end{warning}

\begin{derivation}
    \begin{align*}
        \text{i.i.d. } X_1, X_2, \dots, X_n &\implies \\
        p_{\underline{X}}(\underline{x} | \theta) &= \prod_{i=1}^n p_{X_i}(x_i | \theta) \\
        &= \prod_{i=1}^n p_X(x_i | \theta) \quad \text{drop the i due to i.i.d. assumption}\\
        \log p_{\underline{X}}(\underline{x} | \theta) &= \sum_{i=1}^n \log p_X(x_i | \theta).
    \end{align*}
\end{derivation}

\begin{example}
    \begin{enumerate}
        \item \textbf{Model and Observations:}
        \begin{itemize}
            \item Assume a biased coin with probability $\theta$ of showing heads. Find ML estimator for $\theta$.
            \item Toss the coin $n$ times and obtain Bernoulli random variables $X_1, \dots, X_n$ such that:
            \[
            \text{"heads" } \to 1, \quad \text{"tails" } \to 0.
            \]
            \item Total number of heads is:
            \[
            k = \sum_{i=1}^n X_i.
            \]
            For example:
            \[
            \underline{x} = (1, 0, 0, 0, 1, 1, 0, 1, 1, 1), \quad k = 6.
            \]
            \item Probability of observations $x_1,\ldots,x_n$ corresponding to parameter $\theta$ is:
            \[
            p_{\underline{X}}(\underline{x} | \theta) = \theta^k (1 - \theta)^{n - k}.
            \]
            \begin{itemize}
                \item It is sufficient to know only $k$.
                \item Note: Don't need the $\binom{n}{k}$ term because we are given the specific sequence of heads and tails.
            \end{itemize}
        \end{itemize}
        
        \item \textbf{Log-Likelihood and Maximization:}
        \begin{itemize}
            \item The log-likelihood function is:
            \[
            \log p_{\underline{X}}(\underline{x} | \theta) = k \log(\theta) + (n - k) \log(1 - \theta).
            \]
            \item To maximize the log-likelihood over $\theta$, set:
            \begin{align*}
            0 &= \frac{\partial}{\partial \theta} \log p_{\underline{X}}(\underline{x} | \theta), \\
            0 &= \frac{k}{\theta} - \frac{n - k}{1 - \theta}, \\
            \theta &= \frac{k}{n}.
            \end{align*}
            \item Thus, the Maximum Likelihood Estimator (MLE) is:
            \[
            \hat{\Theta} = \frac{k}{n}, \quad \text{where } k = \sum_{i=1}^n X_i.
            \]
            This corresponds to the observed frequency of heads, which is intuitive b/c the more heads we see, the more likely the coin is biased towards heads.
        \end{itemize}
        
        \item \textbf{Examples:}
        \begin{itemize}
            \item For $\underline{x} = (1, 0, 0, 0, 1, 1, 0, 1, 1, 1)$:
            \begin{align*}
            p_X(\underline{x} | \theta) &= \theta^6 (1 - \theta)^4, \\
            \hat{\theta} &= \frac{6}{10} = 0.6.
            \end{align*}
            \item For $\underline{x} = (0, 1, 1, 1, 0, 0, 1, 0, 1, 0)$:
            \begin{align*}
            p_X(\underline{x} | \theta) &= \theta^5 (1 - \theta)^5, \\
            \hat{\theta} &= \frac{5}{10} = 0.5.
            \end{align*}
        \end{itemize}
    \end{enumerate}
\end{example}

\begin{notes}
    \begin{itemize}
        \item[1.] $k$ is a sufficient statistic for this Maximum Likelihood (ML) estimator.
    
        \item[2.] The expectation of the estimator $\hat{\theta}$ is:
        \begin{align*}
        E[\hat{\Theta}] &= E\left[\frac{1}{n} \sum_{i=1}^n X_i \right] \\
        &= \frac{1}{n} \sum_{i=1}^n E[X_i] \\
        &= \frac{1}{n} (n \theta) \\
        &= \theta \quad \text{(Unbiased)}.
        \end{align*}
        \begin{itemize}
            \item $E[X_i] = (1)\theta + (0)(1 - \theta) = \theta$
        \end{itemize}
        
        \item[3.] In fact, $\hat{\Theta} = \frac{1}{n} \sum_{i=1}^n X_i$ is the sample mean, and $\theta$ is the true mean. Therefore, $\hat{\theta} \to \theta$ in probability, which implies that $\hat{\theta}$ is \textit{consistent}.
    \end{itemize}    
\end{notes}

\subsection{Sample Mean is Not Always an ML Estimator}
\begin{example}
    Given an unknown voltage \( x \), we measure it using a voltmeter that outputs a random reading \( Y \) that is uniform in \( [0, x] \). Suppose we make \( n \) i.i.d. measurements \( Y_1, \dots, Y_n \) and wish to estimate \( \mu = \frac{x}{2} = \mathbb{E}[Y] \).
    \begin{enumerate}
        \item \textbf{Sample Mean:} 
        \[
        \hat{\mu} = \frac{1}{n} \sum_{i=1}^n Y_i \quad \cdots (1)
        \]

        \item \textbf{To Find the ML Estimator:}
        \begin{align*}
            f_Y(\mathbf{y} \mid \mu) &= \prod_{i=1}^n f_Y(y_i \mid \mu) \\
            &= \prod_{i=1}^n \frac{1}{2\mu} \cdot 1(0 \leq y_i \leq 2\mu) \\
            &= \frac{1}{(2\mu)^n} \prod_{i=1}^n 1(0 \leq y_i \leq 2\mu).
        \end{align*}
        
        The likelihood is maximized for:
        \[
        \arg\max_{\mu} f_Y(\mathbf{y} \mid \mu) = \max_{1 \leq i \leq n} \frac{1}{2} Y_i.
        \]

        Therefore:
        \[
        \hat{\mu} = \max_{1 \leq i \leq n} \frac{1}{2} Y_i \quad \cdots (2)
        \]

        \item Clearly, \( (1) \neq (2) \).
    \end{enumerate}
\end{example}



