\subsection{Symmetries in Data}
\begin{summary}
    Transformations that preserve data characteristics. 
    \begin{center}
        \begin{tabular}{ll}
            \toprule
            \textbf{Transformation Type} & \textbf{Description} \\
            \toprule
            Invariance ($f(g(x))=f(x)$) & Invariant to a trans. if output is unchanged when the input does that trans.\\
            \multicolumn{2}{p{\linewidth}}{
            \begin{itemize}
                \item e.g. $f=\text{label}$, $g=\text{translation, scale, rotation}$. Regardless of transformation, label remains the same.
                \item Useful for classification tasks where transformations should not change the label.
            \end{itemize}} \\
            \midrule
            Equivariance ($f(g(x)) = g(f(x))$) & Equivariant to a trans. if output changes in the same way as input. \\
            \multicolumn{2}{p{\linewidth}}{
            \begin{itemize}
                \item e.g. $f=\text{position}$, $g=\text{translation}$. If $f$ is applied first and then $g$, the output changes in the same way as applying $g$ first and then $f$.
                \item Useful in tasks where spatial relationships need to be preserved, such as object localization.
            \end{itemize}} \\
            \bottomrule
        \end{tabular}
    \end{center}
    \vspace{1em}

    \begin{center}
        \begin{tabular}{ll}
            \toprule
            \textbf{Data Type} & \textbf{Symmetry} \\
            \toprule
            Tabular Data & Row permutation invariance \\
            \multicolumn{2}{p{\linewidth}}{
            \begin{itemize}
                \item Ordering of rows does not affect the output.
            \end{itemize}} \\
            \midrule
            Sets & Element permutation invariance \\ 
            \multicolumn{2}{p{\linewidth}}{
            \begin{itemize}
                \item Elements have no inherent order, so the output should not change if elements are swapped.
            \end{itemize}} \\
            \midrule
            Images & Translation, rotation, and scaling invariance \\
            \multicolumn{2}{p{\linewidth}}{
            \begin{itemize}
                \item Image recognition should not be affected by the translation, rotation, or scaling of the image.
            \end{itemize}} \\
            \midrule
            Time-series & Time shift invariance \\
            \multicolumn{2}{p{\linewidth}}{
            \begin{itemize}
                \item Patterns should be the same regardless of when they occur.
            \end{itemize}} \\
            \midrule
            Graphs & Node permutation invariance \\
            \multicolumn{2}{p{\linewidth}}{
            \begin{itemize}
                \item Nodes can be rearranged without changing the graph's structure since the edges remain the same.
            \end{itemize}} \\
            \midrule
            Text & Sentence structure and paraphrasing invariance \\
            \multicolumn{2}{p{\linewidth}}{
            \begin{itemize}
                \item Rewording a sentence or changing its structure should not change its meaning.
            \end{itemize}} \\
            \bottomrule
        \end{tabular}
    \end{center}
\end{summary}
\newpage

\subsection{Learning on Tabular Data}
\begin{notes}
    \begin{itemize}
        \item \textbf{Problem:} DL struggles with tabular data because it lacks the spatial and sequential structures found in images and time-series, making it difficult for NNs to extract meaningful patterns. 
        \item \textbf{Soln:} XGBoost (tree-ensemble method):
        \begin{itemize}
            \item Automatic feature selection.
            \item Mixed data types.
            \item Robust to outliers.
            \item Capture nonlinear relationships.
            \item Computationally efficient and fast.
            \item Easy to set up.
        \end{itemize}
    \end{itemize}
\end{notes} 

\subsection{Learning on Sets}
\begin{summary}
    Unordered collections of distinct/unique elements 
    \begin{center}
        \begin{tabular}{ll}
            \toprule
            \textbf{Concepts} & \textbf{Description} \\
            \toprule
            Data Sets & Each data point is i.i.d. R.V. with no inherent order. \\
            \midrule
            \multicolumn{2}{p{\linewidth}}{
            \begin{itemize}
                \item Points are indep. 
                \item Summing over loss fn is invariant to ordering of elements.
                \item Unbiased estimate of the loss via stochastic subsampling.
            \end{itemize}} \\
            \bottomrule
        \end{tabular}
    \end{center}
\end{summary}