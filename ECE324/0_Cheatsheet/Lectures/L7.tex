\section{Software Engineering}

\subsection{Code Readability Matters}
\begin{notes}
    \begin{itemize}
        \item \textbf{Overview:} Maintainable code enables collaboration and future work.
        \customFigure[0.5]{../Images/L7_1.png}{}
        \customFigure[0.5]{../Images/L7_4.png}{}
        \item \textbf{Naming:} Clear names enhance code understanding 
        \begin{itemize}
            \item \textbf{Functions:} verb\_do
            \item \textbf{Variables:} two\_three\_words
            \item \textbf{Classes:} CapitalizedWords
        \end{itemize}
        \customFigure[0.5]{../Images/L7_2.png}{}
    \end{itemize}
\end{notes}

\subsection{Style Guides}
\begin{notes}
    \begin{itemize}
        \item \textbf{Overview:} Ensure consistent code formatting. 
        \item \textbf{Why?} Provide a standardized set of rules for formatting code, prompting uniformity, and reducing cognitive load when reading code. 
        \item \textbf{Examples:} PEP8, Google Style Guide, etc.
    \end{itemize}
    \customFigure[0.5]{../Images/L7_3.png}{}
    \customFigure[0.5]{../Images/L7_5.png}{}
\end{notes}
\newpage

\section{Coding Mantras / Ideas}
\subsection{Zen of Python}
\begin{definition}
    \customFigure[0.5]{../Images/L7_0.png}{}
\end{definition}

\subsection{KISS: Keep It Simple and Straightforward}
\begin{notes}
    Advocate for solutions that are easy to understand and maintain, reduce unnecessary complexity.
    \customFigure[0.5]{../Images/L7_6.png}{}
\end{notes}

\subsection{YAGNI: You Aren't Gonna Need It}
\begin{notes}
    AKA premature optimization is the root of all evil. 
    \customFigure[0.5]{../Images/L7_7.png}{}
    \customFigure[0.5]{../Images/L7_8.png}{}
\end{notes}

\subsection{Principle of DRY and WET and DAMP}
\begin{notes}
    Don't Repeat Yourself; Write Everything Twice.
    \customFigure[0.5]{../Images/L7_9.png}{}
    \vspace{1em}

    Don't Abstract Methods Prematurely.
    \customFigure[0.5]{../Images/L7_10.png}{}
\end{notes}
\newpage

\section{Into The Weeds (Technical Details)}
\subsection{Python Data Containers Overview}
\begin{definition}
    Choosing the right container for the task:
    \begin{itemize}
        \item \textbf{list:} Ordered, mutable sequence; versatile for collections of items.
        \item \textbf{tuple:} Ordered, immutable sequence; suitable for fixed collections and data integrity.
        \item \textbf{set:} Unordered collection of unique elements; efficient for membership testing and removing duplicates.
        \item \textbf{dict:} Key-value mappings; ideal for efficient data lookup and representing structured information.
    \end{itemize}
    \vspace{1em}
    
    \textbf{Data Structures:}
    \begin{itemize}
        \item \textbf{collections.NamedTuple}
        \item \textbf{dataclasses.Dataclass}
    \end{itemize}
    \vspace{1em}

    \textbf{Tensors}
    \begin{itemize}
        \item \textbf{numpy.ndarray}
        \item \textbf{torch.Tensor}
        \item \textbf{jax.Array}
    \end{itemize}
\end{definition}

\subsection{List Comprehensions}
\begin{notes}
    Provide concise list creation.
    \customFigure[0.5]{../Images/L7_11.png}{}
\end{notes}

\subsection{Leveraging Set Data Structures: Set}
\begin{notes}
    Efficiently handle unique elements and set logic.
    \customFigure[0.5]{../Images/L7_12.png}{}
\end{notes}

\subsection{Dictionaries for Key-Value Pairs}
\begin{notes}
    Enable efficient data lookup by keys.
    \customFigure[0.5]{../Images/L7_13.png}{}
\end{notes}

\subsection{Itertools for Efficient Iteration}
\begin{notes}
    Provides tools for very common iteration patterns.
    \begin{itemize}
        \item e.g. permutations, chunked, chain, zip, filter, product, combinations, etc.
    \end{itemize}
    \customFigure[0.5]{../Images/L7_14.png}{}
    \customFigure[0.5]{../Images/L7_15.png}{}
\end{notes}

\subsection{Functools for Functional Tools}
\begin{notes}
    Need to manipulate a function and get a new function. 
    \begin{itemize}
        \item Caching with \texttt{functools.lru\_cache}: Cache optimizes performance by storing results (i.e. return a previous result when the input of a function has already been observed)
    \end{itemize}
    \customFigure[0.5]{../Images/L7_16.png}{}
    \customFigure[0.5]{../Images/L7_17.png}{}
\end{notes}

\subsection{Abstract Base Classes (ABCs)}
\begin{notes}
    ABCs enforce interfaces for robust design.
    \customFigure[0.5]{../Images/L7_18.png}{}
\end{notes}

\subsection{Dataclasses for Data Storage}
\begin{notes}
    Data classes simplify data-centric class creation.
    \customFigure[0.5]{../Images/L7_19.png}{}
\end{notes}

\subsection{Python Typing Provides Hints}
\begin{notes}
    Type hints enhance code clarity, communicate intent and detect errors.
    \customFigure[0.5]{../Images/L7_20.png}{} 
\end{notes}

\subsection{Tensor Typing: jaxtyping}
\begin{notes}
    Communicate expected tensor shapes and data types. 
    \customFigure[0.5]{../Images/L7_21.png}{}
\end{notes}

\subsection{pytest: Helps you write better programs}
\begin{notes}
    Simplifies testing for reliable code.
    \customFigure[0.5]{../Images/L7_22.png}{}
\end{notes}


