\begin{definition}
    Learnable \textbf{(optimizable)} transformations of data.
    \begin{equation}
        x \overset{\text{Model}}{\mapsto} y
    \end{equation}
\end{definition}

\subsection{Challenges in NN Training}
\begin{notes}
    \begin{itemize}
        \item Non-convex loss landscapes
        \item Vanishing/exploding gradients
        \item Overfitting to training data
        \item Computational cost
    \end{itemize}
\end{notes}
\newpage

\subsection{Data}
\begin{summary}
    \begin{center}
        \begin{tabular}{ll}
        \toprule
        \textbf{Techniques} & \textbf{Description} \\
        \midrule
        \textbf{Data Exploration} & Initial data inspection informs modeling choices. \\
        \multicolumn{2}{p{\linewidth}}{
        \begin{itemize}
            \item Think in terms of "x" and "y".
            \item Summary statistics
            \item Identifying data imbalances
            \item \textbf{Libraries:} Pandas, Matplotlib, Seaborn
        \end{itemize}} \\
        \midrule
        \textbf{Data Splitting} & Proper data partitioning prevents overly optimistic performance estimates. \\
        \multicolumn{2}{p{\linewidth}}{
        \begin{center}
            \customFigure[0.5]{../Images/L4_0.png}{}
            \vspace{-4em}
        \end{center}} \\
        \midrule
        \textbf{Cross-Validation} & Assess model performance across multiple data partitions. \\
        \multicolumn{2}{p{\linewidth}}{
        \begin{itemize}
            \item Provides more reliable performance estimates. 
            \customFigure[0.5]{../Images/L4_1.png}{}
        \end{itemize}} \\
        \bottomrule
        \end{tabular}
    \end{center}
\end{summary}
\newpage

\begin{summary}
    \begin{center}
        \begin{tabular}{ll}
        \toprule
        \textbf{Techniques} & \textbf{Description} \\
        \midrule
        \textbf{Stratified Splitting} & Ensure each data split reflects the original class distribution. \\
        \multicolumn{2}{p{\linewidth}}{
        \begin{itemize}
            \item Maintains class proportions. 
            \begin{itemize}
                \item Stratification will balance one classification label.  
            \end{itemize}
            \customFigure[0.5]{../Images/L4_2.png}{}
        \end{itemize}} \\
        \midrule
        \textbf{Adversarial Splits} & Evaluate model perf. under challenging, out-of-distribution scenarios for robustness. \\
        \multicolumn{2}{p{\linewidth}}{
        \begin{itemize}
            \item Leave out: Clusters, Labels, Temporal Data
            \customFigure[0.5]{../Images/L4_3.png}{}
        \end{itemize}} \\
        \bottomrule
        \end{tabular}
    \end{center}
\end{summary}
\newpage

\begin{summary}
    \begin{center}
        \begin{tabular}{ll}
        \toprule
        \textbf{Techniques} & \textbf{Description} \\
        \midrule
        \textbf{Data Augmentation} & Create new training ex. by applying trans. to existing data for $\uparrow$ data variability. \\
        \multicolumn{2}{p{\linewidth}}{
        \begin{itemize}
            \item Transformations should be relevant to the data.
            \item Increase data size.
            \item Bake in "biases"/"priors" into your model.
            \customFigure[0.5]{../Images/L4_4.png}{}
        \end{itemize}} \\
        \midrule
        \textbf{Image Augmentation} & Apply transformations like rotations, flips, and crops to images. \\
        \multicolumn{2}{p{\linewidth}}{
        \begin{center}
            \customFigure[0.5]{../Images/L4_5.png}{}
            \vspace{-4em}
        \end{center}} \\
        \bottomrule
        \end{tabular}
    \end{center}
\end{summary}
\newpage

\subsection{Evaluation Metrics Quantify Performance}
\begin{summary}
    \begin{center}
        \begin{tabular}{ll}
        \toprule
        \textbf{Metrics} & \textbf{Description} \\
        \midrule
        \textbf{Classification Metrics} &  \\
        \midrule 
        Accuracy & $\frac{\text{Correct Predictions}}{\text{Total Predictions}}$ \\
        \multicolumn{2}{p{\linewidth}}{
        \begin{itemize}
            \item \textbf{When to Use?} (1) classes are balanced and (2) false +/false - have similar importance.
        \end{itemize}} \\
        \midrule
        Precision & $\frac{\text{True Positives}}{\text{True Positives} + \text{False Positives}}$ \\
        \multicolumn{2}{p{\linewidth}}{
        \begin{itemize}
            \item \textbf{When to Use?} (1) false + are costly (e.g. nuclear war, since saying there is a nuclear war when there's not is bad) and (2) you want to minimize false +.
        \end{itemize}} \\
        \midrule 
        Recall & $\frac{\text{True Positives}}{\text{True Positives} + \text{False Negatives}}$ \\
        \multicolumn{2}{p{\linewidth}}{
        \begin{itemize}
            \item \textbf{When to Use?} (1) false - are costly (e.g. cancer detection, since missing cancer when there's cancer is bad) and (2) you want to minimize false -.
        \end{itemize}} \\
        \midrule 
        F1-Score & $2 \times \frac{\text{Precision} \times \text{Recall}}{\text{Precision} + \text{Recall}}$ \\
        \multicolumn{2}{p{\linewidth}}{
        \begin{itemize}
            \item \textbf{When to Use?} (1) you want to balance precision and recall.
        \end{itemize}} \\
        \midrule 
        AUROC & Area under the ROC curve. \\
        \multicolumn{2}{p{\linewidth}}{
        \begin{itemize}
            \item \textbf{How?} Assesses model perf. across all class. thresholds by plotting true + rate against false + rate.
            \item \textbf{Char.} Monotonic, ranking-based, $1.0$ is perfect, $0.5$ is random.
            \customFigure[0.3]{../Images/L4_8.png}{}
        \end{itemize}} \\
        \midrule 
        AUPRC & Area under the Precision-Recall curve. \\
        \multicolumn{2}{p{\linewidth}}{
            \begin{itemize}
                \item \textbf{How?} Assesses model performance across all classification thresholds by plotting precision against recall.
                \item \textbf{When to Use?} (1) classes are imbalanced as it focuses on trade-offs between precision and recall.
        \end{itemize}} \\
        \midrule
        Average Precision (AP) & $\sum_n \text{Precision}(n) \times \text{Recall}(n)$ \\
        \multicolumn{2}{p{\linewidth}}{
            \begin{itemize}
                \item \textbf{What?} Measures how well a model ranks positive examples above negative examples.
                \item \textbf{When to Use?} (1) summarize precision-recall curve, making it suitable for ranking tasks.
            \end{itemize}} \\
        \midrule
        Mean Average Precision (mAP) & $\frac{1}{N} \sum_{n=1}^{N} \text{AP}(n)$ \\
        \multicolumn{2}{p{\linewidth}}{
        \begin{itemize}
                \item \textbf{When to Use?} (1) multi-class or object detection tasks to compute the average AP across classes.
        \end{itemize}} \\
        \midrule
        \end{tabular}
    \end{center}
\end{summary}
\newpage

\begin{summary}
    \begin{center}
        \begin{tabular}{ll}
        \toprule
        \textbf{Metrics} & \textbf{Description} \\
        \midrule
        Intersection over Union (IoU) & $\frac{\text{Area of Overlap}}{\text{Area of Union}}$ \\
        \multicolumn{2}{p{\linewidth}}{
            \begin{itemize}
                \item \textbf{When to Use?} (1) object detection tasks to evaluate the overlap between predicted and ground truth bounding boxes.
            \end{itemize}} \\
        \toprule
        \textbf{Regression Metrics} & $R^2$, MAE, Pearson $r$, RMSE, MSE, MAPE \\ 
        \multicolumn{2}{p{\linewidth}}{
        \begin{itemize}
            \item 
        \end{itemize}} \\
        \midrule
        \textbf{Ranking Metrics} & Kendall Tau, Spearman \\
        \multicolumn{2}{p{\linewidth}}{
        \begin{itemize}
            \item 
        \end{itemize}} \\
        \midrule
        \textbf{Multitask Metrics} & Mean of metrics, holistic metric \\
        \multicolumn{2}{p{\linewidth}}{
        \begin{itemize}
            \item 
        \end{itemize}} \\
        \bottomrule
        \end{tabular}
    \end{center}
\end{summary}
\newpage

\begin{summary}
    \begin{center}
        \begin{tabular}{ll}
        \toprule
        \textbf{Concepts} & \textbf{Description} \\
        \midrule
        \textbf{Adjusting Decision Thresholds} & Modifying the threshold impacts precision and recall. \\
        \multicolumn{2}{p{\linewidth}}{
        \begin{itemize}
            \item How to interpret your probabiltiies is a modelling decision.
            \customFigure[0.4]{../Images/L4_6.png}{}
        \end{itemize}} \\
        \midrule
        \textbf{Calibrating Predicted Probabilities} & Predicted probabilities may not reflect true likelihood. \\
        \multicolumn{2}{p{\linewidth}}{
        \begin{center}
            \customFigure[0.4]{../Images/L4_7.png}{}
            \vspace{-4em}
        \end{center}} \\
        \bottomrule
        \end{tabular}
    \end{center}
\end{summary}
\newpage

\begin{summary}
    \begin{center}
        \begin{tabular}{l}
        \toprule
        \textbf{Tips} \\
        \midrule
        \textbf{Regression Correlation Metrics} \\
        \multicolumn{1}{p{\linewidth}}{
        \begin{itemize}
            \item $R^2$ and Pearson's $r$ quantify linear associations. 
            \begin{itemize}
                \item Pearson $r$ (linear correlation)
            \end{itemize}
            \item \( R^2 = 0 \): Represents the mean of the data.
            \item \( R^2 = 1 \): Indicates a perfect fit.
            \item \( R^2 \): Can be infinitely worse in some cases, context-dependent.
            \customFigure[0.5]{../Images/L4_9.png}{}
        \end{itemize}} \\
        \bottomrule
        \end{tabular}
    \end{center}
\end{summary}
\newpage

\begin{summary}
    \begin{center}
        \begin{tabular}{l}
        \toprule
        \textbf{Tips} \\
        \midrule
        \textbf{Rank Correlation Metrics: Spearman and Kendall Tau} \\
        \multicolumn{1}{p{\linewidth}}{
        \begin{itemize}
            \item Spearman and Kendall measure monotonic relationships.
            \begin{itemize}
                \item Magnitude does not matter.
                \item Spearman: Similar to Pearson correlation but applied to ranks.
                \item Kendall: Pair-based approach, determines if the order is correct or not.
                \item Applicable for regression or classification tasks.
            \end{itemize}
        \end{itemize}} \\
        \midrule
        \textbf{Quantifying Metric Uncertainty} \\
        \multicolumn{1}{p{\linewidth}}{
        \begin{itemize}
            \item Confidence intervals provide a range of plausible values.
            \customFigure[0.5]{../Images/L4_10.png}{}
        \end{itemize}} \\
        \midrule
        \textbf{Bootstrapping for Robust Estimation} \\
        \multicolumn{1}{p{\linewidth}}{
        \begin{itemize}
            \item Resampling with replacement estimates metric variability.
            \begin{itemize}
                \item \textbf{Resample:} Create many datasets by randomly picking data points with replacement from your original data.
                \item \textbf{Calculate:} Compute your metric of interest on each of these resampled datasets.
                \item \textbf{Estimate:} Use the distribution of these calculated metrics to estimate the uncertainty (e.g., confidence interval) of your original metric.
            \end{itemize}
        \end{itemize}} \\
        \bottomrule
        \end{tabular}
    \end{center}
\end{summary}
\newpage

\subsection{Optimization / Training}
\begin{summary}
    \begin{center}
        \begin{tabular}{ll}
        \toprule
        \textbf{Techniques} & \textbf{Description} \\
        \midrule
        \textbf{Gradient Descent} & Use Adam and finetune learning rate. \\
        \midrule
        \textbf{Batch Size} & The number of samples per gradient update affects learning. \\
        \multicolumn{2}{p{\linewidth}}{
        \begin{itemize}
            \item \textbf{Trade-Off:} Smaller batch size $\rightarrow$ gradient variability, larger batch size $\rightarrow$ computationally efficient.
            \item \textbf{Heuristic:} Often, the ideal batch size will be the largest batch size supported by the available hardware.
            \item The batch size governs the training speed and shouldn't be used to directly tune the validation set performance. 
        \end{itemize}} \\
        \midrule
        \textbf{Learning Rate} & Try a range of values. \\
        \multicolumn{2}{p{\linewidth}}{
        \begin{itemize}
            \item \textbf{Trade-Off:} High learning rate $\rightarrow$ faster training, possibility of not stabilizing (offshoot the minima), low learning rate $\rightarrow$ slow.  
            \item \textbf{Heuristics:} Typically in the $1e^{-1}$ to $1e^{-6}$ range, use \texttt{log10} scale, use low for fine-tuning, $1e^{-5} - 1e^{-6}$
            \item Dataset/model/batch dependent
        \end{itemize}} \\
        \midrule
        \textbf{Learning Rate Decay} & Dynamically adjust the learning rate during training to reduce osc. \\ 
        \multicolumn{2}{p{\linewidth}}{
        \begin{itemize}
            \item Learning rate decay helps fine-tune the model in later stages of training, preventing overshooting the optimal parameters.
            \customFigure[0.3]{../Images/L4_11.png}{}
            \item Plateu: Learning rate decays only after hitting a plateau for a certain number of epochs.
            \customFigure[0.3]{../Images/L4_12.png}{}
        \end{itemize}} \\
        \bottomrule
        \end{tabular}
    \end{center}
\end{summary}
\newpage

\begin{summary}
    \begin{center}
        \begin{tabular}{l}
        \toprule
        \textbf{Techniques} \\
        \midrule
        \textbf{Gradient Clipping} \\
        \multicolumn{1}{p{\linewidth}}{
        \begin{itemize}
            \item Prevents exploding gradients.
            \item Limit the magnitude of gradients during training.
            \[
            g \leftarrow 
            \begin{cases} 
            \lambda \frac{g}{\|g\|} & \text{if } \|g\| > \lambda \\
            g & \text{otherwise}
            \end{cases}
            \]
            \begin{itemize}
                \item Typical values: 5 or 1.
            \end{itemize}
            \customFigure[0.5]{../Images/L4_13.png}{}
        \end{itemize}} \\
        \midrule
        \textbf{Early Stopping} \\
        \multicolumn{1}{p{\linewidth}}{
        \begin{itemize}
            \item Monitor validation performance and stop training when it deteriorates.
            \begin{itemize}
                \item Stop after negligible improvements.
                \item Save model and restore.
                \item Patience (how much to wait before stopping).
            \end{itemize}
            \customFigure[0.5]{../Images/L4_14.png}{}
        \end{itemize}} \\
        \midrule
        \textbf{Loss Function Weighting} \\
        \multicolumn{1}{p{\linewidth}}{
        \begin{itemize}
            \item Assign different weights to different classes during loss calculation.
            \[
            L_w(y, \hat{y}) = \frac{1}{\sum w_n} \sum_{n} w_n L(y_n, \hat{y}_n)
            \]
            \begin{itemize}
                \item Weights can be inverse class frequency.
            \end{itemize}
        \end{itemize}} \\
        \bottomrule
        \end{tabular}
    \end{center}
\end{summary}
\newpage

\begin{summary}
    \begin{center}
        \begin{tabular}{l}
        \toprule
        \textbf{Techniques} \\
        \midrule
        \textbf{Focal Loss} \\
        \multicolumn{1}{p{\linewidth}}{
        \begin{itemize}
            \item Focuses on hard examples by down-weighting the contribution of easy examples to the loss.
            \customFigure[0.5]{../Images/L4_15.png}{}
        \end{itemize}} \\
        \bottomrule
        \end{tabular}
    \end{center}
\end{summary}
\newpage

\subsection{Regularization \& Modelling}
\begin{summary}
    \begin{center}
        \begin{tabular}{ll}
        \toprule
        \textbf{Tools} & \textbf{Description} \\
        \midrule
        \textbf{Normalization Standardizes Feature Distributions} & Scale input features to be $\mathcal{N}(0,1)$ \\
        \multicolumn{2}{p{\linewidth}}{
        \begin{itemize}
            \item Stability, Speed, Prevents large scale features from dominating.
            \customFigure[0.3]{../Images/L4_16.png}{}
        \end{itemize}} \\
        \midrule
        \textbf{Batch and Layer Normalization} & Normalize the activations across batch or features. \\
        \multicolumn{2}{p{\linewidth}}{
        \begin{center}
            \customFigure[0.3]{../Images/L4_17.png}{}
            \vspace{-4em}
        \end{center}} \\
        \midrule
        \textbf{Residual Connections Facilitate Deeper Networks} & Add the input of a layer to its output. \\
        \multicolumn{2}{p{\linewidth}}{
        \begin{center}
            \customFigure[0.3]{../Images/L4_18.png}{}
            \vspace{-4em}
        \end{center}} \\
        \bottomrule
        \end{tabular}
    \end{center}
\end{summary}
\newpage

\begin{summary}
    \begin{center}
        \begin{tabular}{ll}
        \toprule
        \textbf{Tools} & \textbf{Description} \\
        \midrule
        \textbf{Dropout: Randomly Setting Activations Elements to Zero} \\
        \multicolumn{1}{p{\linewidth}}{
        \begin{itemize}
            \item Reduce overfitting, create an ensemble of subnetworks implicitly.
            \customFigure[0.3]{../Images/L4_19.png}{}
        \end{itemize}} \\
        \midrule
        \textbf{Ensemble Methods Combine Multiple Models} \\
        \multicolumn{1}{p{\linewidth}}{
        \begin{itemize}
            \item Improve predictive performance by aggregating predictions from diverse models.
            \begin{equation*}
                \hat{y} = \frac{1}{n} \sum_n^{\text{models}} f_n(x)
            \end{equation*}
            \item Always will give an equal or better model w.r.t. predictive performance.
        \end{itemize}} \\
        \midrule
        \textbf{Random Initialization Ensembles} \\
        \multicolumn{1}{p{\linewidth}}{
        \begin{itemize}
            \item Diverse initial states improves ensemble diversity.
            \begin{itemize}
                \item Same model, different random seeds.
            \end{itemize}
            \begin{equation*}
                \hat{y} = \frac{1}{n} \sum_n^{\text{models}} f_n(x;\theta_n)
            \end{equation*}
        \end{itemize}} \\
        \midrule
        \textbf{Hyperparameter Ensembles} \\
        \multicolumn{1}{p{\linewidth}}{
        \begin{itemize}
            \item Hyperparameter explorations give you more diverse ensembles.
            \[
            f_n \sim \text{comes from top-k models from a hyperparameter optimization, same model class}
            \]
            
            \[
            \hat{y} = \frac{1}{n} \sum_{n}^{\text{models}} f_n(x; \theta_n)
            \]
            \customFigure[0.3]{../Images/L4_20.png}{}
        \end{itemize}} \\
        \bottomrule
        \end{tabular}
    \end{center}
\end{summary}
\newpage

\begin{summary}
    \begin{center}
        \begin{tabular}{l}
        \toprule
        \textbf{Tips} \\
        \midrule
        \textbf{Uncertainty Quantification: Variability in Prediction} \\
        \multicolumn{1}{p{\linewidth}}{
        \begin{itemize}
            \item Uncertainty can represent the range of likely outcomes rather than relying solely on a single point prediction.
            \item Uncertainty $~\text{std}(f_1\ldots f_n)$
            \customFigure[0.5]{../Images/L4_21.png}{}
        \end{itemize}} \\
        \midrule
        \textbf{Not All Uncertainty is the Same} \\
        \multicolumn{1}{p{\linewidth}}{
        \begin{itemize}
            \item Some models offer more reliable uncertainty estimates.
            \customFigure[0.5]{../Images/L4_22.png}{}
        \end{itemize}} \\
        \midrule
        \textbf{L1 Regularization Encourages Sparsity} \\
        \multicolumn{1}{p{\linewidth}}{
        \begin{itemize}
            \item Add the sum of absolute values of weights to the loss function:
            \begin{itemize}
                \item Encouraging sparse weights
                \item Feature selection
                \item Loss function with L1 penalty:
            \end{itemize}
            \[
            L(\theta) + \lambda \sum_i |\theta_i|
            \]
        \end{itemize}} \\
        \midrule
        \textbf{L2 Regularization Penalizes Large Weights} \\
        \multicolumn{1}{p{\linewidth}}{
        \begin{itemize}
            \item Add the sum of squared values of weights to the loss function:
            \begin{itemize}
                \item Weight decay
                \item Preventing overfitting
                \item Loss function with L2 penalty:
            \end{itemize}
            \[
            L(\theta) + \lambda \sum_i \theta_i^2
            \]
        \end{itemize}} \\
        \bottomrule
        \end{tabular}
    \end{center}
\end{summary}
\newpage

\subsection{Experiments}
\begin{summary}
    \begin{center}
        \begin{tabular}{ll}
        \toprule
        \textbf{Tools} & \textbf{Description} \\
        \midrule
        \textbf{Logging Metrics During Training} & Track relevant metrics to monitor progress and diagnose issues. \\
        \multicolumn{2}{p{\linewidth}}{
        \begin{center}
            \customFigure[0.5]{../Images/L4_23.png}{}
            \vspace{-4em}
        \end{center}} \\
        \midrule
        \textbf{Setting Up Baselines (Ref. Point)} & Establish a simple models to compare against. \\
        \multicolumn{2}{p{\linewidth}}{
        \begin{itemize}
            \item GLM model, Random forest, XGBoost, NGBoost (if you want uncertainty), SKlearn-like models
        \end{itemize}} \\
        \midrule
        \textbf{Seed Setting} & Set random seeds for reproducibility of results. \\
        \multicolumn{2}{p{\linewidth}}{
        \begin{center}
            \customFigure[0.5]{../Images/L4_26.png}{}
            \vspace{-4em}
        \end{center}} \\
        \bottomrule
        \end{tabular}
    \end{center}
\end{summary}
\newpage

\begin{summary}
    \begin{center}
        \begin{tabular}{ll}
        \toprule
        \textbf{Tools} & \textbf{Description} \\
        \midrule
        \textbf{Ablation Studies Evaluate Systematically One Axis} & Assess impact by controlled single-parameter changes. \\
        \multicolumn{2}{p{\linewidth}}{
        \begin{center}
            \customFigure[0.5]{../Images/L4_24.png}{}
            \vspace{-4em}
        \end{center}} \\
        \midrule
        \textbf{Context Matters: No Free Lunch Theorem} & Best tricks depend on the specific task and dataset. \\
        \midrule
        \textbf{Storytelling Matters} & How you report number and values will change how ppl. think about it.  \\
        % \midrule
        % \textbf{MLP Hyperparameter Space} \\
        % \multicolumn{1}{p{\linewidth}}{
        % \begin{itemize}
        %     \item Better to simplify modelling choices
        %     \customFigure[0.3]{../Images/L4_27.png}{}
        % \end{itemize}} \\
        \bottomrule
        \end{tabular}
    \end{center}
\end{summary}