\documentclass{article}
\usepackage{style}
\title{ECE324 Lectures}
\author{Hanhee Lee}
\lhead{ECE324}
\rhead{Hanhee Lee}

\begin{document}
\section{Problem 1: Description of the Problem, Subscribe a ML/NN Based Solution}
\begin{process}

\end{process}

\begin{definition}
    
\end{definition}

\begin{example}
    
\end{example}

\section{Problem 2: Prescribe a strategy to optimize the NN.}
\begin{process}

\end{process}

\begin{definition}
    
\end{definition}

\begin{example}
    
\end{example}

\section{Problem 3: Explain step by step inference for basic algorithms (MLP, CNN, GNN, attention mechanism) in terms of numpy or basic tensor operations.}
\begin{process}

\end{process}

\begin{definition}
    
\end{definition}

\begin{example}
    
\end{example}

\section{Problem 4: Be able to explain why such a solution might work or fail.} 
\begin{process}

\end{process}

\begin{definition}
    
\end{definition}

\begin{example}
    
\end{example}
\newpage

\section{ML Concepts}
\subsection{Bias-Variance Tradeoff}
\begin{definition}
    \begin{itemize}
        \item \textbf{Bias (Underfitting):} Error due to overly simplistic assumptions in the model.
        \begin{itemize}
            \item A model with high bias makes strong assumptions about the data, leading to oversimplification.
            \item This results in poor training and test performance because the model fails to capture important patterns.
        \end{itemize}
        \item \textbf{Variance (Overfitting):} Error due to the model's sensitivity to fluctuations in the training data.
        \begin{itemize}
            \item A model with high variance is too complex and captures noise in the training data, rather than just the underlying pattern.
            \item This results in excellent training performance but poor generalization to new data.
        \end{itemize}
        \item \textbf{Low Bias, High Variance:} Overly complex models (e.g., deep neural networks with excessive parameters) suffer from overfitting, performing well on training data but poorly on test data.
        \item \textbf{High Bias, Low Variance:} Overly simplistic models (e.g., linear regression) suffer from underfitting, performing poorly on both training and test data.
    \end{itemize}
\end{definition}

\subsubsection{Solution to High Variance}
\begin{definition}
    \begin{itemize}
        \item Regularization
        \item Data Augmentation
        \item More trianing data 
        \item Simpler models
        \item Ensemble methods
    \end{itemize}
\end{definition}

\subsubsection{Solution to High Bias}
\begin{definition}
    \begin{itemize}
        \item More complex models
        \item Feature engineering
        \item Hyperparameter tuning
        \item Reduce regularization
    \end{itemize}
\end{definition}
\newpage

\section{Loss Functions}
\begin{summary}
    \begin{center}
        \begin{tabular}{ll}
            \toprule
            \textbf{Loss Fn} & \textbf{Equation} \\ 
            \toprule
            \textbf{Mean Squared Error (MSE)} & $\frac{1}{n} \sum_{i=1}^{n} (y_i - \hat{y}_i)^2$ \\
            \multicolumn{2}{p{\linewidth}}{
            \begin{itemize}
                \item \textbf{When to Use?} Penalizes large errors.
            \end{itemize}} \\
            \midrule
            \textbf{Root Mean Squared Error (RMSE)} & $\sqrt{\frac{1}{n} \sum_{i=1}^{n} (y_i - \hat{y}_i)^2}$ \\
            \multicolumn{2}{p{\linewidth}}{
            \begin{itemize}
                \item \textbf{When to Use?} Error needs to be in the same units as the target.
            \end{itemize}} \\
            \midrule
            \textbf{Mean Absolute Error (MAE)} & $\frac{1}{n} \sum_{i=1}^{n} |y_i - \hat{y}_i|$ \\
            \multicolumn{2}{p{\linewidth}}{
            \begin{itemize}
                \item \textbf{When to Use?} Equal weighting of all errors (robust to outliers).
            \end{itemize}} \\
            \midrule
            \textbf{Binary Cross Entropy} & $- \frac{1}{n} \sum_{i=1}^{n} y_i \log(\hat{y}_i) + (1 - y_i) \log(1 - \hat{y}_i)$ \\
            \multicolumn{2}{p{\linewidth}}{
            \begin{itemize}
                \item \textbf{When to Use?} Binary classification problems.
            \end{itemize}} \\
            \midrule
            \textbf{Categorical Cross Entropy} & $- \frac{1}{n} \sum_{i=1}^{n} \sum_{j=1}^{m} y_{ij} \log(\hat{y}_{ij})$ \\
            \multicolumn{2}{p{\linewidth}}{
            \begin{itemize}
                \item \textbf{When to Use?} Multi-class classification problems.
            \end{itemize}} \\
            \midrule
            \textbf{AE: Reconstruction Error} & Minimize reconstruction error. \\
            & $L_{AE} = \left\| \mathbf{x} - f_{\text{dec}}(f_{\text{enc}}(\mathbf{x})) \right\|^2 = \left\| \mathbf{x} - \hat{\mathbf{x}} \right\|^2$ \\
            \multicolumn{2}{p{\linewidth}}{
            \begin{itemize}
                \item If our data is binary, what loss function should you use? Binary cross entropy.
            \end{itemize}} \\
            \midrule
            \textbf{VAE: Reconstruction and KL Divergence} & Balances reconstruction and latent space regularization. \\
            &  $L_{\text{VAE}} = L_{\text{AE}} (x, \hat{x}) + \text{KL} (q_\phi (\mathbf{z} \mid \mathbf{x}) \parallel p(\mathbf{z}))$ \\
            \multicolumn{2}{p{\linewidth}}{
            \begin{itemize}
                \item For Gaussian Distribution: $KL(P \| Q) = KL\left(\mathcal{N}(\mu_1, \sigma_1^2) \| \mathcal{N}(\mu_2, \sigma_2^2)\right) = \frac{1}{2} \left[ \log\left(\frac{\sigma_2^2}{\sigma_1^2}\right) + \frac{\sigma_1^2 + (\mu_1 - \mu_2)^2}{\sigma_2^2} - 1 \right]$ 
            \end{itemize}} \\
            \bottomrule
        \end{tabular}
    \end{center}
\end{summary}
\newpage

\section{Algorithms}
\begin{summary}
    \begin{center}
        \begin{tabular}{llll}
            \toprule
            \textbf{Algorithm} & \textbf{Inputs} & \textbf{Outputs} & \textbf{Equations} \\
            \toprule
            PCA & $x$ & $x'$ & $x \underset{\text{Encode}}{\mapsto} z \underset{\text{Decode}}{\mapsto} x'$ \\
            & & & $\text{Encoder}(x) = Wx$, $\text{Decoder}(z) = W^{-1}z$ \\
            \multicolumn{4}{p{\linewidth}}{
            \begin{itemize}
                \item \textbf{Overview:} Linear transformation for dimension reduction 
            \end{itemize}} \\
            \midrule
            AE & $x$ & $x'$ & $x \underset{\text{Encode}}{\mapsto} z \underset{\text{Decode}}{\mapsto} x'$ \\
            & & & $f_{\text{enc}}(x), \; \text{Encoder}(x) = \text{Neural network}$, $f_{\text{dec}}(z), \; \text{Decoder}(z) = \text{Neural network}$ \\
            \multicolumn{4}{p{\linewidth}}{
            \begin{itemize}
                \item Bottleneck (latent space) forces network to learn compressed representation
                \item Encoder maps input to lower dimension, decoder reconstructs input
            \end{itemize}} \\
            \midrule
            VAE & $x$ & $x'$ & $x \underset{\text{Encode}}{\mapsto} z \underset{\text{Decode}}{\mapsto} x'$ \\
            & & & $f_{\text{enc}, \phi}(\mathbf{z} \mid \mathbf{x}) = \mathcal{N}(\mathbf{z} \mid \boldsymbol{\mu}(\mathbf{x}), \sigma^2(\mathbf{x}) \mathbf{I})$, $f_{\text{dec}}(\mathbf{z})$\\ 
            \multicolumn{4}{p{\linewidth}}{
            \begin{itemize}
                \item \textbf{Overview:} A probabilistic approach by learning smooth latent spaces by modelling probability distributions.
                \item $\boldsymbol{\mu}(\mathbf{x})$: Mean of the distribution, $\sigma^2(\mathbf{x})$: Variance of the distribution (independent)
                \item Encoder maps input to a distribution, decoder samples from it.
                \begin{itemize}
                    \item Encoder estimates parameters for a distribution in the latent space
                    \item Decoder samples from the distribution to generate an output sample
                \end{itemize}
                \item \textbf{Key:} Works on any data.
            \end{itemize}} \\
            \midrule
            RNN & $x_t,h_{t-1}$ & $y_{t},h_{t}$ & $h_t = \text{tanh}(\text{Linear} (h_{t-1}) + \text{Linear}(x_t))$ \\ 
            & & & $y_t = \text{MLP}(h_t)$ \\
            \multicolumn{4}{p{\linewidth}}{
            \begin{itemize}
                \item $x_t$: Input, $h_t$: Hidden state, $y_t$: Output
            \end{itemize}} \\
            \midrule
            GRU & $x_t,h_{t-1}$ & $y_t,h_t$ & $z_t = \text{sigmoid}(\text{Linear}(x_t) + \text{Linear}(h_{t-1}))$ \\
            & & & $r_t = \text{sigmoid}(\text{Linear}(x_t) + \text{Linear}(h_{t-1}))$ \\
            & & & $\tilde{h}(t) = \text{tanh}(\text{Linear}(x_t) + \text{Linear}(r_t \odot h_{t-1}))$ \\
            & & & $h_t = (1-z_t) \odot h_{t-1} + z_t \odot \tilde{h}_t$ \\
            \multicolumn{4}{p{\linewidth}}{
            \begin{itemize}
                \item $z_t$: Update gate, $r_t$: Reset gate
                \item $h_t$: Hidden state
            \end{itemize}} \\
            \midrule
            LSTM & $x_t,h_{t-1}$ & $h_t,c_t$ & $f_t = \text{sigmoid}(\text{Linear}(x_t) + \text{Linear}(h_{t-1}))$ \\
            & & & $i_t = \text{sigmoid}(\text{Linear}(x_t) + \text{Linear}(h_{t-1}))$ \\
            & & & $o_t = \text{sigmoid}(\text{Linear}(x_t) + \text{Linear}(h_{t-1}))$ \\
            & & & $\tilde{c}_t = \text{tanh}(\text{Linear}(x_t) + \text{Linear}(h_{t-1}))$ \\
            & & & $c_t = f_t \odot c_{t-1} + i_t \odot \tilde{c}_t$ \\
            & & & $h_t = o_t \odot \text{tanh}(c_t)$ \\
            \multicolumn{4}{p{\linewidth}}{
            \begin{itemize}
                \item $f_t$: Forget gate, $i_t$: Input/Update gate, $o_t$: Output gate
                \item $h_t$: Hidden state, $c_t$: Cell state
            \end{itemize}} \\
            \bottomrule
        \end{tabular}
    \end{center}
\end{summary}

\subsection{Geometric DL Blueprint for NNs}
\begin{summary}
    Unify various networks around symmetry. 
    ADD IMAGE
\end{summary}
\newpage

\section{Intro to ML}
\begin{summary}
    \begin{center}
        \begin{tabular}{llccccc}
        \toprule
        \textbf{Alg.} & Halting & Sound & Complete & Optimal & Time & Space \\
        \midrule
        \multicolumn{7}{p{\linewidth}}{
            Choose \texttt{REMOVE(·)} so algo. exhibits the characteristics:
            \begin{itemize}
                \item \textbf{Halting:} Terminates after finitely many nodes explored $\mid$ \textbf{Sound:} Returned (possibly NULL) soln. is correct 
                \item \textbf{Complete:} Halting \& sound when a non-NULL soln. exists $\mid$ \textbf{Opt.:} Returns an opt. soln. when mult. exist 
                \item \textbf{Time:} Minimizes nodes \textbf{explored}/expanded/exported $\mid$ \textbf{Space:} Minimizes nodes simultaneously open
            \end{itemize}} \\
        \midrule
        \multicolumn{7}{p{\linewidth}}{
            Choose \texttt{REMOVE($\cdot$)} so algo. exhibits the characteristics for as many path trees as possible. 
            \begin{itemize}
                \item $b$ ($b < \infty$): Branching factor (the maximum number of children a node can have)
                \item $d$: Depth (the length of the longest path), $l^*$: Length of the shortest solution
                \item $c^*$: Cost of the cheapest solution, $\epsilon$: Cost of the cheapest edge
            \end{itemize}} \\
        \midrule
        \multicolumn{7}{p{\linewidth}}{
        \begin{center}
            \textbf{Uninformed Search Algorithms}
        \end{center}} \\
        \midrule
        \textbf{BFS} & $d<\infty \mid$ non-\texttt{NULL} soln. & always & always & constant cst & $b^{l^{*}}$ & $b^{l^{*} + 1}$ \\
        \multicolumn{7}{p{\linewidth}}{
        \begin{itemize}
            \item Explores the least-recently expanded open node first.
        \end{itemize}} \\
        \midrule
        \textbf{DFS} & $d<\infty$ & always & $d < \infty$ & never & $b^{d}$ & $bd$ \\
        \multicolumn{7}{p{\linewidth}}{
        \begin{itemize}
            \item Explores the most-recently expanded open node first.
        \end{itemize}} \\
        \midrule
        \textbf{IDDFS} & always & always & always & constant cst & $b^{l^{*}}$ & $bl^{*}$ \\
        \multicolumn{7}{p{\linewidth}}{
        \begin{itemize}
            \item Same as DFS but with iterative deepening.
        \end{itemize}} \\
        \midrule
        \textbf{CFS} & $d<\infty \mid$ non-\texttt{NULL} soln. & yes & $\epsilon >0$ & $\epsilon >0$ & $b^{c^{*} / \epsilon}$ & $b^{c^{*}/\epsilon + 1}$ \\
        \multicolumn{7}{p{\linewidth}}{
        \begin{itemize}
            \item Explores the cheapest open node first.
        \end{itemize}} \\
        \midrule
        \multicolumn{7}{p{\linewidth}}{
        \begin{center}
            \textbf{Informed Search Algorithms}
        \end{center}} \\
        \midrule
        \textbf{HFS} & $d<\infty$ & never & never & never & - & - \\
        \multicolumn{7}{p{\linewidth}}{
        \begin{itemize}
            \item Explores the node with the smallest hur-value first, $\text{ecst}(p) = \text{hur}(p)$
        \end{itemize}} \\
        \midrule
        \textbf{A$^*$} & hur admissible, $\epsilon > 0$ & always & hur admissible, $\epsilon > 0$ & hur admissible, $\epsilon > 0$ & $O\left(b^{c^{*}/\epsilon}\right)$ & $O\left(b^{c^{*}/\epsilon + 1}\right)$ \\
        \multicolumn{7}{p{\linewidth}}{
        \begin{itemize}
            \item Explores the node with the smallest ecst-value first, $\text{ecst}(p) = \text{cst}(p) + \text{hur}(p)$
        \end{itemize}} \\
        \midrule
        \textbf{IIA$^*$} & always & always & always & always & $b^{l^{*}}$ & $bl^{*} $ \\
        \multicolumn{7}{p{\linewidth}}{
        \begin{itemize}
            \item Same as A$^*$ but with iterative inflating on ecst.
        \end{itemize}} \\
        \midrule
        \textbf{WA$^*$} & - & - & - & - & - & - \\
        \multicolumn{7}{p{\linewidth}}{
            \begin{itemize}
                \item Same as A$^*$ but $\text{ecst}(s) = \text{wcst}(s) + (1-w)\text{hur}(s)$ w/ $w \in [0,1]$
                \item $w=0$: HFS, $w=0.5$: A$^*$, $w=1$: CFS, iteratively increasing $w$ from 0 to 1: anytime version of $WA^*$
            \end{itemize}} \\
        \bottomrule
        \end{tabular}
    \end{center}
\end{summary}
\newpage

\subsection{Modifications to Search Algorithms:}
\begin{summary}
    \begin{center}
        \begin{tabular}{l}
        \toprule
        \textbf{Modifications} \\
        \midrule
        \textbf{Depth-Limiting} \\
        \multicolumn{1}{p{\linewidth}}{
        \begin{itemize}
            \item Enforce a depth limit, $d_{\text{max}}$, to any search algorithm.
        \end{itemize}} \\
        \midrule
        \textbf{Iterative-Deepening} \\
        \multicolumn{1}{p{\linewidth}}{
        \begin{itemize}
            \item Iteratively increase the depth-limit to any search algorithm w/ depth-limiting.
        \end{itemize}} \\
        \midrule
        \textbf{Cost-Limiting} \\
        \multicolumn{1}{p{\linewidth}}{
        \begin{itemize}
            \item Enforce a cost limit of $c_\text{max}$ to any search algorithm.
        \end{itemize}} \\
        \midrule
        \textbf{Iterative Inflating} \\
        \multicolumn{1}{p{\linewidth}}{
        \begin{itemize}
            \item Iteratively increase the cost limit, $c_{\text{max}}$, to any search algorithm w/ cost-limiting.
        \end{itemize}} \\
        \midrule
        \textbf{Intra-Path Cycle Checking} \\
        \multicolumn{1}{p{\linewidth}}{
        \begin{itemize}
            \item Do not expand a path if it is cyclic.
        \end{itemize}} \\
        \midrule
        \textbf{Inter-Path Cycle Checking} \\
        \multicolumn{1}{p{\linewidth}}{
        \begin{itemize}
            \item Do not expand a path if its destination is that of an explored path. 
        \end{itemize}} \\
        \bottomrule
        \end{tabular}
    \end{center}
\end{summary}
\newpage

\subsection{Setup}
\begin{definition} In a search problem, it is assumed that: 
    \begin{itemize}
        \item There is only one agent (us).
        \item For each state, $s \in S$, we have a discrete set of actions, $\mathcal{A}(s)$.
        \item The transition resulting from a move, $(s, a)$, is deterministic; the resulting state is $tr(s, a)$.
        \item $cst(s, a, tr(s, a))$ is our cost for the transition, $(s, a, tr(s, a))$.
        \item We want to realize a path that minimizes our cost.
    \end{itemize}
    
    A search problem may have no solutions, in which case, we define the solution as \texttt{NULL}.
\end{definition}

\begin{warning}
    A NULL solution is not the same as $p = \langle \rangle$ (an empty solution w/ $s^{(0)} \in \mathcal{G}$).
\end{warning}

\subsection{Search Graphs}
\begin{definition}
    In a search graph (a graph representing a search problem):
    \begin{itemize}
        \item $S$ is defined by the vertices.
        \item $\mathcal{G}$ is a subset of the vertices.
        \item $s^{(0)}$ is some vertex.
        \item $tr(\cdot, \cdot)$ and $\mathcal{T}$ are defined by the edges.
        \item $cst(\cdot, \cdot, \cdot)$ is defined by the edge weights.
    \end{itemize}
\end{definition}

\subsection{Path Trees}
\begin{definition}
    A search algorithm explores a tree of possible paths. 
    \begin{itemize}
        \item In such a tree, each node represents the path from the root to itself.
        \begin{itemize}
            \item The node may also include other info (such as the path's origin, cost, etc).
        \end{itemize}
    \end{itemize}
\end{definition}

\subsection{Search Algorithms}
\begin{algo}
    All search algorithms follow the template below:

\begin{lstlisting}
$\mathcal{O} \gets \{(\langle \rangle, 0)\}$ (*\hfill $\triangleright$ initialize a set of open nodes*) 
SEARCH($\mathcal{O}$)
\end{lstlisting}
\begin{itemize}
    \item $\langle \rangle$: Empty path, $0$: Cost of empty path.
\end{itemize}

\begin{lstlisting}
procedure SEARCH($\mathcal{O}$)
    if $\mathcal{O} = \emptyset$ then
        return NULL  (*\hfill $\triangleright$ the search algorithm failed to find a path to a goal*)
    $n \gets \textsc{Remove}(\mathcal{O})$ (*\hfill $\triangleright$ "explore" a node $n$*)
    if $\textsc{dst}(n) \in \mathcal{G}$ then
        return $n$ (*\hfill $\triangleright$ the search algorithm found a path to a goal*)
    for $n' \in \textsc{chl}(n)$ do (*\hfill $\triangleright$ "expand" $n$ and "export" its children*)
        $\mathcal{O} \gets \mathcal{O} \cup \{n'\}$ 
    SEARCH($\mathcal{O}$)
\end{lstlisting}
\begin{itemize}
    \item Explore: Remove a node from the open set.
    \item Expand: Generate the children of the node.
    \item Export: Add the children to the open set.
\end{itemize}

\end{algo}

\begin{warning}
    The key difference is in the order that \textsc{Remove}($\cdot$) removes nodes.
\end{warning}

\subsection{Modifications to Search Algorithms}
\subsubsection{Depth-Limiting}
\begin{algo}
\begin{lstlisting}
procedure SEARCHDL($\mathcal{O}$, $d_{\text{max}}$):
    if $\mathcal{O} = \emptyset$ then
        return NULL (*\hfill $\triangleright$ the search algorithm failed to find a path to a goal*)
    $n \leftarrow \text{REMOVE}(\mathcal{O})$ (*\hfill $\triangleright$ "explore" a node, $n$*)
    if dst($n$) $\in \mathcal{G}$ then
        return $n$ (*\hfill $\triangleright$ the search algorithm found a path to a goal*)
    for $n' \in \text{chl}(n)$ do (*\hfill $\triangleright$ "expand" $n$ and "export" its children*)
        if len($n'$) $\leq d_{\text{max}}$ then (*\hfill $\triangleright$ unless the child is too long*)
            $\mathcal{O} \leftarrow \mathcal{O} \cup \{n'\}$
    SEARCHDL($\mathcal{O}$, $d_{\text{max}}$)
\end{lstlisting}

\end{algo}

\subsubsection{Iterative Deepening}
\begin{algo}
\begin{lstlisting}
procedure SEARCHID():
    $n \leftarrow \text{NULL}$
    $d_{\text{max}} = 0$
    (*$\triangleright$ while a solution has not been found, reset the open set, run the search algorithm, then increase the depth-limit*)
    while $n = \text{NULL}$ do
        $\mathcal{O} \leftarrow \{(\langle \rangle, 0)\}$
        $n \leftarrow \text{SEARCHDL}(\mathcal{O}, d_{\text{max}})$
        $d_{\text{max}} \leftarrow d_{\text{max}} + 1$
    return $n$
\end{lstlisting}
    
\end{algo}

\begin{warning}
    Increasing $d_{\text{max}}$ can be done in different ways.
\end{warning}

\subsubsection{Cost-Limiting}
\begin{algo}

\begin{lstlisting}
procedure SEARCHCL($\mathcal{O}$, $c_{\text{max}}$):
    if $\mathcal{O} = \emptyset$ then
        return NULL (*\hfill $\triangleright$ the search algorithm failed to find a path to a goal*)
    $n \leftarrow \text{REMOVE}(\mathcal{O})$ (*\hfill $\triangleright$ "explore" a node, $n$*)
    if dst($n$) $\in \mathcal{G}$ then
        return $n$ (*\hfill $\triangleright$ the search algorithm found a path to a goal*)
    for $n' \in \text{chl}(n)$ do (*\hfill $\triangleright$ "expand" $n$ and "export" its children*)
        if cst($n'$) $\leq c_{\text{max}}$ then (*\hfill $\triangleright$ unless the child is too expensive*)
            $\mathcal{O} \leftarrow \mathcal{O} \cup \{n'\}$
    SEARCHCL($\mathcal{O}$, $c_{\text{max}}$)
\end{lstlisting}

\end{algo}
\newpage

\subsubsection{Iterative-Inflating}
\begin{algo}
\begin{lstlisting}
procedure SEARCHII():
    $n \leftarrow \text{NULL}$
    $c_{\text{max}} = 0$
    (*$\triangleright$ while a solution has not been found, reset the open set, run the search algorithm, then increase the cost-limit*)
    while $n = \text{NULL}$ do
        $\mathcal{O} \leftarrow \{(\langle \rangle, 0)\}$
        $n \leftarrow \text{SEARCHCL}(\mathcal{O}, c_{\text{max}})$
        $c_{\text{max}} \leftarrow c_{\text{max}} + \epsilon$
    return $n$
\end{lstlisting}

\end{algo}

\begin{warning}
    Increasing $c_{\text{max}}$ can be done in different ways.
\end{warning}

\subsubsection{Intra-Path Cycle Checking}
\begin{algo}
\begin{lstlisting}
procedure SEARCH($\mathcal{O}$):
    if $\mathcal{O} = \emptyset$ then
        return NULL
    $n \leftarrow \text{REMOVE}(\mathcal{O})$
    if dst($n$) $\in \mathcal{G}$ then
        return $n$
    for $n' \in \text{chl}(n)$ do (*\hfill $\triangleright$ "expand" $n$ and "export" its children*)
        if not CYCLIC($n'$) then (*\hfill $\triangleright$ unless the child is cyclic*)
            $\mathcal{O} \leftarrow \mathcal{O} \cup \{n'\}$
    SEARCH($\mathcal{O}$)
\end{lstlisting}
\begin{itemize}
    \item Optimately of an algorithm is preserved provided $\epsilon>0$.
\end{itemize}
\end{algo}

\subsubsection{Inter-Path Cycle Checking}
\begin{algo}
\begin{lstlisting}
procedure SEARCH($\mathcal{O}$, $\mathcal{C}$):
    if $\mathcal{O} = \emptyset$ then
        return NULL
    $n \leftarrow \text{REMOVE}(\mathcal{O})$
    $\mathcal{C} \leftarrow \mathcal{C} \cup \{n\}$ (*\hfill $\triangleright$ add $n$ to the closed set*)
    if dst($n$) $\in \mathcal{G}$ then
        return $n$
    for $n' \in \text{chl}(n)$ do (*\hfill $\triangleright$ "expand" $n$ and "export" its children*)
        if $n' \notin \mathcal{C}$ then (*\hfill $\triangleright$ unless the child's destination is closed*)
            $\mathcal{O} \leftarrow \mathcal{O} \cup \{n'\}$
    SEARCH($\mathcal{O}$, $\mathcal{C}$)
\end{lstlisting}

and then call the algorithm as follows:

\begin{lstlisting}[mathescape=true, escapeinside={(*}{*)}, numbers=left, frame=single]
$\mathcal{O} \leftarrow \{(\langle \rangle, 0)\}$
$\mathcal{C} \leftarrow \{\}$ (*\hfill $\triangleright$ initialize a set of closed vertices*)
SEARCH($\mathcal{O}$, $\mathcal{C}$)
\end{lstlisting}

\end{algo}
\newpage

\subsection{Informed Search Algorithms}
\subsubsection{Estimated Cost}
\begin{definition}
    $\text{ecst}(\cdot)$: estimate the total cost to a goal given a path, $p$, based on:
    \begin{itemize}
        \item $\text{cst}(p)$: Cost of path $p$
        \item $\text{hur}: S \to \mathbb{R}_+$: Estimate of the extra cost needed to get to a goal from $\text{dst}(p)$
        \begin{itemize}
            \item $\text{hur}(s)$ estimates the cost to get to $\mathcal{G}$ from $s$ and $\text{hur}(p)$ means $\text{hur}(\text{dst}(p))$.
            \item $\text{hur}^*(s)$: The true cost to get to $\mathcal{G}$ from $s$.
        \end{itemize}
    \end{itemize}
\end{definition}

\subsubsection{Admissible}
\begin{motivation}
    We want to find a heuristic that under estimates (i.e. make paths look better than they are) the costs, rather than over estimate (i.e. make paths look worse than they are).
    \begin{itemize}
        \item Least useful heuristic: $\text{hur}(s) = 0$ for all $s \in \mathcal{S}$ or any other constant.
        \item Most useful heuristic: $\text{hur}(s) = \text{hur}^*(s)$ for all $s \in \mathcal{S}$.
    \end{itemize}
\end{motivation}
\begin{definition}
    A heuristic, $\text{hur}(\cdot)$, is said to be \textbf{admissible} if

    \begin{equation*}
        \text{hur}(s) \leq \text{hur}^*(s)
    \end{equation*}

    for all $s \in \mathcal{S}$ and

    \begin{equation*}
        \text{hur}(s) = 0
    \end{equation*}

    for all $s \in \mathcal{G}$.
\end{definition}

\begin{warning}
    Never over-estimates the overall cost, but may still estimate the cost of individual transition. 
\end{warning}

\subsubsection{Consistent}
\begin{definition}
    A heuristic, $\text{hur}(\cdot)$, is said to be \textbf{consistent} if

    \begin{equation*}
        \underbrace{\text{hur}(s) - \text{hur}(\text{tr}(s,a))}_{\text{estimated cost of the transition }(s,a,\text{tr}(s,a))}
        \leq 
        \underbrace{\text{cst}(s,a,\text{tr}(s,a))}_{\text{true cost of the transition, }(s,a,\text{tr}(s,a))}
    \end{equation*}

    for all $s \in \mathcal{S}$, and $a \in \mathcal{A}(s)$, and

    \begin{equation*}
        \text{hur}(s) = 0
    \end{equation*}

    for all $s \in \mathcal{G}$.
\end{definition}

\begin{warning}
    Never over-estimates the cost of individual transitions (and hence the overall cost).
\end{warning}

\begin{theorem}
    If a heuristic, $\text{hur}(\cdot)$, is consistent, then it is also admissible.
\end{theorem}
\newpage

\subsubsection{Domination}
\begin{definition}
    If $\text{hur}_1$ and $\text{hur}_2$ are admissible, then:
    \begin{itemize}
        \item $\text{hur}_1$ \textbf{strongly dominates} $\text{hur}_2$ if for all $s \in \mathcal{S} \setminus \mathcal{G}$:
        \begin{equation*}
            \text{hur}_1(s) > \text{hur}_2(s)
        \end{equation*}

        \item $\text{hur}_1$ \textbf{weakly dominates} $\text{hur}_2$ if for all $s \in \mathcal{S}$:
        \begin{equation*}
            \text{hur}_1(s) \geq \text{hur}_2(s)
        \end{equation*}
        and for some $s \in \mathcal{S}$:
        \begin{equation*}
            \text{hur}_1(s) > \text{hur}_2(s)
        \end{equation*}
    \end{itemize}
\end{definition}

\begin{notes}
    Want the heuristic that dominates but is also admissible.
\end{notes}

\subsubsection{Designing Heuristics via Problem Relaxation}
\begin{definition}
    Let $\text{hur}^*_{\text{ori}}$ be the perfect heuristic for a search problem, and $\text{cst}^*_{\text{rel}}$ be the optimal cost for a relaxed version of the problem. Then
    \[
    \text{cst}^*_{\text{rel}}(s) \leq \text{hur}^*_{\text{ori}}(s) \text{ for all } s \in \mathcal{S}.
    \]
\end{definition}

\subsubsection{Combining Heuristics}
\begin{definition}
    If $\{ \text{hur}_k(\cdot) \}_k$ are admissible (or consistent), then $\max_k \{\text{hur}_k\} (\cdot)$ is also admissible (or consistent).
\end{definition}

\begin{definition}
    If $\text{hur}_{\max} \equiv \max \{\text{hur}_1, \text{hur}_2\}$, then if $\text{hur}_k$ is consistent:
    \[
    \text{hur}_k(s) - \text{hur}_k(\text{tr}(s,a)) \leq \text{cst}(s,a,\text{tr}(s,a))
    \]
    \[
    \text{hur}_{\max} (s)= \text{hur}_{\max}(\text{tr}(s,a)) - \text{cst}^*(s,a,\text{tr}(s,a))
    \]
\end{definition}

\subsubsection{Anytime Search Algorithms}
\begin{definition}
    An \textbf{anytime algorithm} finds a solution quickly (even if it is sub-optimal), and then iteratively improves it (if time permits).
\end{definition}
\newpage

\subsection{Canonical Examples}
\subsubsection{How to setup a search problem?}
\begin{process} 
    \begin{enumerate}
        \item Given a search graph, we need to define the following:
        \begin{itemize}
            \item $\mathcal{S}$: set of vertices
            \item $\mathcal{G}$: goal states (subset of $\mathcal{S}$)
            \item $s^{(0)}$: initial state
            \item $\mathcal{T}$: set of edges (defined by $\text{tr}(\cdot, \cdot)$)
            \begin{itemize}
                \item $\text{tr}(\cdot, \cdot)$: transition function
            \end{itemize}
            \item $\text{cst}(\cdot, \cdot, \cdot)$: cost function (defined by edge weights)
        \end{itemize}
    \end{enumerate}
\end{process}

\begin{example}
    \customFigure[0.5]{../Images/L2_0.png}{}
    \customFigure[0.5]{../Images/L2_1.png}{}
\end{example}
\newpage

\begin{example}
    \customFigure[0.5]{../Images/L2_9.png}{}
    \customFigure[0.5]{../Images/L2_10.png}{}
    \begin{itemize}
        \item $\mathcal{S} = \{0,\ldots,4 \}^2$
        \item $\mathcal{G} = \left\{ \begin{bmatrix}
            1 \\
            4
        \end{bmatrix} \right\}$
        \item $s^{(0)} = \begin{bmatrix}
            1 \\
            0
        \end{bmatrix}$
    \end{itemize}
\end{example}
\newpage

\subsubsection{How to setup a path tree?}
\begin{process} 
    \begin{enumerate}
        \item Start at $s^{(0)}$
        \item Choose a path until you reach a goal state.
        \item Repeat until you have found all paths (probably infinite).
    \end{enumerate}
\end{process}

\begin{example}
    \customFigure[0.5]{../Images/L2_2.png}{}
    \customFigure[0.5]{../Images/L2_3.png}{}
\end{example}

\newpage

\subsubsection{When to use each algorithm?}
\begin{process} 
    \begin{enumerate}
        \item Do we have a heuristic?
        \begin{itemize}
            \item \textbf{Yes:} Use informed search algorithms.
            \item \textbf{No:} Use uninformed search algorithms.
        \end{itemize}
        \item  Are path costs non-uniform?
        \begin{itemize}
            \item \textbf{Yes:} Eliminate BFS.
            \item \textbf{No:} Eliminate CFS, $A*$
        \end{itemize}
        \item 
        \item Is the search space finite or infinite? 
        \begin{itemize}
            \item \textbf{Finite:} Use any algorithm.
            \item \textbf{Infinite:} Use BFS, IDDFS, CFS, or A*.
        \end{itemize}
        \item Do we need to guarantee finding a solution (completeness)?
        \begin{itemize}
            \item \textbf{Yes:} Use BFS, IDDFS, IIA*, CFS (if $\epsilon > 0$).
            \item \textbf{No:} Use DFS, HFS, WA*
        \end{itemize}
        \item Find properties needed for the problem and match them to the characteristics of the algorithm.
        \item Choose the algorithm that best matches the properties.
        \begin{itemize}
            \item \textbf{BFS:} Need shortest path in an unweighted graph. 
            \item \textbf{DFS:} Explore a deep path quickly, and completeness is not needed.
            \item \textbf{IDDFS:} Want completeness of BFS but with the complexity of DFS.
            \item \textbf{CFS:} Need the least-cost path in a weighted graph. 
            \item \textbf{HFS:}
            \item \textbf{A*:} 
            \item \textbf{IIA*:}
            \item \textbf{WA*:}
        \end{itemize}
    \end{enumerate}
\end{process}

\begin{example}
    
\end{example}
\newpage

\subsubsection{Tracing Search Algorithms}
\begin{example}
    \customFigure[0.5]{../Images/LR_0.png}{}
\end{example}
\newpage

\begin{process} \textbf{BFS}
    \begin{enumerate}
        \item Start at $s_0$ as \textbf{current node}
        \item Expand all neighboring nodes of the \textbf{current node} and add them to the open set (queue).
        \item Remove the \textbf{current node} from the open set and add it to the path. 
        \item Choose the least-recently expanded node from the open set as the \textbf{current node}.
        \item Repeat steps 2 and 4 until the goal state is reached or the open set is empty.
    \end{enumerate}
\end{process}

\begin{example} \textbf{BFS}
    \begin{center}
        \begin{tabular}{ll}
        \toprule
        \textbf{Path} & \textbf{Open Set} \\
        \midrule
         & $\{A\}$ \\
        $A$ & $\{AB, AC, AD\}$ \\
        $AB$ & $\{AC, AD, ABA, ABC\}$ \\
        $AC$ & $\{AD, ABA, ABC, ACD, ACE\}$ \\
        $AD$ & $\{ABA, ABC, ACD, ACE, ADA, ADE\}$ \\
        $ABA$ & $\{ABC, ACD, ACE, ADA, ADE, ABAB, ABAC, ABAD\}$ \\
        $ABC$ & $\{ACD, ACE, ADA, ADE, ABAB, ABAC, ABAD, ABCD, ABCE\}$ \\
        $ACD$ & $\{ACE, ADA, ADE, ABAB, ABAC, ABAD, ABCD, ABCE, ACDA, ACDE\}$ \\
        $ACE$ & $\{ADA, ADE, ABAB, ABAC, ABAD, ABCD, ABCE, ACDA, ACDE\}$ \\
        \bottomrule
        \end{tabular}
    \end{center}
    \vspace{1em}

    \textbf{Intra:}
    \begin{center}
        \begin{tabular}{ll}
        \toprule
        \textbf{Path} & \textbf{Open Set} \\
        \midrule
         & $\{A\}$ \\
        $A$ & $\{AB, AC, AD\}$ \\
        $AB$ & $\{AC, AD, ABC, \cancel{ABA}\}$ \\
        $AC$ & $\{AD, ABC, ACD, ACE\}$ \\
        $AD$ & $\{ABC, ACD, ACE, ADE, \cancel{ADA}\}$ \\
        $ABC$ & $\{ACD, ACE, ADE, ABCD, ABCE\}$ \\
        $ACD$ & $\{ACE, ADE, ABCD, ABCE, ACDE, \cancel{ACBA}\}$ \\
        $ACE$ & $\{ADE, ABCD, ABCE, ACDE\}$ \\
        \bottomrule
        \end{tabular}
    \end{center}
    \vspace{1em}

    \textbf{Inter:}
    \begin{center}
        \begin{tabular}{lll}
        \toprule
        \textbf{Path} & \textbf{Open Set} & \textbf{Closed Set} \\
        \midrule
        - & $\{A\}$ & - \\
        $A$ & $\{AB, AC, AD\}$ & $\{A\}$ \\
        $AB$ & $\{AC, AD, ABC, \cancel{ABA}\}$ & $\{A, B\}$ \\
        $AC$ & $\{AD, ABC, ACD, ACE\}$ & $\{A, B, C\}$ \\
        $AD$ & $\{ABC, ACD, ACE, ADE, \cancel{ADA}\}$ & $\{A, B, C, D\}$ \\
        $ABC$ & $\{ACD, ACE, ADE, ABCE, \cancel{ABCD}\}$ & $\{A, B, C, D\}$ \\
        $ACD$ & $\{ACE, ADE, ABCE, ACDE, \cancel{ACDA}\}$ & $\{A, B, C, D\}$ \\
        $ACE$ & $\{ADE, ABCE, ACDE\}$ & $\{A, B, C, D, E\}$ \\
        \bottomrule
        \end{tabular}
    \end{center}
\end{example}
\newpage



\begin{process} \textbf{DFS}
    \begin{enumerate}
        \item Start at $s_0$ as \textbf{current node}
        \item Expand all neighboring nodes of the \textbf{current node} and add them to the open set (stack).
        \item Remove the \textbf{current node} from the open set and add it to the path. 
        \item Choose the most-recently expanded node from the open set as the \textbf{current node}.
        \item Repeat steps 2 and 4 until the goal state is reached or the open set is empty.
    \end{enumerate}
\end{process}

\begin{example} \textbf{DFS}
    \begin{center}
        \begin{tabular}{ll}
        \toprule
        \textbf{Path} & \textbf{Open Set} \\
        \midrule
         & $\{A\}$ \\
        $A$ & $\{AB,AC,AD\}$ \\
        $AD$ & $\{AB,AC,ADA,ADE\}$ \\
        $ADE$ & $\{AB, AC,ADA\}$ \\
        \bottomrule
        \end{tabular}
    \end{center}
    \vspace{1em}

    \textbf{Intra:}
    \begin{center}
        \begin{tabular}{lll}
        \toprule
        \textbf{Path} & \textbf{Open Set} \\
        \midrule
        - & $\{A\}$ \\
        $A$ & $\{AB,AC,AD\}$ \\
        $AD$ & $\{AB,AC,ADE, \cancel{ADA}\}$ \\
        $ADE$ & $\{AB, AC\}$ \\
        \bottomrule
        \end{tabular}
    \end{center}
    \vspace{1em}

    \textbf{Inter:}
    \begin{center}
        \begin{tabular}{lll}
        \toprule
        \textbf{Path} & \textbf{Open Set} & \textbf{Closed Set} \\
        \midrule
        - & $\{A\}$ & - \\
        $A$ & $\{AB,AC,AD\}$ & $\{A\}$ \\
        $AD$ & $\{AB,AC,ADE, \cancel{ADA}\}$ & $\{A, D\}$ \\
        $ADE$ & $\{AB, AC\}$ & $\{A, D, E\}$ \\
        \bottomrule
        \end{tabular}
    \end{center}
\end{example}
\newpage

\begin{process} \textbf{IDDFS}
    \begin{enumerate}
        \item Start with a depth limit of 0.
        \item Perform DFS up to the current depth limit.
        \item If the goal state is not reached, increment the depth limit based on given fcn and repeat step 2.
        \item Continue until the goal state is found or all nodes are explored.
    \end{enumerate}
\end{process}

\begin{example} \textbf{IDDFS}
    \begin{center}
        \begin{tabular}{lll}
        \toprule
        \textbf{Depth} & \textbf{Path} & \textbf{Open Set} \\
        \midrule
        0 & & $\{A\}$ \\
        0 & A & $\{\}$ \\
        \midrule
        1 & $A$ & $\{AB, AC, AD\}$ \\
        1 & $AD$ & $\{AB, AC\}$ \\
        1 & $AC$ & $\{AB\}$ \\
        1 & $AB$ & $\{\}$ \\
        \midrule
        2 & $AB$ & $\{ABA, ABC\}$ \\
        2 & $ABC$ & $\{ABA\}$ \\
        2 & $ABA$ & $\{\}$ \\
        \midrule
        3 & $ABA$ & $\{ABAB, ABAC, ABAD\}$ \\
        3 & $ABAB$ & $\{ABAC, ABAD\}$ \\
        3 & $ABAC$ & $\{ABAD\}$ \\
        3 & $ABAD$ & $\{\}$ \\
        \midrule
        4 & $ABAD$ & $\{ABADA, ABADE\}$ \\
        4 & $ABADA$ & $\{ABADE\}$ \\
        4 & $ABADE$ & $\{\}$ \\
        \bottomrule
        \end{tabular}
    \end{center}
\end{example}
\newpage

\begin{process} \textbf{CFS}
    \begin{enumerate}
        \item Start at $s_0$ as \textbf{current node}
        \item Expand all neighboring nodes of the \textbf{current node} and add them to the open set (priority queue).
        \item Remove the \textbf{current node} from the open set and add it to the path. 
        \item Choose the cheapest expanded node from the open set as the \textbf{current node}.
        \item Repeat steps 2 and 4 until the goal state is reached or the open set is empty.
    \end{enumerate}
\end{process}

\begin{example} \textbf{CFS}
    \begin{center}
        \begin{tabular}{ll}
        \toprule
        \textbf{Path} & \textbf{Open Set} \\
        \midrule
        - & $\{A \mid 0\}$ \\
        $A$ & $\{AB \mid 2, \; AC \mid 4, \; AD \mid 6\}$ \\
        $AB$ & $\{AC \mid 4, \; AD \mid 6, \; ABC \mid 3, \; ABA \mid 4\}$ \\
        $ABC$ & $\{AC \mid 4, \; AD \mid 6, \; ABA \mid 4, \; ABCE \mid 5, \; ABCD \mid 6\}$ \\
        $AC$ & $\{AD \mid 6, \; ABA \mid 4, \; ABCE \mid 5, \; ABCD \mid 6, \; ACD \mid 7, \; ACE \mid 6\}$ \\
        $ABA$ & $\{AD \mid 6, \; ABCE \mid 5, \; ABCD \mid 6, \; ACD \mid 7, \; ACE \mid 6, \; ABAB \mid 6, \; ABAC \mid 8, \; ABAD \mid 10\}$ \\
        $ABCE$ & $\{AD \mid 6, \; ABCD \mid 6, \; ACD \mid 7, \; ACE \mid 6, \; ABAB \mid 6, \; ABAC \mid 8, \; ABAD \mid 10\}$ \\
        \bottomrule
        \end{tabular}
    \end{center}
    \vspace{1em}

    \textbf{Intra:}
    \begin{center}
        \begin{tabular}{ll}
        \toprule
        \textbf{Path} & \textbf{Open Set} \\
        \midrule
        - & $\{A \mid 0\}$ \\
        $A$ & $\{AB \mid 2, \; AC \mid 4, \; AD \mid 6\}$ \\
        $AB$ & $\{AC \mid 4, \; AD \mid 6, \; ABC \mid 3, \; \cancel{ABA} \}$ \\
        $ABC$ & $\{AC \mid 4, \; AD \mid 6, \; ABCE \mid 5, \; ABCD \mid 6\}$ \\
        $AC$ & $\{AD \mid 6, \; ABCE \mid 5, \; ABCD \mid 6, \; ACD \mid 7, \; ACE \mid 6\}$ \\
        $ABCE$ & $\{AD \mid 6, \; ABCD \mid 6, \; ACD \mid 7, \; ACE \mid 6\}$ \\
        \bottomrule
        \end{tabular}
    \end{center}
    \vspace{1em}

    \textbf{Inter:}
    \begin{center}
        \begin{tabular}{lll}
        \toprule
        \textbf{Path} & \textbf{Open Set} & \textbf{Closed Set} \\
        \midrule
        - & $\{A \mid 0\}$ & - \\
        $A$ & $\{AB \mid 2, \; AC \mid 4, \; AD \mid 6\}$ & $\{A\}$ \\
        $AB$ & $\{AC \mid 4, \; AD \mid 6, \; ABC \mid 3, \; \cancel{ABA} \} $ & $\{A, B\}$ \\
        $ABC$ & $\{AC \mid 4, \; AD \mid 6, \; ABCE \mid 5, \; ABCD \mid 6\} $ & $\{A, B, C\}$ \\
        $AC$ & $\{AD \mid 6, \; ABCE \mid 5, \; ABCD \mid 6, \; ACD \mid 7, \; ACE \mid 6\}$ & $\{A, B, C\}$ \\
        $ABCE$ & $\{AD \mid 6, \; ABCD \mid 6, \; ACD \mid 7, \; ACE \mid 6\}$ & $\{A, B, C, E\}$ \\
        \bottomrule
        \end{tabular}
    \end{center}
\end{example}
\newpage

\begin{process} \textbf{HFS}
    \begin{enumerate}
        \item Start at $s_0$ as \textbf{current node}
        \item Expand all neighboring nodes of the \textbf{current node} and add them to the open set (priority queue).
        \item Remove the \textbf{current node} from the open set and add it to the path. 
        \item Choose the lowest heuristic value expanded node from the open set as the \textbf{current node}.
        \item Repeat steps 2 and 4 until the goal state is reached or the open set is empty.
    \end{enumerate}
\end{process}

\begin{example} \textbf{HFS}
    \begin{center}
        \begin{tabular}{ll}
        \toprule
        \textbf{Path} & \textbf{Open Set} \\
        \midrule
         & $\{A \mid 3\}$ \\
        $A$ & $\{AB \mid 2, \; AC \mid 1, \; AD \mid 1\}$ \\
        $AC$ & $\{AB \mid 2, \; AD \mid 1, \; ACE \mid 0\}$ \\
        $ACE$ & $\{AB \mid 2, \; AD \mid 1\}$ \\
        \bottomrule
        \end{tabular}
    \end{center}
\end{example}
\newpage


\begin{process} \textbf{A$^*$}
    \begin{enumerate}
        \item Start at $s_0$ as \textbf{current node}
        \item Expand all neighboring nodes of the \textbf{current node} and add them to the open set (priority queue).
        \item Remove the \textbf{current node} from the open set and add it to the path. 
        \item Choose the lowest $\text{esct}(p) = \text{cst}(p) + \text{hur}(p)$ expanded node from the open set as the \textbf{current node}.
        \item Repeat steps 2 and 4 until the goal state is reached or the open set is empty.
    \end{enumerate}
\end{process}

\begin{example} \textbf{A$^*$}
    \begin{center}
        \begin{tabular}{ll}
        \toprule
        \textbf{Path} & \textbf{Open Set} \\
        \midrule
        - & $\{A \mid 3 \}$ \\
        $A$ & $\{AB \mid 2 + 2, \; AC \mid 4 + 1, \; AD \mid 6 + 1\}$ \\
        $AB$ & $\{AC \mid 5, \; AD \mid 7, \; ABC \mid (2 + 1) + 1, \; ABA \mid (2 + 2) + 3 \}$ \\
        $ABC$ & $\{AC \mid 5, \; AD \mid 7, \; ABA \mid 7, \; ABCD \mid (2 + 1 + 3) + 1, \; ABCE \mid (2 + 1 + 2) + 0, \}$ \\
        $AC$ & $\{AD \mid 7, \; ABA \mid 7, \; ABCD \mid 7, \; ABCE \mid 5, \; ACD \mid (4 + 3) + 1, \; ACE \mid (4 + 2) + 0\}$ \\
        $ABCE$ & $\{AD \mid 7, \; ABA \mid 7, \; ABCD \mid 7, \; ACD \mid 8, \; ACE \mid 6 \}$ \\
        \bottomrule
        \end{tabular}
    \end{center}
    \vspace{1em}

    \textbf{Intra:}
    \begin{center}
        \begin{tabular}{ll}
        \toprule
        \textbf{Path} & \textbf{Open Set} \\
        \midrule
        - & $\{A \mid 3 \}$ \\
        $A$ & $\{AB \mid 2 + 2, \; AC \mid 4 + 1, \; AD \mid 6 + 1\}$ \\
        $AB$ & $\{AC \mid 5, \; AD \mid 7, \; ABC \mid (2 + 1) + 1, \; \cancel{ABA} \}$ \\
        $ABC$ & $\{AC \mid 5, \; AD \mid 7, \; ABCD \mid (2 + 1 + 3) + 1, \; ABCE \mid (2 + 1 + 2) + 0, \}$ \\
        $AC$ & $\{AD \mid 7, \; ABCD \mid 7, \; ABCE \mid 5, \; ACD \mid (4 + 3) + 1, \; ACE \mid (4 + 2) + 0\}$ \\
        $ABCE$ & $\{AD \mid 7, \; ABCD \mid 7, \; ACD \mid 8, \; ACE \mid 6 \}$ \\
        \bottomrule
        \end{tabular}
    \end{center}
    \vspace{1em}

    \textbf{Inter:}
    \begin{center}
        \begin{tabular}{lll}
        \toprule
        \textbf{Path} & \textbf{Open Set} & \textbf{Closed Set} \\
        \midrule
        - & $\{A \mid 3 \}$ & - \\
        $A$ & $\{AB \mid 2 + 2, \; AC \mid 4 + 1, \; AD \mid 6 + 1\}$ & $\{A\}$ \\
        $AB$ & $\{AC \mid 5, \; AD \mid 7, \; ABC \mid (2 + 1) + 1, \; \cancel{ABA} \}$ & $\{A, B\}$ \\
        $ABC$ & $\{AC \mid 5, \; AD \mid 7, \; ABCD \mid (2 + 1 + 3) + 1, \; ABCE \mid (2 + 1 + 2) + 0, \}$ & $\{A, B, C\}$ \\
        $AC$ & $\{AD \mid 7, \; ABCD \mid 7, \; ABCE \mid 5, \; ACD \mid (4 + 3) + 1, \; ACE \mid (4 + 2) + 0\}$ & $\{A, B, C\}$ \\
        $ABCE$ & $\{AD \mid 7, \; ABCD \mid 7, \; ACD \mid 8, \; ACE \mid 6 \}$ & $\{A, B, C, E\}$ \\
        \bottomrule
        \end{tabular}
    \end{center}
\end{example}
\newpage

\begin{process} \textbf{IIA*}
    \begin{enumerate}
        \item Start with a cost limit of 0.
        \item Perform A* up to the current cost limit.
        \item If the goal state is not reached, increment the cost limit based on given fcn and repeat step 2.
        \item Continue until the goal state is found or all nodes are explored.
    \end{enumerate}
\end{process}

\begin{example} \textbf{IIA*}
    \begin{center}
        \begin{tabular}{lll}
        \toprule
        \textbf{Cost} & \textbf{Path} & \textbf{Open Set} \\
        \midrule
        0 & $\langle \rangle$ & $\{\}$ \\
        \midrule
        1 & $\langle \rangle$ & $\{\}$ \\
        \midrule
        2 & $\langle \rangle$ & $\{\}$ \\
        \midrule
        3 & $\langle \rangle$ & $\{A \mid 3\}$ \\
        3 & $A$ & $\{\}$ \\
        \midrule
        4 & $A$ & $\{AB \mid 2 + 2\}$ \\
        4 & $AB$ & $\{ABC \mid 3 + 1\}$ \\
        4 & $ABC$ & $\{\}$ \\
        \midrule
        5 & $ABC$ & $\{ABCE \mid 5 + 0\}$ \\
        5 & $ABCE$ & $\{\}$ \\
        \bottomrule
        \end{tabular}
    \end{center}
\end{example}

\begin{process} \textbf{WA*}
    \begin{enumerate}
        \item Start at $s_0$ as \textbf{current node}
        \item Expand all neighboring nodes of the \textbf{current node} and add them to the open set (priority queue).
        \item Remove the \textbf{current node} from the open set and add it to the path. 
        \item Choose the lowest $\text{esct}(p) =  w \cdot \text{cst}(p) + (1-w) \cdot \text{hur}(p)$ expanded node from the open set as the \textbf{current node}.
        \item Repeat steps 2 and 4 until the goal state is reached or the open set is empty.
    \end{enumerate}
\end{process}
\newpage

\begin{process} \textbf{How to Prove Consistent/Admissible Given a Search Graph?}
   
    \textbf{Admissible:}
    \begin{enumerate}
        \item Given $\text{hur}(s)$ and search graph with $\text{cst}(s,a,\text{tr}(s,a))$. If consistent, then it is admissible.
        \item Check $\forall s \in \mathcal{G}$, $\text{hur}(s) = 0$. If not, then it is not admissible.
        \item For each $s \in \mathcal{S}$, calculate $\text{hur}^*(s)$ (i.e. actual cost of optimal soln.) using the search graph.
        \begin{enumerate}
            \item Start at $s$ and choose path that gives the lowest cost to $s \in \mathcal{G}$. 
        \end{enumerate}
        \item Check if $\text{hur}(s) \leq \text{hur}^*(s) \; \forall s \in \mathcal{S}$. If not, then it is not admissible.
        \item Repeat $\forall s \in \mathcal{S}$. 
        \item If all are true, then it is admissible.
    \end{enumerate}
    \vspace{1em}

    \textbf{Consistent:}
    \begin{enumerate}
        \item Given $\text{hur}(s)$ and search graph with $\text{cst}(s,a,\text{tr}(s,a))$. 
        \item Check $\forall s \in \mathcal{G}$, $\text{hur}(s) = 0$. If not, then it is not consistent.
        \item For each $s \in \mathcal{S}$, calculate $\text{hur}(s) - \text{hur}(\text{tr}(s, a))$.
        \begin{enumerate}
            \item check if it is $\leq \text{cst}(s,a,\text{tr}(s,a))$. If not, then it is not consistent.
            \item Repeat $\forall a \in \mathcal{A}(s)$
        \end{enumerate}
        \item Repeat $\forall s \in \mathcal{S}$.
        \item If all are true, then it is consistent.
    \end{enumerate}
\end{process}

\begin{warning}
    Be careful of bidirectional edges bc for consistency you need compute the cost of the heuristic edge in both directions.
\end{warning}
\newpage

\begin{example}
    \customFigure[0.5]{../Images/L2_8.png}{Jungle ($s^{(0)}$), Desert, Swamp, Mountain, Plains (Goal)}

    \textbf{Admissible:}
    \begin{enumerate}
        \item \textbf{$s=$Plains:} $\text{hur}(\text{Plains}) = 0$ 
        \item \textbf{$s=$Jungle:} $\text{hur}(\text{Jungle}) = 3 \leq \text{hur}^*(\text{Jungle}) = 2 + 1 + 2 = 5$
        \item \textbf{$s=$Desert:} $\text{hur}(\text{Desert}) = 2 \leq \text{hur}^*(\text{Desert}) = 1 + 2$
        \item \textbf{$s=$Swamp:} $\text{hur}(\text{Swamp}) = 1 \leq \text{hur}^*(\text{Swamp}) = 2$
        \item \textbf{$s=$Mountain:} $\text{hur}(\text{Mountain}) = 1 \leq \text{hur}^*(\text{Mountain}) = 1$
        \item Therefore, it is admissible.
    \end{enumerate}
    \vspace{1em}

    \textbf{Consistent:}
    \begin{enumerate}
        \item \textbf{$s=$Plains:} $\text{hur}(\text{Plains}) = 0$
        \item \textbf{$s=$Jungle:}
        \begin{enumerate}
            \item $\text{hur}(\text{Jungle}) - \text{hur}(\text{Desert}) = 3 - 2 = 1 \leq \text{cst}(\text{Jungle}, \cdot, \text{Desert}) = 2$
            \item $\text{hur}(\text{Jungle}) - \text{hur}(\text{Swamp}) = 3 - 1 = 2 \leq \text{cst}(\text{Jungle}, \cdot, \text{Swamp}) = 4$
            \item $\text{hur}(\text{Jungle}) - \text{hur}(\text{Mountain}) = 3 - 1 = 2 \leq \text{cst}(\text{Jungle}, \cdot, \text{Mountain}) = 6$
        \end{enumerate}
        \item \textbf{$s=$Desert:} 
        \begin{enumerate}
            \item $\text{hur}(\text{Desert}) - \text{hur}(\text{Jungle}) = 2 - 3 = -1 \leq \text{cst}(\text{Desert}, \cdot, \text{Jungle}) = 2$
            \item $\text{hur}(\text{Desert}) - \text{hur}(\text{Swamp}) = 2 - 1 = 1 \leq \text{cst}(\text{Desert}, \cdot, \text{Swamp}) = 1$
        \end{enumerate}
        \item \textbf{$s=$Swamp:}
        \begin{enumerate}
             \item $\text{hur}(\text{Swamp}) - \text{hur}(\text{Mountain}) = 1 - 1 = 0 \leq \text{cst}(\text{Swamp}, \cdot, \text{Mountain}) = 3$
            \item $\text{hur}(\text{Swamp}) - \text{hur}(\text{Plains}) = 1 - 0 = 1 \leq \text{cst}(\text{Swamp}, \cdot, \text{Plains}) = 2$
        \end{enumerate}
        \item \textbf{$s=$Mountain:} 
        \begin{enumerate}
            \item $\text{hur}(\text{Mountain}) - \text{hur}(\text{Jungle}) = 1 - 3 = -2 \leq \text{cst}(\text{Mountain}, \cdot, \text{Desert}) = 6$
            \item $\text{hur}(\text{Mountain}) - \text{hur}(\text{Plains}) = 1 - 0 = 1 \leq \text{cst}(\text{Mountain}, \cdot, \text{Plains}) = 1$
        \end{enumerate}
        \item Therefore, it is consistent.
    \end{enumerate}
\end{example}
\newpage

\begin{process} \textbf{How to Design Heuristic via Problem Relaxation?}
    \begin{enumerate}
        \item Make an assumption to simplify the problem as a relaxed problem. 
        \item Find the cost of the optimal solution of the relaxed problem, $\text{cst}_{\text{rel}}(s)$.
        \item HOW TO FIND THE COST OF THE OPTIMAL SOLUTION?
    \end{enumerate}
\end{process}

\begin{example}
    \customFigure[0.5]{../Images/L2_11.png}{}
\end{example}









\newpage

\section{MLP}
\begin{summary}
    \begin{center}
        \begin{tabular}{ll}
            \toprule
            \textbf{Concepts} & \textbf{Description} \\
            \toprule
            \textbf{Math symbols} & \\
            \multicolumn{2}{p{\linewidth}}{
            \begin{itemize}
                \item \(x, y, b, h, z\) - Vectors (input, outputs, bias, hidden, latent)
                \item \(\hat{y}\) - Predictions
                \item \(\theta\) - Parameters (collections of tensors)
                \item \(f, f_i, f_j, C\) - Functions
                \item \(x_i\) - Scalar, \(i\)-indexed coordinate of \(x\)
                \item \(X, W\) - Matrices
                \item \(\frac{\partial f}{\partial x_i}\) - Partial derivative of \(f\) with respect to \(x_i\)
                \item \(\nabla f\) - Gradient of \(f\)
            \end{itemize}} \\
            \midrule
            \textbf{Hyperparameter choices} & \\
            \multicolumn{2}{p{\linewidth}}{
                \begin{center}
                    \customFigure[0.5]{../Images/L3_12.png}{}
                    \vspace{-4em}
                \end{center}} \\
            \midrule
            \textbf{Finding the best parameters} & Minimize the error b/w predictions and actual values. \\
            & $ C(\theta) = \sqrt{\frac{1}{N} \sum_n (y_n - f(x_n; \theta))^2} = \text{RMSE}(y, f(x; \theta))$ \\ 
            \midrule
            \textbf{Gradient Descent: Rolling Down Hill} & Update rule to iteratively adjust parameters to reduce loss. \\
            & $\theta_n = \theta_{n-1} - \alpha \nabla C(\theta_{n-1})$ \\
            \multicolumn{2}{p{\linewidth}}{
                \begin{itemize}
                    \item \textbf{Notes:} Multiple minima, gradient info (sign (\textcolor{green}{+}, \textcolor{red}{-}), magnitude), initialization matters, learning rate $\alpha$. 
                    \customFigure[0.3]{../Images/L3_3.png}{}
                \end{itemize}} \\
            \bottomrule
        \end{tabular}
    \end{center}
\end{summary}
\newpage

\begin{summary}
    \begin{center}
        \begin{tabular}{ll}
            \toprule
            \textbf{Concepts} & \textbf{Description} \\
            \midrule
            \textbf{Backprop: Forward and Backward Passes} & Calculating pred., tracking op., and computing grad. \\
            \multicolumn{2}{p{\linewidth}}{
                \begin{itemize}
                    \customFigure[0.5]{../Images/L3_13.png}{}
                    \item $g \leftarrow \nabla_{\hat{y}} C = \nabla_{\hat{y}} L(\hat{y}, y)$: Compute gradient on output layer.
                    \item $g \leftarrow \nabla_{h^{(k)}} C = g \odot \text{act}'(h^{(k)})$: Convert gradient on layer output into a gradient on the pre-nonlinearity activation.
                    \item $\nabla_{b^{(k)}} C = g$: Compute gradient on biases.
                    \item $\nabla_{W^{(k)}} C = g z^{(k-1)\top}$: Compute gradient on weights.
                    \item $g \gets \nabla_{z^{(k-1)}} C = W^{(k)\top} g$: Propagate gradients w.r.t. the next lower-level activations.
                \end{itemize}
            } \\
            \midrule
            \textbf{Inductive Bias (Learning Bias)} & Set of assumptions that the learner applies to a model for a task, \\
            & making an algorithm learn one pattern over another. \\ 
            \midrule
            \textbf{Useful Trick in DL} & Trans. smoothly Gaussian-like data into Non-Gaussian data \\
            & using Linear Transforms + Activations. \\
            \multicolumn{2}{p{\linewidth}}{
                \begin{itemize}
                    \item \textbf{Differentiable tools:} relu, sum, mean, var, softplus, sigmoid, softmax, masking.
                    \item \textbf{Non-differentiable:} argmax, top-k, binarize.
                \end{itemize}
            } \\
            \midrule
            \textbf{MLE} & Choosing parameters that maximize the likelihood (probability) \\
            & of observing the given data we actually have. \\
            \multicolumn{2}{p{\linewidth}}{
                \begin{itemize}
                    \item \textbf{Notes:} Minimizing negative log-likelihood results in loss functions.
                \end{itemize}
            } \\
            \midrule
            Regression - Gaussian Distribution & \\
            \multicolumn{2}{p{\linewidth}}{
                \begin{itemize}
                    \item \textbf{Likelihood:} The probability density of observing the actual output values given the inputs and our model parameters (weights, bias, and the variance of the noise).
                    \item \textbf{MLE:} Minimizing negative Log-Likelihood leads to the Mean Squared Error.
                    \item \textbf{Activation Function:} $\text{Act} = x$: Identity function.
                    \customFigure[0.4]{../Images/L3_4.png}{}
                \end{itemize}
            } \\
            \bottomrule
        \end{tabular}
    \end{center}
\end{summary}
\newpage

% \begin{example}
%     \begin{itemize}
%         \item \textbf{Why should the transformation be differentiable?} 
%         \begin{itemize}
%             \item Smooth
%         \end{itemize}
%     \end{itemize}    
% \end{example}

\begin{summary}
    \begin{center}
        \begin{tabular}{ll}
            \toprule
            \textbf{Concepts} & \textbf{Description} \\
            \midrule
            Binary Classification - Binomial Distribution & \\ 
            \multicolumn{2}{p{\linewidth}}{
                \begin{itemize}
                    \item \textbf{Likelihood:} The probability of observing the actual class labels (0 or 1) given the inputs and our model parameters.
                    \item \textbf{MLE:} Minimizing negative Log-Likelihood leads to the Binary Cross Entropy loss (BCE).
                    \item \textbf{Activation Function:} $\text{Act} = \text{sigmoid}$: Squashes the output to $[0, 1].$
                    \customFigure[0.3]{../Images/L3_6.png}{}
                    \vspace{-1em}
                \end{itemize}
            } \\
            \midrule
            Multilabel Classification - N-dim Binomial Distribution & \\
            \multicolumn{2}{p{\linewidth}}{
                \begin{itemize}
                    \item \textbf{Loss function:} Sum of Binary Cross Entropy (BCE) for each label.
                    \item \textbf{Activation Function:} $\text{Act} = \text{sigmoid}$: Squashes the output to $[0, 1].$
                    \customFigure[0.3]{../Images/L3_7.png}{}
                    \vspace{-1em}
                \end{itemize}
            } \\
            \bottomrule
        \end{tabular}
    \end{center}
\end{summary}
\newpage

\begin{summary}
    \begin{center}
        \begin{tabular}{ll}
            \toprule
            \textbf{Concepts} & \textbf{Description} \\
            \midrule
            Multiclass Classification - Multinomial Distribution & \\
            \multicolumn{2}{p{\linewidth}}{
                \begin{itemize}
                    \item \textbf{Loss function:} Cross entropy (CE).
                    \item \textbf{Activation Function:} $\text{Act} = \text{softmax}$, where $\text{softmax}(z) = \frac{e^{z_i}}{\sum_j e^{z_j}}$
                    \customFigure[0.3]{../Images/L3_8.png}{}
                    \vspace{-1em}
                \end{itemize}
            } \\
            \midrule
            Ordinal Classification & \\
            \multicolumn{2}{p{\linewidth}}{
                \begin{itemize}
                    \item \textbf{Activation Function:} $\text{Act} = \text{sigmoid}$
                    \customFigure[0.3]{../Images/L3_9.png}{}
                    \vspace{-1em}
                \end{itemize}
            } \\
            \midrule
            Zero-Inflated Distribution & \\
            \multicolumn{2}{p{\linewidth}}{
                \begin{itemize}
                    \item \textbf{Prediction:} Two separate predictions for active and value. $\text{Pred}(z) =
                    \begin{cases}
                        \text{active}: \text{sigmoid}(W_b \cdot z) \\
                        \text{value}: W_r \cdot z
                    \end{cases}$
                    \item \textbf{Loss function:} BCE + Masked MSE: $\text{BCE}(\text{mask}, y_{\text{pred,binary}}) + \text{mask} \cdot * \text{MSE}(y_{\text{true}}, y_{\text{pred,reg}})$
                    \customFigure[0.5]{../Images/L3_11.png}{}
                    \vspace{-1em}
                \end{itemize}
            } \\
            \bottomrule
        \end{tabular}
    \end{center}
\end{summary}


\newpage

\section{Neural Network Engineering}
\subsection{Probably Approximately Correct (PAC) Estimations}
\begin{motivation}
    More than one fcn may be consistent w/ the data, how to find the best one?
\end{motivation}

\subsubsection{Hoeffding's Inequality}
\begin{motivation}
    Bound $|\mu - \nu|$ w.r.t. $N$.
\end{motivation}

\begin{definition}
    For any $\epsilon > 0$,
    \begin{equation}
        \mathbb{P}(|\nu - \mu| \geq \epsilon) \leq 2e^{-2\epsilon^2N}
    \end{equation}
    \begin{itemize}
        \item $\mu$: Probabillity of an event.
        \item $\nu$: Relative frequency in a sample size $N$.
        \item $\epsilon$: Tolerance (i.e. how close we want $\nu$ to be to $\mu$).
        \begin{itemize}
            \item $\epsilon \rightarrow 0$: $\nu = \mu$
        \end{itemize}
        \item $\mu \overset{?}{\approx} \nu $: $\mu$ is probably approximately equal to $\nu$. As $N \rightarrow \infty$: $\nu \rightarrow \mu$
    \end{itemize}
\end{definition}

\begin{warning}
    Approx. the true dist. w/ high prob. by taking a large enough $N$ (i.e. empirical dist. converges to true dist.).
    \begin{itemize}
        \item i.e. Probability of a sig. deviation shrinks exp. w/ $N$.
    \end{itemize}
\end{warning}

\subsection{PAC Learning}
\subsubsection{Error}
\begin{definition}
    \begin{itemize}
        \item \textbf{Out-Sample Error:}
        \begin{equation*}
            E_{\text{out}} = \mathbb{P}[f \neq h]
        \end{equation*}
        \item \textbf{In-Sample Error:}
        \begin{equation*}
            E_{\text{in}} = \frac{1}{N} \sum_{i=1}^{N} \mathbb{I}[f(x^{(i)}) \neq h(x^{(i)})]
        \end{equation*}
    \end{itemize}
\end{definition}

\subsubsection{Union Bound Theorem}
\begin{theorem}
    The prob. of at least one of the events $E_1, \ldots, E_M$ occurring is bounded by the sum of the prob. of each event occurring:
    \begin{equation*}
        \mathbb{P} \left[E_1 \lor \cdots \lor E_M \right] \leq \sum_{i=1}^{M} \mathbb{P}[E_i]
    \end{equation*}
\end{theorem}

\begin{notes}
    \begin{itemize}
        \item If the events are mutually exclusive, then the union bound is tight (i.e. equality holds).
        \item If the events are highly correlated, then the union bound is loose (i.e. inequality holds)
        \begin{itemize}
            \item Some events may be more likely to occur together.
        \end{itemize}
    \end{itemize}
\end{notes}
\newpage

\subsubsection{Generalization of Hoeffding's Inequality}
\begin{definition}
    Assuming that $h$ is chosen from a set of hypotheses $\mathcal{H}$, derive a (loose) upper-bound on $|E_{\text{out}} - E_{\text{in}}|$:
    \begin{align*}
        \mathbb{P} \left[ \bigvee_{h \in \mathcal{H}} \left( |E_{\text{out}} - E_{\text{in}}(h)| > \varepsilon \right) \right]
        &\leq \sum_{h \in \mathcal{H}} \mathbb{P} \left[ |E_{\text{out}} - E_{\text{in}}(h)| > \varepsilon \right] \\
        &\leq \sum_{h \in \mathcal{H}} 2e^{-2\varepsilon^2 N} \\
        &= 2 |\mathcal{H}| e^{-2\varepsilon^2 N} 
    \end{align*}
    \begin{itemize}
        \item Endow $\mathcal{F}$ (i.e. fcn space) w/ prob. distribution, $P : \mathcal{X} \to [0,1]$, then 
        \begin{itemize}
            \item $E_{\text{out}}$ (i.e. true error of a hyp. over entire dist. of data) is analogous to $\mu$ 
            \item $E_{\text{in}}(h)$ (i.e. empirical error of hyp. on a finite sample) is analogous to $\nu$. 
        \end{itemize}
    \end{itemize}
\end{definition}

\begin{notes}
    \begin{itemize}
        \item $E_{\text{in}}(h) \stackrel{?}{\approx} E_{\text{out}}$ requires small $|\mathcal{H}|$ (generalization)
        \begin{itemize}
            \item Look at inequality, small $|\mathcal{H}|$ $\rightarrow$ small $E_{\text{out}} - E_{\text{in}}$ (i.e. prevents overfitting but leads to underfitting)
        \end{itemize}
        \item $E_{\text{in}}(h) \approx 0$ requires large $|\mathcal{H}|$ (discrimination)
        \begin{itemize}
            \item Need large $|\mathcal{H}|$ to capture the true dist. (i.e. prevents underfitting but leads to overfitting)
        \end{itemize}
    \end{itemize}
\end{notes}
\newpage

\begin{example}
    \begin{enumerate}
        \item \textbf{Given:} An opaque box containing \textcolor{red}{red} and \textcolor{blue}{blue} balls. Take $N$ IID samples.
        \begin{itemize}
            \item $\mu$: Probability of drawing a \textcolor{blue}{blue} balls (unknown).
            \item $\nu$: Relative frequency of \textcolor{blue}{blue} balls in the sample (known).
        \end{itemize}
        \item \textbf{Problem 1:} What is $\nu$ in this case? 8 balls total, 5 are blue. 
        \item \textbf{Solution 1:} $\nu = \frac{5}{8}$
        \item \textbf{Problem 2:} How to partition $\mathcal{F}$ into regions where \textcolor{blue}{$f=h$} and \textcolor{red}{$f \neq h$}?
        \item \textbf{Solution 2:} 
        \customFigure[0.5]{../Images/L5_8.png}{LS $h$, MS $f$}
        \item \textbf{Problem 3:} What is the out-sample error?
        \item \textbf{Solution 3:} In words, the probability of the hypothesis being wrong.
        \customFigure[0.3]{../Images/L5_9.png}{}
        \item \textbf{Problem 4:} What is the in-sample error given this sample of 11 balls s.t. $f=h$, $1$ ball s.t. $f \neq h$?
        \item \textbf{Solution 4:} $E_{\text{in}} = \frac{1}{12}$
        
    \end{enumerate}
\end{example}




\newpage

\section{Hyperparameter Optimization}
\subsection{Maximum Likelihood Estimation}
\begin{motivation}
    Choose parameter $\theta$ that is most likely to generate the observation $x_1,x_2,\ldots,x_n$.
    \customFigure[0.5]{../Images/L5_0.png}{}
\end{motivation}

\begin{definition}
    \begin{equation}
        \hat{\Theta} = \arg\max_\theta \, P_{\underline{X}}(\underline{x}|\theta), \text{ discrete } X.
    \end{equation}
        
    \begin{equation}
        \hat{\Theta} = \arg\max_\theta \, f_{\underline{X}}(\underline{x}|\theta), \text{ continuous } X.
    \end{equation}
        
\end{definition}

\subsubsection{Log-Likelihood}
\begin{definition}
    \begin{equation}
        \hat{\theta} = \arg\max_\theta \sum_{i=1}^n \log P_X(x_i|\theta) 
    \end{equation}

    \begin{equation}
        \hat{\theta} = \arg\max_\theta \sum_{i=1}^n \log f_X(x_i|\theta).
    \end{equation}    
\end{definition}

\begin{example}
    
\end{example}

\begin{notes}
    
\end{notes}




\newpage

\section{Representations and VAE}
\subsection{MLE for Categorical Random Variables}
\begin{example}
    \begin{enumerate}
        \item We say that $X \sim \text{Cat}(\underline{\theta})$ if 
        \[
        P[X = m] = \theta_m, \quad m = 1, 2, \dots, M.
        \]
        \begin{itemize}
            \item Going from 2 to $M$ categories is a generalization of the Bernoulli distribution.
        \end{itemize}
        The parameter $\underline{\theta}$ is a vector:
        \[
        \underline{\theta} = 
        \begin{bmatrix}
        \theta_1 \\
        \theta_2 \\
        \vdots \\
        \theta_M
        \end{bmatrix},
        \]
        such that $\theta_m \geq 0$ and $\sum_{m=1}^M \theta_m = 1.$
    
        \item Given $n$ i.i.d. observations $X_1, \dots, X_n$, we aim to find the maximum likelihood estimator (MLE) of $\underline{\theta}$.
    
        \item Define $n_m$ as the number of observations that equal $m$:
        \[
        n_m = \sum_{i=1}^n 1(x_i = m),
        \]
        where $1(x_i = m)$ is the indicator function. Note that $\sum_{m=1}^M n_m = n.$
    
        \item The likelihood function is:
        \[
        p_{\underline{X}}(\underline{x} \mid \underline{\theta}) = \prod_{m=1}^M \theta_m^{n_m}.
        \]
        \begin{itemize}
            \item Similar to the Bernoulli distribution, but with $M$ categories.
        \end{itemize}
        Taking the log, we get:
        \[
        \log p_{\underline{X}}(\underline{x} \mid \underline{\theta}) = \sum_{m=1}^M n_m \log \theta_m.
        \]
    
        \item To find the optimal $\underline{\theta}$, we minimize the negative log-likelihood:
        \[
        \min_{\underline{\theta}} -\sum_{m=1}^M n_m \log \theta_m,
        \]
        subject to the constraints $\theta_m \geq 0$ for $1 \leq m \leq M$ and $\sum_{m=1}^M \theta_m = 1.$
    
        \item Solving this optimization problem, the MLE is:
        \[
        \hat{\Theta}_m = \frac{N_m}{n} = \frac{\sum_{i=1}^{n} 1(X_i = n)}{n}, \quad \underline{\hat{\Theta}} = 
        \begin{bmatrix}
        \frac{N_1}{n} \\
        \vdots \\
        \frac{N_m}{n}
        \end{bmatrix}.
        \]
    \end{enumerate}
\end{example}
\newpage

\subsection{MLE for Gaussian Random Variables}
\begin{example}
    \begin{enumerate}
        \item Given $n$ i.i.d. observations $X_1, \dots, X_n$ of a Gaussian random variable with parameters $(\mu, \sigma^2)$, we aim to find the maximum likelihood estimators (MLEs) of $\mu$ and $\sigma^2$.

        \begin{align*}
        f_{\underline{X}}(\underline{x} | \mu, \sigma^2) &= \prod_{i=1}^n \frac{1}{\sqrt{2\pi} \sigma} \exp\left(-\frac{(x_i - \mu)^2}{2\sigma^2}\right), \\
        \log f_{\underline{X}}(\underline{x} | \mu, \sigma^2) &= \sum_{i=1}^n \left(-\frac{(x_i - \mu)^2}{2\sigma^2} - \log \sigma - \log \sqrt{2\pi}\right).
        \end{align*}
    
        \item To find $\mu$, take the derivative of the log-likelihood with respect to $\mu$ and set it to zero:
        \begin{align*}
        0 &= \frac{\partial}{\partial \mu} \sum_{i=1}^n \left(-\frac{(x_i - \mu)^2}{2\sigma^2}\right), \\
        0 &= \frac{1}{n} \sum_{i=1}^n x_i - \mu, \\
        \mu &= \frac{1}{n} \sum_{i=1}^n x_i.
        \end{align*}
    
        \item To find $\sigma^2$, take the derivative of the log-likelihood with respect to $\sigma^2$ and set it to zero:
        \begin{align*}
        0 &= \frac{\partial}{\partial \sigma^2} \sum_{i=1}^n \left(-\frac{(x_i - \mu)^2}{2\sigma^2} - \frac{1}{2} \log \sigma^2\right), \\
        0 &= -\frac{1}{2} \sum_{i=1}^n \left(\frac{(x_i - \mu)^2}{\sigma^4}\right) + \frac{1}{2\sigma^2}, \\
        \sigma^2 &= \frac{1}{n} \sum_{i=1}^n (x_i - \mu)^2.
        \end{align*}
    
        \item Thus, the MLEs are:
        \begin{align*}
        \hat{\mu} &= \frac{1}{n} \sum_{i=1}^n X_i, \quad \text{(sample mean)} \\
        \hat{\sigma}^2 &= \frac{1}{n} \sum_{i=1}^n (X_i - \hat{\mu})^2. \quad \text{(sample variance)}
        \end{align*}
        \begin{itemize}
            \item Note: The sample variance is biased, so we often use $\frac{1}{n-1}$ instead of $\frac{1}{n}$ to make it unbiased.
            \item Note: The sample mean is unbiased.
        \end{itemize}
    \end{enumerate}
\end{example}

\subsection{Will the Sun Rise Tomorrow? (Laplace's Problem)}
\begin{example}
    \begin{itemize}
        \item Observation: The Sun has risen for $n$ consecutive days. Estimate the probability that it will rise tomorrow.
        \item Model: Assume $n$ i.i.d. Bernoulli random variables $X_1, \dots, X_n$ with $P[X_i = 1] = \theta$.
    \end{itemize}
\end{example}
\subsubsection{Frequentist Approach}
\begin{example}
    \begin{enumerate}
        \item The Maximum Likelihood Estimator (MLE) is:
        \[
        \hat{\theta} = \frac{K}{n} = \frac{\sum_{i=1}^n X_i}{n}.
        \]
        \item If $K = n$ (i.e., the Sun has risen every day so far), then:
        \[
        \hat{\theta} = \frac{n}{n} = 1.
        \]
        \item Conclusion: The Sun will rise tomorrow with probability $1$, regardless of what $n$ is, based on the Frequentist approach. 
        \begin{itemize}
            \item This doesn't make sense b/c if $n=1$ then we are assuming 100\% it will rise based on one observation.
        \end{itemize}
    \end{enumerate}
\end{example}

\subsubsection{Bayesian Approach}
\begin{example}
    \begin{enumerate}
        \item Assume that $\theta$ is not fixed but drawn from a uniform distribution in $[0, 1]$. This means that the probability of the Sun rising is based on a uniform distribution.
        \item We want to find the probability that the sun will rise tomorrow given that it has risen for $n$ consecutive days:
        \[
        P[X_{n+1} = 1 | X_1 = 1, \dots, X_n = 1].
        \]
        Using Bayes' Theorem:
        \[
        P[X_{n+1} = 1 | X_1 = 1, \dots, X_n = 1] = \frac{P[X_1 = 1, \dots, X_{n+1} = 1]}{P[X_1 = 1, \dots, X_n = 1]}.
        \]
        \item Compute $P[X_1 = 1, \dots, X_n = 1]$:
        \[
        P[X_1 = 1, \dots, X_n = 1] = \int_0^1 P[X_1 = 1, \dots, X_n = 1 | \Theta = \theta] f_\Theta(\theta) \, d\theta.
        \]
        \begin{itemize}
            \item The joint probability is calculated by integrating the product of the likelihood and the prior (i.e. marginalizing over $\theta$).
            \item The likilihood becomes $\theta^n$ because the observations are i.i.d.
            \item The prior is uniform, so $f_\Theta(\theta) = 1$.
        \end{itemize}
        Since $f_\Theta(\theta) = 1$ (uniform prior) and $P[X_1 = 1, \dots, X_n = 1 | \Theta = \theta] = \theta^n$, we have:
        \[
        P[X_1 = 1, \dots, X_n = 1] = \int_0^1 \theta^n \, d\theta = \frac{1}{n+1}.
        \]
        \item Compute $P[X_1 = 1, \dots, X_{n+1} = 1]$ similarly:
        \[
        P[X_1 = 1, \dots, X_{n+1} = 1] = \int_0^1 \theta^{n+1} \, d\theta = \frac{1}{n+2}.
        \]
        \item Combine results:
        \[
        P[X_{n+1} = 1 | X_1 = 1, \dots, X_n = 1] = \frac{\frac{1}{n+2}}{\frac{1}{n+1}} = \frac{n+1}{n+2}.
        \]
        \item Conclusion: As $n$ increases, the probability approaches $1$, providing more certainty with more data.
    \end{enumerate}
\end{example}

\subsection{Sample Mean is Not Always an ML Estimator}
\begin{example}
    Given an unknown voltage \( x \), we measure it using a voltmeter that outputs a random reading \( Y \) that is uniform in \( [0, x] \). Suppose we make \( n \) i.i.d. measurements \( Y_1, \dots, Y_n \) and wish to estimate \( \mu = \frac{x}{2} = \mathbb{E}[Y] \).
    \begin{enumerate}
        \item PDF of \( Y \):
        \begin{equation*}
            f_Y(y \mid \mu) = 
            \begin{cases}
                \frac{1}{2\mu} & \text{if } 0 \leq y \leq 2\mu, \\
                0 & \text{otherwise}.
            \end{cases}
        \end{equation*}
        \begin{itemize}
            \item This is a uniform distribution, where $x = 2\mu$. 
        \end{itemize}
        \item \textbf{Sample Mean:} 
        \[
        \hat{\mu} = \frac{1}{n} \sum_{i=1}^n Y_i \quad \cdots (1)
        \]

        \item \textbf{To Find the ML Estimator:}
        \begin{align*}
            f_{\mathbf{Y}}(\mathbf{y} \mid \mu) &= \prod_{i=1}^n f_Y(y_i \mid \mu) \\
            &= \prod_{i=1}^n \frac{1}{2\mu} \cdot 1(0 \leq y_i \leq 2\mu) \\
            &= \frac{1}{(2\mu)^n} \prod_{i=1}^n 1(0 \leq y_i \leq 2\mu).
        \end{align*}
        \begin{itemize}
            \item The indicator function ensures that all measurements are within the range \( [0, 2\mu] \) for the likelihood to be non-zero. Therefore, $\mu$ must be chosen such that all data points lie within this range.
        \end{itemize}
        
        The likelihood is maximized for:
        \[
        \arg\max_{\mu} f_{\mathbf{Y}}(\mathbf{y} \mid \mu) = \max_{1 \leq i \leq n} \frac{1}{2} Y_i.
        \]

        Therefore:
        \[
        \hat{\mu} = \max_{1 \leq i \leq n} \frac{1}{2} Y_i \quad \cdots (2)
        \]
        \begin{itemize}
            \item The likelihood is non-zero only if $\mu$ is greater than or equal to the maximum of the measurements because all data points must lie within the range $[0, 2\mu]$. 
            \item i.e. $2 \mu > \max_{1 \leq i \leq n} Y_i$, which is to ensure that ALL measurements are within the range $[0, 2\mu]$, so this must be $\mu$.
        \end{itemize}

        \item Clearly, \( (1) \neq (2) \).
    \end{enumerate}
\end{example}


\newpage

\section{Software Development}
\subsection{Software Engineering}
\begin{summary}
    \begin{center}
        \begin{tabular}{ll}
            \toprule
            \textbf{Concept} & \textbf{Description} \\
            \midrule
            Code Readability & Maintainable code enables collaboration and future work. \\
            \multicolumn{2}{p{\linewidth}}{
                \centering
                \begin{minipage}{0.48\linewidth}
                    \centering
                    \customFigure[1.0]{../Images/L7_4.png}{Image 1 Description}
                \end{minipage}
                \hfill
                \begin{minipage}{0.48\linewidth}
                    \centering
                    \customFigure[1.0]{../Images/L7_1.png}{Image 2 Description}
                \end{minipage} 
            } \\
            \bottomrule
        \end{tabular}
    \end{center}
\end{summary}
\subsection{Code Readability Matters}
\begin{notes}
    \begin{itemize}
        \item \textbf{Overview:} Maintainable code enables collaboration and future work.
        \customFigure[0.5]{../Images/L7_1.png}{}
        \customFigure[0.5]{../Images/L7_4.png}{}
        \item \textbf{Naming:} Clear names enhance code understanding 
        \begin{itemize}
            \item \textbf{Functions:} verb\_do
            \item \textbf{Variables:} two\_three\_words
            \item \textbf{Classes:} CapitalizedWords
        \end{itemize}
        \customFigure[0.5]{../Images/L7_2.png}{}
    \end{itemize}
\end{notes}

\subsection{Style Guides}
\begin{notes}
    \begin{itemize}
        \item \textbf{Overview:} Ensure consistent code formatting. 
        \item \textbf{Why?} Provide a standardized set of rules for formatting code, prompting uniformity, and reducing cognitive load when reading code. 
        \item \textbf{Examples:} PEP8, Google Style Guide, etc.
    \end{itemize}
    \customFigure[0.5]{../Images/L7_3.png}{}
    \customFigure[0.5]{../Images/L7_5.png}{}
\end{notes}
\newpage

\section{Coding Mantras / Ideas}
\subsection{Zen of Python}
\begin{definition}
    \customFigure[0.5]{../Images/L7_0.png}{}
\end{definition}

\subsection{KISS: Keep It Simple and Straightforward}
\begin{notes}
    Advocate for solutions that are easy to understand and maintain, reduce unnecessary complexity.
    \customFigure[0.5]{../Images/L7_6.png}{}
\end{notes}

\subsection{YAGNI: You Aren't Gonna Need It}
\begin{notes}
    AKA premature optimization is the root of all evil. 
    \customFigure[0.5]{../Images/L7_7.png}{}
    \customFigure[0.5]{../Images/L7_8.png}{}
\end{notes}

\subsection{Principle of DRY and WET and DAMP}
\begin{notes}
    Don't Repeat Yourself; Write Everything Twice.
    \customFigure[0.5]{../Images/L7_9.png}{}
    \vspace{1em}

    Don't Abstract Methods Prematurely.
    \customFigure[0.5]{../Images/L7_10.png}{}
\end{notes}
\newpage

\section{Into The Weeds (Technical Details)}
\subsection{Python Data Containers Overview}
\begin{definition}
    Choosing the right container for the task:
    \begin{itemize}
        \item \textbf{list:} Ordered, mutable sequence; versatile for collections of items.
        \item \textbf{tuple:} Ordered, immutable sequence; suitable for fixed collections and data integrity.
        \item \textbf{set:} Unordered collection of unique elements; efficient for membership testing and removing duplicates.
        \item \textbf{dict:} Key-value mappings; ideal for efficient data lookup and representing structured information.
    \end{itemize}
    \vspace{1em}
    
    \textbf{Data Structures:}
    \begin{itemize}
        \item \textbf{collections.NamedTuple}
        \item \textbf{dataclasses.Dataclass}
    \end{itemize}
    \vspace{1em}

    \textbf{Tensors}
    \begin{itemize}
        \item \textbf{numpy.ndarray}
        \item \textbf{torch.Tensor}
        \item \textbf{jax.Array}
    \end{itemize}
\end{definition}

\subsection{List Comprehensions}
\begin{notes}
    Provide concise list creation.
    \customFigure[0.5]{../Images/L7_11.png}{}
\end{notes}

\subsection{Leveraging Set Data Structures: Set}
\begin{notes}
    Efficiently handle unique elements and set logic.
    \customFigure[0.5]{../Images/L7_12.png}{}
\end{notes}

\subsection{Dictionaries for Key-Value Pairs}
\begin{notes}
    Enable efficient data lookup by keys.
    \customFigure[0.5]{../Images/L7_13.png}{}
\end{notes}

\subsection{Itertools for Efficient Iteration}
\begin{notes}
    Provides tools for very common iteration patterns.
    \begin{itemize}
        \item e.g. permutations, chunked, chain, zip, filter, product, combinations, etc.
    \end{itemize}
    \customFigure[0.5]{../Images/L7_14.png}{}
    \customFigure[0.5]{../Images/L7_15.png}{}
\end{notes}

\subsection{Functools for Functional Tools}
\begin{notes}
    Need to manipulate a function and get a new function. 
    \begin{itemize}
        \item Caching with \texttt{functools.lru\_cache}: Cache optimizes performance by storing results (i.e. return a previous result when the input of a function has already been observed)
    \end{itemize}
    \customFigure[0.5]{../Images/L7_16.png}{}
    \customFigure[0.5]{../Images/L7_17.png}{}
\end{notes}

\subsection{Abstract Base Classes (ABCs)}
\begin{notes}
    ABCs enforce interfaces for robust design.
    \customFigure[0.5]{../Images/L7_18.png}{}
\end{notes}

\subsection{Dataclasses for Data Storage}
\begin{notes}
    Data classes simplify data-centric class creation.
    \customFigure[0.5]{../Images/L7_19.png}{}
\end{notes}

\subsection{Python Typing Provides Hints}
\begin{notes}
    Type hints enhance code clarity, communicate intent and detect errors.
    \customFigure[0.5]{../Images/L7_20.png}{} 
\end{notes}

\subsection{Tensor Typing: jaxtyping}
\begin{notes}
    Communicate expected tensor shapes and data types. 
    \customFigure[0.5]{../Images/L7_21.png}{}
\end{notes}

\subsection{pytest: Helps you write better programs}
\begin{notes}
    Simplifies testing for reliable code.
    \customFigure[0.5]{../Images/L7_22.png}{}
\end{notes}

\subsection{Ideas}
\begin{summary}
    \begin{center}
        \begin{tabular}{l}
            \toprule
            \textbf{Exercise} \\
            \toprule
            Write research ideas. Get a mentor to rate them \\
            \midrule
            Ask other researchers about their taste \\
            \midrule
            Read about history of research ("The Structure of Scientific Revolutions") \\
            \midrule
            De-risk your ideas: Proactive idea evaluation mitigates research risks (kill fast, learn fast) \\
            \multicolumn{1}{p{\linewidth}}{
            \begin{enumerate}
                \item Identify potential bottlenecks
                \item Prioritize and commit $X$ amt. of time to exploring them
                \item Decide if you should continue or pivot
            \end{enumerate}} \\
            \bottomrule
        \end{tabular}
    \end{center}
    \begin{itemize}
        \item \href{https://colah.github.io/notes/taste/}{Research Taste}
    \end{itemize}
\end{summary}

\begin{warning}
    \begin{itemize}
        \item Getting attached to one direction. 
        \item Lack of research knowledge / intimacy.
        \item Environment is not supportive of your interests.
    \end{itemize}
\end{warning}
\newpage

\subsection{Code / Experiments}
\begin{summary}
    \begin{center}
        \begin{tabular}{ll}
            \toprule
            \textbf{Tools} & \textbf{Links} \\
            \toprule
            Artifacts to create/track w/ experiments & \\
            \multicolumn{2}{p{\linewidth}}{
            \begin{itemize}
                \item Data, code (scripts / modules), models (weights, configurations), results, predictions, plots, meeting notes, papers, documentation, etc.
            \end{itemize}} \\
            \midrule
            Git (Version Control): Enables effective tracking and collaborative changes & \href{https://rogerdudler.github.io/git-guide/}{Git Guide} \\
            \multicolumn{2}{p{\linewidth}}{
            \begin{itemize}
                \item Tracking changes, collaboration, backup, revert.
                \item Add, commit -m "message", push, pull, merge, diff, revert, branch, checkout, log, status, etc.
                \item .gitignore: Ignore files, directories, or patterns
            \end{itemize}} \\
            \midrule
            GitHub (Collaborative Code Hosting): Facilitates sharing and collaboration on code & \href{https://github.com/git-guides}{GitHub} \\
            \multicolumn{2}{p{\linewidth}}{
            \begin{itemize}
                \item Collaboration, sharing, open science, project hosting.
            \end{itemize}} \\
            \midrule
            Cookiecutter (Project Template): Standardizes project structure & \href{https://cookiecutter.readthedocs.io/en/1.7.2/}{Cookiecutter} \\
            & \href{https://github.com/drivendataorg/cookiecutter-data-science}{Repo} \\
            \multicolumn{2}{p{\linewidth}}{
            \begin{itemize}
                \item Logical, flexible, and reasonably standardized project structure for doing and sharing data science work.
                \item File Structure
                \begin{itemize}
                    \item data/: 
                    \begin{itemize}
                        \item external/: Data from third party sources.
                        \item interim/: Intermediate data that has been transformed.
                        \item processed/: The final, canonical data sets for modeling.
                        \item raw/: The original, immutable data dump.
                    \end{itemize}
                    \item src/: Source code for use in this project.
                    \begin{itemize}
                        \item \_\_init\_\_.py: Makes src a Python module.
                        \item config.py: Configuration settings.
                        \item dataset.py: Code to load data.
                        \item features.py: Code to build features.
                        \item modeling/: Code to train models.
                        \begin{itemize}
                            \item \_\_init\_\_.py: Makes modeling a Python module.
                            \item predict.py: Code to make predictions.
                            \item train.py: Code to train models.
                        \end{itemize}
                        \item plots.py: Code to create plots.
                    \end{itemize}
                    \item docs/: Documentation for this project.
                    \item models/: Trained and serialized models, model predictions, or model summaries.
                    \item notebooks/: Jupyter notebooks.  
                    \item references/: Data dictionaries, manuals, and all other explanatory materials.
                    \item reports/: Generated analysis as HTML, PDF, LaTeX, etc.
                    \begin{itemize}
                        \item figures/: Generated graphics and figures to be used in reporting.
                    \end{itemize}
                    \item pyproject.toml: Project information and dependencies.
                    \item requirements.txt: The requirements file for reproducing the analysis environment.
                    \item setup.cfg: Configuration file for setting up the project.
                    \item LICENSE: 
                    \item Makefile: 
                    \item README.md: 
                \end{itemize}
            \end{itemize}} \\
            \bottomrule
        \end{tabular}
    \end{center}
\end{summary}
\newpage

\begin{summary}
    \begin{center}
        \begin{tabular}{ll}
            \toprule
            \textbf{Tools} & \textbf{Links} \\
            \toprule
            Cookiecutter (Project Template): Standardizes project structure & \href{https://cookiecutter-data-science.drivendata.org/opinions/}{Opinions} \\
            \multicolumn{2}{p{\linewidth}}{
            \begin{itemize}
                \item \textbf{Design Philosophy:} Prioritizes conventions and reasonable defaults to streamline project setup. Opinions:
                \begin{itemize}
                    \item Data analysis is a DAG:
                    \begin{enumerate}
                        \item Raw data
                        \item Compute features
                        \item Plot analysis on raw data 
                        \item Train model
                        \item Compute statistics on features
                    \end{enumerate}
                    \item Raw data is immutable (i.e. never change raw data)
                    \begin{itemize}
                        \item \textbf{Dos:}
                        \begin{itemize}
                            \item Pipeline code: Process raw data $\rightarrow$ final analysis.
                            \item Cache outputs: Serialize or cache intermediate steps. 
                            \item Reproducible results: Enable full reproduction from code and raw data only.
                        \end{itemize}
                        \item \textbf{Don'ts:}
                        \begin{itemize}
                            \item Never edit raw data: Avoid manual edits or format changes
                            \item Never overwrite raw data: Do not replace raw data with processed data.
                            \item Single raw data version: Maintain only one version of raw data.
                        \end{itemize}
                    \end{itemize}
                    \item Data should (mostly) not be kept in source control
                    \begin{itemize}
                        \item GitHub warns for files over 50MB and rejects files over 100MB.
                        \item Use s3, azcopy, gcloud, drive to store data (i.e. cloud services)
                        \item Use cloudpathlib to access cloud data in the same way as pathlib to access local data.
                    \end{itemize}
                    \item Notebooks are for exploration and communication, source files are for repetitions.
                    \item Refactor the good parts into source code (\href{https://cookiecutter-data-science.drivendata.org/opinions/\#notebooks-are-for-exploration-and-communication-source-files-are-for-repetition}{Refactor Example}).
                    \begin{itemize}
                        \item Don't write code to do the same task in multiple notebooks.
                    \end{itemize}
                    \item Keep your modelling organized \href{https://pytorch.org/tutorials/beginner/basics/saveloadrun_tutorial.html}{(PyTorch Example)}
                    \begin{itemize}
                        \item Predictions (csv), training log (csv), stats (txt), model config / hyperparameters (json)
                    \end{itemize}
                    \item Build from the environment up \href{https://github.com/mamba-org/mamba}{(Mamba)}
                    \begin{itemize}
                        \item Use mamba rather than conda for faster environment management.
                        \item Create a environment.yml file to manage dependencies.
                    \end{itemize}
                \end{itemize}
            \end{itemize}} \\
            \bottomrule
        \end{tabular}
    \end{center}
\end{summary}
\newpage

\subsection{Writing / Analyzing}
\begin{summary}
    \begin{center}
        \begin{tabular}{ll}
            \toprule
            \textbf{Exercise} & \textbf{Links} \\
            \toprule
            Writing Skeleton & \href{https://esajournals.onlinelibrary.wiley.com/doi/full/10.1002/bes2.1258}{Step-by-Step Guide to Undergraduate Writing} \\
            & \href{https://www.nature.com/articles/d41586-018-02404-4}{How to write a first-class paper (Nature)} \\
            & \href{https://www.ncbi.nlm.nih.gov/pmc/articles/PMC5037950/}{Preparing Manuscript: Scientific Writing for Publication} \\
            \multicolumn{2}{p{\linewidth}}{
            \begin{enumerate}
                \item Start w/ Figures (How would you want to telll the story?)
                \item Write the structure (Introduction, Methods, Experiments and Results, Discussion)
                \item 2-3 sentence pitch for your idea
                \item Bullet points inside of each section (What are you expecting to cover?)
                \item Fill in text, repeat.
            \end{enumerate}} \\
            \midrule
            Figures: Move quick, perfect later & \\
            \multicolumn{2}{p{\linewidth}}{
            \begin{itemize}
                \item Figure \#1: Tells problem in simple way (30s elevator pitch)
                \item Figure \#2-3: Conceptual or data centric
                \begin{itemize}
                    \item How are you solving the problem?
                    \item What does the data look like?
                \end{itemize}
                \item Figure \#4-8: Quantitative evidence
                \item \textbf{Recommendations:}
                \begin{itemize}
                    \item Napkin/whiteboard figures first
                    \item Make good enough version w/ code (svg, png) using matplotlib, seaborn, etc.
                    \item Finetune w/ InkScape, Illustrator, GIMP, etc
                \end{itemize}
                \item \textbf{Anatomy of a Figure Examples (L8):} 
                \begin{itemize}
                    \item Slide 44: Task, Slide 45: Model + EDA
                    \item Slide 46-47: Quantitative evidence, Slide 48: Different ways of telling same story (e.g. tables or plots)
                \end{itemize}
            \end{itemize}} \\
            \midrule
            Pick good, consistent colors & \href{https://colorbrewer2.org/#type=sequential&scheme=BuGn&n=3}{ColorBrewer}, \href{https://matplotlib.org/stable/gallery/color/colormap_reference.html}{Matplotlib Colormaps} \\
            & \href{http://seaborn.pydata.org/tutorial/aesthetics.html}{Seaborn Palettes}, \href{https://www.youtube.com/watch?v=ze08gwVPaXk}{NeurIPS 18 Visualization for ML tutorial} \\
            \multicolumn{2}{p{\linewidth}}{
            \begin{itemize}
                \item Be mindful of how colours can help tell a story, accessibility is also important.
            \end{itemize}}\\
            \midrule
            Pair-writing (Como Pair-Coding) & \href{https://en.wikipedia.org/wiki/Rubber_duck_debugging}{Rubber Duck Debugging}, \href{https://en.wikipedia.org/wiki/Pair_programming}{Pair Programming} \\
            & \href{http://sunnyday.mit.edu/16.355/williams.pdf}{Pair Writing}, \href{https://pds.blog.parliament.uk/2017/03/29/pair-writing/}{Pair Writing in Government} \\
            \multicolumn{2}{p{\linewidth}}{
            \begin{itemize}
                \item Working together $\rightarrow$ Help communicate thoughts adn put you in a diff. attitude.
            \end{itemize}}\\
            \midrule
            Communal writing (Social pressure $\rightarrow$ accountable) & \href{https://gsas.harvard.edu/academics/writing}{Harvard Writing Center} \\
            \multicolumn{2}{p{\linewidth}}{
            \begin{enumerate}
                \item Setup an objective, measurable goal. 
                \item Set a time for writing period, take breaks
                \item Share progress at the end of each session, share writing stuggles if needed.
                \item Reflect if there are some reasons why it is hard to write.
            \end{enumerate}}\\
            \midrule
            Writing: Focus on quick iterations & \\
            \multicolumn{2}{p{\linewidth}}{
            \begin{itemize}
                \item Google docs and Paperpile (Copy DOI, paste, click, done) $\overset{\text{Export w/ bibtex}}{\Longrightarrow}$ LaTex (Overleaf)
            \end{itemize}}\\
            \midrule
            Interactive Apps & \href{https://streamlit.io/}{Streamlit}, \href{https://www.gradio.app/}{Gradio} \\
            & \href{https://www.huggingface.co}{Hugging Face}, \href{https://www.hf.co/spaces}{Hugging Face Spaces} \\
            & \href{https://github.com/eliahuhorwitz/Academic-project-page-template}{Academic Project Page Template} \\
            \bottomrule
        \end{tabular}
    \end{center}
\end{summary}



\newpage

\section{Symmetries, Tabular Data, Sets}

\begin{enumerate}
    \item \textbf{Assume prior:} 
    \[
    p_\Theta(\theta) \text{ or } f_\Theta(\theta) \quad \text{(Bayesian)}
    \]
    Observations: \( \bar{X} = x \).

    \item \textbf{LMS Estimator:}
    \begin{align*}
        \hat{\theta} &= g(x) = \mathbb{E}[\Theta \mid \bar{X} = x] \\
        \text{or } \hat{\Theta} &= g(\bar{X}) = \mathbb{E}[\Theta \mid \bar{X}].
    \end{align*}

    \textbf{Note:}
    \begin{itemize}
        \item \textbf{MAP:} Use the most probable \( \theta \) given \( x \).
        \item \textbf{LMS:} Use the expected value (conditional on \( \bar{X} = x \)) of \( \Theta \), i.e., the "Conditional Expectation Estimator."
    \end{itemize}

    \item \textbf{Unbiasedness of LMS Estimator:}
    \begin{align*}
        \mathbb{E}[\hat{\Theta}] &= \mathbb{E}[\mathbb{E}[\Theta \mid \bar{X}]] = \mathbb{E}[\Theta], \\
        \implies \mathbb{E}[\hat{\Theta} - \Theta] &= 0.
    \end{align*}

    \item \textbf{LMS Estimator Minimizes Conditional MSE:}
    \[
    \mathbb{E}\big[(\Theta - \hat{\Theta})^2 \mid \bar{X} = x \big].
    \]
    \textbf{Proof:}
    \begin{enumerate}
        \item First, suppose no observations: \( \hat{\Theta} \) is a constant.
        \begin{align*}
            \hat{\Theta} &= \arg\min_c \mathbb{E}\big[(\Theta - c)^2\big], \\
            0 &= \frac{d}{dc} \big[ -2\mathbb{E}[\Theta] + 2c \big], \\
            c &= \mathbb{E}[\Theta].
        \end{align*}

        \item Alternate view:
        \begin{align*}
            \mathbb{E}\big[(\Theta - c)^2\big] &= \mathrm{Var}[\Theta] + (\mathbb{E}[\Theta] - c)^2.
        \end{align*}
        To minimize: Set bias \( \mathbb{E}[\Theta] - c \) to zero.

        \item Now, with observations \( \bar{X} = x \):
        \begin{align*}
            \mathbb{E}\big[(\Theta - g(x))^2 \mid \bar{X} = x\big] &= \mathrm{Var}[\Theta \mid \bar{X} = x] + (\mathbb{E}[\Theta \mid \bar{X} = x] - g(x))^2.
        \end{align*}
        To minimize: Set \( g(x) = \mathbb{E}[\Theta \mid \bar{X} = x] \).
    \end{enumerate}

    \item \textbf{Conclusion:}
    \[
    \hat{\Theta} = g(x) = \mathbb{E}[\Theta \mid \bar{X} = x].
    \]
\end{enumerate}

\begin{enumerate}
    \item \textbf{Example: Prior Coin Toss Problem}
    \begin{align*}
        \hat{\theta}_{\text{LMS}} &= \mathbb{E}[\Theta \mid X = k] \\
        &= \frac{k + \alpha}{n + \alpha + \beta}.
    \end{align*}

    \item \textbf{Example: Prior Voltage Problem}
    \begin{enumerate}
        \item \textbf{Setup:}
        \begin{itemize}
            \item Unknown voltage \( \Theta \).
            \item Prior: \( \Theta \sim \text{Uniform}[0, 1] \).
            \item Volt meter reading \( Y \) given \( \Theta \): \( Y \sim \text{Uniform}[0, \Theta] \).
            \item Independent measurements: \( Y_1, \dots, Y_n \) given \( \Theta \).
        \end{itemize}

        \item \textbf{Likelihood:}
        \begin{align*}
            f_{Y \mid \Theta}(\mathbf{y} \mid \theta) &= \prod_{i=1}^n f_{Y}(y_i \mid \theta) \\
            &= \frac{1}{\theta^n} \cdot 1(\theta \geq \max_i y_i).
        \end{align*}

        \item \textbf{Posterior:}
        \begin{align*}
            f_{\Theta \mid Y}(\theta \mid \mathbf{y}) &= \frac{\frac{1}{\theta^n} \cdot 1(\theta \geq \max_i y_i)}{f_Y(\mathbf{y})}.
        \end{align*}

        \item \textbf{Estimators:}
        \begin{itemize}
            \item Maximum Likelihood (ML): 
            \[
            \hat{\theta} = \max_{1 \leq i \leq n} y_i.
            \]

            \item LMS:
            \[
            \hat{\theta} = \mathbb{E}[\Theta \mid Y = y] = \int_{0}^\infty \theta f_{\Theta \mid Y}(\theta \mid y) d\theta.
            \]
        \end{itemize}

        \item \textbf{Derivation for LMS:}
        \begin{enumerate}
            \item Compute \( f_Y(y) \) for \( n = 1 \):
            \begin{align*}
                f_Y(y) &= \int_y^1 \frac{1}{\theta} d\theta \\
                &= \ln(\theta) \Big|_y^1 \\
                &= -\ln(y).
            \end{align*}

            \item Compute \( \hat{\theta} \) for \( n = 1 \):
            \begin{align*}
                \hat{\theta} &= \int_y^1 \frac{\theta \cdot 1(\theta \geq y)}{-\ln(y)} d\theta \\
                &= \frac{1}{-\ln(y)} \int_y^1 \theta d\theta \\
                &= \frac{1}{-\ln(y)} \cdot \frac{y^2 - 1}{2}.
            \end{align*}
        \end{enumerate}

        \item \textbf{Graphical Interpretation:}
        \begin{itemize}
            \item \( f_{\Theta \mid Y}(\theta \mid y) = \frac{1}{-\ln(y)} \cdot 1(\theta \geq y) \cdot 1(0 \leq \theta \leq 1) \).
            \item The MAP estimator corresponds to the most probable \( \theta \).
            \item The LMS estimator minimizes the mean squared error, representing the "safest" choice.
        \end{itemize}
    \end{enumerate}
\end{enumerate}
\newpage

\section{CNN}
\begin{summary}
    In a \textbf{POMDPs}, we assume that: 
    \begin{itemize}
        \item environment modelled using state space, $\mathcal{S}$
        \item single agent
        \item $S_t$ = state after transition $t$
        \item $A_t$ = action inducing transition $t$
        \item stochastic state transitions with memoryless property:
        \[
        S_T \perp S_0, A_1, \dots, A_{T-1}, S_{T-2} \mid S_{T-1}, A_T
        \]
        \item $R_t$ = reward for transition $t$, i.e., $(S_{T-1}, A_T, S_T)$
        \item $O_t$ = observation of $S_t$
    \end{itemize}
    \vspace{1em}

    \begin{center}
        \begin{tabular}{ll}
            \toprule
            \textbf{Name} & \textbf{Function:} \\
            \midrule
            Initial state distribution & $p_0(s) := \mathbb{P}[S_0 = s]$ \\
            \midrule
            Transition distribution & $p(s'|s,a) := \mathbb{P}[S_t = s' | A_t = a, S_{t-1} = s]$ \\
            \midrule
            Reward function & $r(s,a,s') :=$ reward for transition $(s, a, s')$ \\
            \midrule
            Policy for choosing actions & $\pi_t(a | o_0, \dots, o_t) := \mathbb{P}[A_t = a | O_0 = o_0, \dots, O_t = o_t]$ \\
            \midrule
            Measurement model & $m(o | s) := \mathbb{P}[O_t = o | S_t = s]$ \\
            \bottomrule
        \end{tabular}
    \end{center}
\end{summary}
\newpage

\subsection{Bayesian Network}
\begin{notes}
    $S_0, O_0, A_1, R_1, S_1, O_1, A_2, R_2, S_2, O_2, \dots$ form a Bayesian network:
    \customFigure[0.5]{../Images/L10_0.png}{}
\end{notes}

\begin{example}
    
\end{example}
\newpage

\section{RNN}
\begin{summary}
    In a \textbf{Multi-Agent problem}, we assume that:
    \begin{itemize}
        \item Set of states for environment is $\mathcal{S}$
        \item $P$ agents within environment. 
        \item For each state $s \in \mathcal{S}$: 
        \begin{itemize}
            \item possible actions for agent $i$ is $\mathcal{A}_i(s)$
            \item set of action profiles is $\mathcal{A}(s) = \prod_{i=1}^P \mathcal{A}_i(s)$
        \end{itemize}
        \item possible state-action pairs are $\mathcal{T} = \{(s,a) \text{ s.t. } s \in \mathcal{S}, a \in \mathcal{A}(s)\}$
        \item environment in some origin state, $s_0$ 
        \item environment destroyed after $N$ transitions 
        \item agent $j$ wants to find policy $\pi_j (a_j \mid s)$ so that $\mathbb{E}[r_j(p)]$ is maximized
        \item agents act independently given the environmen
    \end{itemize}

    \begin{center}
        \begin{tabular}{ll}
            \toprule
            \textbf{Name} & \textbf{Function:} \\
            \midrule
            State transition given state-action pair defined by $\text{tr}: \mathcal{T} \to \mathcal{S}$ & $\text{tr}(s,a) = \text{state transition from $s$ under $a$}$ \\ 
            \midrule
            Reward to each agent, $i$ defined by $r_i$: $\mathcal{Q} \times \mathcal{S} \rightarrow \mathbb{R}_+$ & $r_i(s,a,\text{tr}(s,a)) = \text{rwd to agent $i$ for $(s,a,tr(s,a))$}$ \\
            \midrule
            State evolution of environment after $N$ transitions & $p = \langle (s_0,a^{(1)},s_{1}),\ldots,(s_{N-1},a^{(N)},s_{N})\rangle$ \\ 
            \multicolumn{2}{p{\linewidth}}{
            \begin{itemize}
                \item Given sequence of actions: $p.a = \langle a^{(1)},\ldots,a^{(n)}\rangle$
                \item $s_N = \tau (s_{n-1},a^{(n)})$
            \end{itemize}} \\
            \midrule
            reward to agent $i$ & $r_i(p) = \sum_{n=1}^N r_i (s_{n-1},a^{(n)}, s_n)$ \\
            \midrule
            expected-reward (value) of playing $a$ from $s$ for agent $j$ & $p(s'|s) := \mathbb{P}[S_{t+1} = s' | S_t = s]$ \\
            \midrule 
            Prob. that state of the env. after $T$ transitions is $s$ & $p_T(s) := \mathbb{P}[S_T = s]$ \\
            & $\quad \quad \; \; \; \;= \sum_{s'} p_{T-1}(s') p(s|s')$ \\
            \multicolumn{2}{p{\linewidth}}{
            \begin{itemize}
                \item $p_{T-1}(s')$: Prob. $s'$ at $T$-$1$ (given) 
                \begin{itemize}
                    \item $p_0(s)$: Base case
                \end{itemize}
                \item $p(s|s')$: Prob. $s$ given $s'$ (from graph)
            \end{itemize}} \\
            \bottomrule            
        \end{tabular}
    \end{center}
\end{summary}

\subsection{Action Equilibria}

\subsubsection{Finding Action Equilibria}

\subsection{Strategy Equilibria}

\subsubsection{Finding Strategy Equilibria}

\subsubsection{Existence of Stategy Equilibria}

\subsubsection{Convergence of Stategy Equilibria}

\subsection{Examples}
\subsubsection{Optimal Action Profiles}


\newpage

\section{GNN}
\subsection{Graphs as General Structured Data}
\begin{definition}
    \begin{itemize}
        \item Nodes (Vertices): Attributes 
        \begin{itemize}
            \item e.g. Node identity, \# of neighbors, etc.
            \customFigure[0.25]{../Images/L12_0.png}{}
        \end{itemize}
        \item Edges (Link): Attributes and Directions
        \begin{itemize}
            \item e.g. edge identity, edge weight, etc.
            \customFigure[0.25]{../Images/L12_1.png}{}
        \end{itemize}
        \item Global (master node): Attributes
        \begin{itemize}
            \item e.g. \# of nodes, longest path, etc.
            \customFigure[0.25]{../Images/L12_2.png}{}
        \end{itemize}
    \end{itemize}
\end{definition}

\subsubsection{What Info Can Store in a Graph?}
\begin{notes} Add vector (embeddings) to each of the graph components.
    \begin{itemize}
        \item Nodes (Vertices): Embeddings
        \item Edges (Link): Attributes and Embeddings
        \begin{itemize}
            \item Undirected and directed edges
        \end{itemize}
        \item Global (master node): Embeddings
    \end{itemize}
    \customFigure[0.25]{../Images/L12_3.png}{}
\end{notes}
\newpage

\subsection{Examples of Graphs}
\begin{example}
    \begin{enumerate}
        \item Maps
        \item Code, algorithms, mathematical formulas, neural networks 
        \item Images: Pixels connected to their neighboring pixels ("Locality" inductive bias)
        \customFigure[0.5]{../Images/L12_4.png}{}
        \item Test: Chain of words or characters ("Sequential" inductive bias)
        \customFigure[0.5]{../Images/L12_5.png}{}
        \item Molecules: Atoms are nodes, covalent bonds are edges
        \customFigure[0.5]{../Images/L12_6.png}{}
        \item Social Networks: People can be nodes, and their interactions are edges. 
        \customFigure[0.5]{../Images/L12_7.png}{}
    \end{enumerate}
\end{example}
\newpage

\subsection{Graph Problems}
\begin{notes}
    \begin{itemize}
        \item Global tasks:
        \customFigure[0.5]{../Images/L12_8.png}{}
        \item Node tasks:
        \customFigure[0.5]{../Images/L12_9.png}{}
        \item Edge tasks:
        \customFigure[0.5]{../Images/L12_10.png}{}
    \end{itemize}
\end{notes}
\newpage

\subsubsection{Representing Graphs Numerically, Mathematically, Programmatically}
\begin{notes} Represent the connectivity of the graph.
    \begin{itemize}
        \item Numerically 
        \begin{itemize}
            \item Adjacency Matrix: $A \in \mathbb{R}^{N \times N}$
            \customFigure[0.5]{../Images/L12_11.png}{}
            \item Adjacency List: $A \in \mathbb{R}^{N \times E}$
            \customFigure[0.5]{../Images/L12_12.png}{}
        \end{itemize}
        \item Mathematically: Structured tensors $G = (X,E,A,U)$
        \item Programmatically: Graph data structure
        \customFigure[0.5]{../Images/L12_13.png}{}
    \end{itemize}
\end{notes}
\newpage

\subsection{GNN}
\begin{summary}
    \begin{center}
        \begin{tabular}{ll}
            \toprule
            \textbf{Concept} & \textbf{Description} \\
            \midrule
            \textbf{Prediction Methods} & \\
            Node to node & Take nodes and pass in our favorite classifer \\
            \multicolumn{2}{p{\linewidth}}{\begin{center}
                \customFigure[0.5]{../Images/L12_15.png}{}
                \vspace{-4em}
            \end{center}} \\
            Edge to node & Passing info from one attribute to another requires a pooling op. \\
            \multicolumn{2}{p{\linewidth}}{\begin{center}
                \customFigure[0.5]{../Images/L12_16.png}{}
                \vspace{-4em}
            \end{center}} \\
            Node to edges & Passing info from one attribute to another requires a pooling op. \\
            \multicolumn{2}{p{\linewidth}}{\begin{center}
                \customFigure[0.5]{../Images/L12_18.png}{}
                \vspace{-4em}
            \end{center}} \\
            \midrule
        \end{tabular}
    \end{center}
\end{summary}
\newpage

\begin{summary}
    \begin{center}
        \begin{tabular}{ll}
            \toprule
            \textbf{Concept} & \textbf{Description} \\
            \midrule
            \textbf{Pooling from Edges to Nodes} & Route info b/w diff. parts and aggregate them. \\
            \multicolumn{2}{p{\linewidth}}{\begin{center}
                \customFigure[0.5]{../Images/L12_17.png}{}
                \vspace{-4em}
            \end{center}} \\
            \midrule
            \textbf{Prediction Pipeline} & $\textbf{Graph} \rightarrow \text{Transform} \rightarrow \text{Predict on attribute of interest}$ \\
            \multicolumn{2}{p{\linewidth}}{\begin{center}
                \customFigure[0.5]{../Images/L12_19.png}{}
                \vspace{-4em}
            \end{center}} \\
        \end{tabular}
    \end{center}
\end{summary}
\newpage

\begin{summary}
    \begin{center}
        \begin{tabular}{ll}
            \toprule
            \textbf{Concept} & \textbf{Description} \\
            \midrule
            \textbf{Message passing} & Pool information, aggregate it, transform it, and update it \\
            \multicolumn{2}{p{\linewidth}}{\begin{center}
                \customFigure[0.5]{../Images/L12_20.png}{}
                \vspace{-4em}
            \end{center}} \\
            Propogation pattern of message passing & After a few layers we get increasingly more complex patterns of info. \\
            \multicolumn{2}{p{\linewidth}}{\begin{center}
                \customFigure[0.5]{../Images/L12_21.png}{}
                \vspace{-4em}
            \end{center}} \\
            \midrule
            \textbf{Conditioning Info} & Many ways of adding context to a specific part of a graph \\
            \multicolumn{2}{p{\linewidth}}{\begin{center}
                \customFigure[0.5]{../Images/L12_23.png}{}
                \vspace{-4em}
            \end{center}} \\
            \midrule
        \end{tabular}
    \end{center}
\end{summary}
\newpage

\subsection{Graph Topics}
\begin{summary}
    \begin{center}
        \begin{tabular}{ll}
            \toprule
            \textbf{Concept} & \textbf{Description} \\
            \midrule
            Learning subgraph representations & Learning a function that works on portions of a graph \\
            \multicolumn{2}{p{\linewidth}}{\begin{center}
                \customFigure[0.5]{../Images/L12_25.png}{}
                \vspace{-4em}
            \end{center}} \\
            \midrule
            Batching in graphs & Very context dependent, no general solution \\
            \multicolumn{2}{p{\linewidth}}{\begin{center}
                \customFigure[0.5]{../Images/L12_26.png}{}
                \vspace{-4em}
            \end{center}} \\
            \midrule
        \end{tabular}
    \end{center}
\end{summary}
\newpage

\begin{summary}
    \begin{center}
        \begin{tabular}{ll}
            \toprule
            \textbf{Concept} & \textbf{Description} \\
            \midrule
            Heterogeneous graphs & Different types of nodes and edges \\
            \multicolumn{2}{p{\linewidth}}{\begin{center}
                \customFigure[0.5]{../Images/L12_27.png}{}
                \vspace{-4em}
            \end{center}} \\
            \midrule
            Interpretability & Many ways to extract information from graphs \\
            \multicolumn{2}{p{\linewidth}}{\begin{center}
                \customFigure[0.5]{../Images/L12_28.png}{}
                \vspace{-4em}
            \end{center}} \\
            \midrule
            Attention & Transformers can be viewed as a GNN on a fully connected graph \\
            \multicolumn{2}{p{\linewidth}}{\begin{center}
                \customFigure[0.5]{../Images/L12_29.png}{}
                \vspace{-4em}
            \end{center}} \\
            \midrule
        \end{tabular}
    \end{center}
\end{summary}

\subsection{Examples}

\subsubsection{Message passing / GraphNets on a small graph}
\begin{example}
    \customFigure[0.5]{../Images/L12_24.png}{}
\end{example}


\end{document}