\subsection{Probably Approximately Correct (PAC) Estimations}
\begin{motivation}
    More than one fcn may be consistent w/ the data, how to find the best one?
\end{motivation}

\subsubsection{Hoeffding's Inequality}
\begin{motivation}
    Bound $|\mu - \nu|$ w.r.t. $N$.
\end{motivation}

\begin{definition}
    For any $\epsilon > 0$,
    \begin{equation}
        \mathbb{P}(|\nu - \mu| \geq \epsilon) \leq 2e^{-2\epsilon^2N}
    \end{equation}
    \begin{itemize}
        \item $\mu$: Probabillity of an event.
        \item $\nu$: Relative frequency in a sample size $N$.
        \item $\epsilon$: Tolerance (i.e. how close we want $\nu$ to be to $\mu$).
        \begin{itemize}
            \item $\epsilon \rightarrow 0$: $\nu = \mu$
        \end{itemize}
        \item $\mu \overset{?}{\approx} \nu $: $\mu$ is probably approximately equal to $\nu$. As $N \rightarrow \infty$: $\nu \rightarrow \mu$
    \end{itemize}
\end{definition}

\begin{warning}
    Approx. the true dist. w/ high prob. by taking a large enough $N$ (i.e. empirical dist. converges to true dist.).
    \begin{itemize}
        \item i.e. Probability of a sig. deviation shrinks exp. w/ $N$.
    \end{itemize}
\end{warning}

\subsection{PAC Learning}
\subsubsection{Error}
\begin{definition}
    \begin{itemize}
        \item \textbf{Out-Sample Error:}
        \begin{equation*}
            E_{\text{out}} = \mathbb{P}[f \neq h]
        \end{equation*}
        \item \textbf{In-Sample Error:}
        \begin{equation*}
            E_{\text{in}} = \frac{1}{N} \sum_{i=1}^{N} \mathbb{I}[f(x^{(i)}) \neq h(x^{(i)})]
        \end{equation*}
    \end{itemize}
\end{definition}

\subsubsection{Union Bound Theorem}
\begin{theorem}
    The prob. of at least one of the events $E_1, \ldots, E_M$ occurring is bounded by the sum of the prob. of each event occurring:
    \begin{equation*}
        \mathbb{P} \left[E_1 \lor \cdots \lor E_M \right] \leq \sum_{i=1}^{M} \mathbb{P}[E_i]
    \end{equation*}
\end{theorem}

\begin{notes}
    \begin{itemize}
        \item If the events are mutually exclusive, then the union bound is tight (i.e. equality holds).
        \item If the events are highly correlated, then the union bound is loose (i.e. inequality holds)
        \begin{itemize}
            \item Some events may be more likely to occur together.
        \end{itemize}
    \end{itemize}
\end{notes}
\newpage

\subsubsection{Generalization of Hoeffding's Inequality}
\begin{definition}
    Assuming that $h$ is chosen from a set of hypotheses $\mathcal{H}$, derive a (loose) upper-bound on $|E_{\text{out}} - E_{\text{in}}|$:
    \begin{align*}
        \mathbb{P} \left[ \bigvee_{h \in \mathcal{H}} \left( |E_{\text{out}} - E_{\text{in}}(h)| > \varepsilon \right) \right]
        &\leq \sum_{h \in \mathcal{H}} \mathbb{P} \left[ |E_{\text{out}} - E_{\text{in}}(h)| > \varepsilon \right] \\
        &\leq \sum_{h \in \mathcal{H}} 2e^{-2\varepsilon^2 N} \\
        &= 2 |\mathcal{H}| e^{-2\varepsilon^2 N} 
    \end{align*}
    \begin{itemize}
        \item Endow $\mathcal{F}$ (i.e. fcn space) w/ prob. distribution, $P : \mathcal{X} \to [0,1]$, then 
        \begin{itemize}
            \item $E_{\text{out}}$ (i.e. true error of a hyp. over entire dist. of data) is analogous to $\mu$ 
            \item $E_{\text{in}}(h)$ (i.e. empirical error of hyp. on a finite sample) is analogous to $\nu$. 
        \end{itemize}
    \end{itemize}
\end{definition}

\begin{notes}
    \begin{itemize}
        \item $E_{\text{in}}(h) \stackrel{?}{\approx} E_{\text{out}}$ requires small $|\mathcal{H}|$ (generalization)
        \begin{itemize}
            \item Look at inequality, small $|\mathcal{H}|$ $\rightarrow$ small $E_{\text{out}} - E_{\text{in}}$ (i.e. prevents overfitting but leads to underfitting)
        \end{itemize}
        \item $E_{\text{in}}(h) \approx 0$ requires large $|\mathcal{H}|$ (discrimination)
        \begin{itemize}
            \item Need large $|\mathcal{H}|$ to capture the true dist. (i.e. prevents underfitting but leads to overfitting)
        \end{itemize}
    \end{itemize}
\end{notes}
\newpage

\begin{example}
    \begin{enumerate}
        \item \textbf{Given:} An opaque box containing \textcolor{red}{red} and \textcolor{blue}{blue} balls. Take $N$ IID samples.
        \begin{itemize}
            \item $\mu$: Probability of drawing a \textcolor{blue}{blue} balls (unknown).
            \item $\nu$: Relative frequency of \textcolor{blue}{blue} balls in the sample (known).
        \end{itemize}
        \item \textbf{Problem 1:} What is $\nu$ in this case? 8 balls total, 5 are blue. 
        \item \textbf{Solution 1:} $\nu = \frac{5}{8}$
        \item \textbf{Problem 2:} How to partition $\mathcal{F}$ into regions where \textcolor{blue}{$f=h$} and \textcolor{red}{$f \neq h$}?
        \item \textbf{Solution 2:} 
        \customFigure[0.5]{../Images/L5_8.png}{LS $h$, MS $f$}
        \item \textbf{Problem 3:} What is the out-sample error?
        \item \textbf{Solution 3:} In words, the probability of the hypothesis being wrong.
        \customFigure[0.3]{../Images/L5_9.png}{}
        \item \textbf{Problem 4:} What is the in-sample error given this sample of 11 balls s.t. $f=h$, $1$ ball s.t. $f \neq h$?
        \item \textbf{Solution 4:} $E_{\text{in}} = \frac{1}{12}$
        
    \end{enumerate}
\end{example}



