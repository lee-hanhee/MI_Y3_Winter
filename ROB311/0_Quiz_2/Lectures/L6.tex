\begin{definition}
    Vertices represent random variables and edges represent dependencies between variables.
\end{definition}

\subsection{Junction}
\begin{definition}
    A \textbf{junction} $\mathcal{J}$ consists of three vertices, $X_1$, $X_2$, and $X_3$, connected by two edges, $e_1$ and $e_2$:
    \customFigure[0.5]{../Images/L6_2.png}{}
    \begin{itemize}
        \item $X_1$ and $X_2$ are not independent, $X_2$ and $X_3$ are not independent, but when is $X_1$ and $X_3$ independent?
    \end{itemize}
\end{definition}

\subsubsection{Causal Chain}
\begin{definition}
    A causal chain is a junction $\mathcal{J}$ s.t. 
    \customFigure[0.5]{../Images/L6_3.png}{}
    \begin{itemize}
        \item $X_1$ and $X_3$ are not independent (unconditionally), but are independent given $X_2$.
    \end{itemize}
\end{definition}

\begin{notes}
    \begin{itemize}
        \item \textbf{Analogy:}  Given $X_2$, $X_1$ and $X_3$ are independent. Why? $X_2$'s door closes when you know $X_2$, so $X_1$ and $X_3$ are independent.
        \item \textbf{Distinction b/w Causal and Dependence:} $X_1$ and $X_2$ are dependent. However, from a causal perspective, $X_1$ is influencing $X_2$ (i.e. $X_1 \rightarrow X_2$).
    \end{itemize}
\end{notes}

\subsubsection{Common Cause}
\begin{definition}
    A common cause is a junction $\mathcal{J}$ s.t.
    \customFigure[0.5]{../Images/L6_4.png}{}
    \begin{itemize}
        \item $X_1$ and $X_3$ are not independent (unconditionally), but are independent given $X_2$.
    \end{itemize}
\end{definition}

\begin{notes}
    \begin{itemize}
        \item \textbf{Analogy:} Given $X_2$, $X_1$ and $X_3$ are independent. Why? Consider the following example:
        \begin{itemize}
            \item Let $X_2$ represent whether a person smokes or not, $X_1$ represent whether they have yellow teeth, $X_3$ represent whether they have lung cancer.
        \end{itemize}
        \item Without knowing $X_2$, observing $X_1$ provides information about $X_3$ because yellow teeth are associated with smoking, which in turn increases the likelihood of lung cancer. 
        \item If $X_2$ is known, then knowing whether a person has yellow teeth provides no additional information about whether they have lung cancer beyond what is already known from smoking status. 
    \end{itemize}
\end{notes}


\subsubsection{Common Effect}
\begin{definition}
    A common effect is a junction $\mathcal{J}$ s.t.
    \customFigure[0.5]{../Images/L6_5.png}{}
    \begin{itemize}
        \item $X_1$ and $X_3$ are independent (unconditionally), but are not independent given $X_2$ or any of $X_2$'s descendents.
    \end{itemize}
\end{definition}

\begin{notes}
    \begin{itemize}
        \item \textbf{Analogy:} Consider the following example:
        \begin{itemize}
            \item Let $X_2$ represent whether the grass is wet, $X_1$ represent whether it rained, $X_3$ represent whether the sprinkler was on.
        \end{itemize}
        \item Without knowing whether the grass is wet ($X_2$), the occurrence of rain ($X_1$) and the sprinkler being on ($X_3$) are independent events. The rain may occur regardless of the sprinkler, and vice versa.
        \item However, once we observe that the grass is wet ($X_2$), the two events become dependent:
        \begin{itemize}
            \item If we learn that the sprinkler was not on, then the wet grass must have been caused by rain.
            \item If we learn that it did not rain, then the wet grass must have been caused by the sprinkler.
        \end{itemize}
    \end{itemize}
\end{notes}
\newpage

\subsection{Dependence Separation}
\subsubsection{Blocked}
\begin{definition}
    $\mathcal{J} = (\{X_1, X_2, X_3\}, \{e_1, e_2\})$ is \textbf{blocked} given $\mathcal{K} \subseteq \mathcal{V}$ if $X_1$ and $X_3$ are independent given $\mathcal{K}$.
\end{definition}

\subsubsection{Blocked Undirected Path}
\begin{definition}
    An undirected path,
    \[
    p = \langle (X_1, e_1, X_2), \dots, (X_{|p|-1}, e_{|p|-1,|p|}, X_{|p|}) \rangle,
    \]
    \customFigure[0.5]{../Images/L6_13.png}{}
    is \textbf{blocked} given $\mathcal{K} \subseteq \mathcal{V}$ if any of its junctions,

    \[
    \mathcal{J}^{(n)} = \{(X_{n-1}, X_n, X_{n+1}), (e_{n-1}, e_n)\},
    \]
    is blocked given $\mathcal{K}$.
\end{definition}

\subsubsection{Independence}
\begin{theorem}
    Any two variables, $X_1$ and $X_2$, in a Bayesian network, $\mathcal{B} = (\mathcal{V}, \mathcal{E})$, are independent given $\mathcal{K} \subseteq \mathcal{V}$ if every undirected path is blocked.
\end{theorem}

\subsubsection{Consequence of Dependence Separation}
\begin{theorem}
    For any variable, $X \in \mathcal{V}$, it can be shown that $X$ is independent of $X$'s non-descendants, $\mathcal{V} \setminus \operatorname{des}(X)$, given $X$'s parents, $\operatorname{pts}(X)$.
\end{theorem}

\begin{notes}
    \customFigure[0.5]{../Images/L6_8.png}{}
    \begin{itemize}
        \item Given $X$'s parent, apply junction rules to determine that $X$ is independent of its non-descendants.
        \item $\mathcal{J} = \{(Z_1, \text{pt}(X), X), (e_1,e)\}$ shows that $Z_1$ and $X$ are independent given $\text{pt}(X)$ (causal chain).
        \item $\mathcal{J} = \{(Z_2, \text{pt}(X), X), (e_2,e)\}$ shows that $Z_2$ and $X$ are independent given $\text{pt}(X)$ (common cause).
        \item Given $\text{ch}(X)$'s parent, apply junction rules to determine that $\text{ch}(X)$ is independent of its non-descendants.
        \item $\mathcal{J} = \{\text{pt}(X),X, \text{ch}(X)), (e,e')\}$ shows that $\text{pt}(X)$ and $\text{ch}(X)$ are independent given $X$ (causal chain).
        \item Given $Z_4$'s parent, apply junction rules to determine that $Z_4$ is independent of its non-descendants.
        \item $\mathcal{J} = \{X, \text{ch}(X), Z_4, (e',e_4)\}$ shows that $X$ and $Z_4$ are independent given $\text{ch}(X)$ (causal chain).
        \item CHECK THIS OVER AGAIN WITH THE PROFESSOR.
    \end{itemize}
\end{notes}
\newpage

