\subsection{Bayesian Network}
\begin{definition}
    Vertices represent random variables and edges represent dependencies between variables.
\end{definition}

\subsubsection{Junction}
\begin{definition}
    A \textbf{junction} consists of three vertices, $X_1$, $X_2$, and $X_3$, connected by two edges, $e_1$ and $e_2$:
    \begin{itemize}
        \item Both arrows pointing in one direction
        \item Both arrows pointing in opposite directions
        \item One arrow pointing in each direction
    \end{itemize}
\end{definition}

\begin{warning}
    Want to look for causal relationships. 
    Arrows, what's causing what, what's influencing what.
\end{warning}

\subsubsection{Causal Chain}
\begin{definition}
    A causal chain is a junction of the following form:
    \begin{itemize}
        \item $X_1$ and $X_2$ are dependent. $X_2$ is dependent on $X_1$. Vice versa. From a causal perspective, $X_1$ is influencing $X_2$. Subtle difference, just bc $X_1 \rightarrow X_2$.
        \item $X_2$ and $X_3$ are dependent.
        \item $X_1$ and $X_3$ are dependent. 
        \begin{itemize}
            \item Given $X_2$, $X_1$ and $X_3$ are independent. Why? $X_2$'s door closes when you know $X_2$, so $X_1$ and $X_3$ are independent.
        \end{itemize}
    \end{itemize}
\end{definition}

\begin{warning}
    $X_1$ is influeincing $X_2$ and $X_2$ is influencing $X_3$.
\end{warning}

\subsubsection{Common Cause}
\begin{definition}
    A common cause is a junction of the form: 
\end{definition}

\begin{notes}
    \begin{itemize}
        \item $X_1$ and $X_3$ are dependent. 
        \begin{itemize}
            \item Given $X_2$, $X_1$ and $X_3$ are independent. Why? $X_2$ whether you smoke or not, $X_1$ whether you have yellow teeth, $X_3$ whether you have lung cancer, if you don't know $X_2$, if they have yellow teeth, then they might smoke, then they might have lung cancer. If you know $X_2$, yellow teeth and lung cancer are independent b/c you already know if they smoke or not, and yellow teeth implies smoke, 
        \end{itemize}
    \end{itemize}
\end{notes}

\subsubsection{Common Effect}
\begin{definition}
    A common effect is a junction of the form:
    
\end{definition}

\begin{notes}
    \begin{itemize}
        \item $X_1$ and $X_3$ are independent. 
        \item Given $X_2$ or any of $X_2$'s descendents, $X_1$ and $X_3$ are dependent.
    \end{itemize}
\end{notes}

\begin{warning}
    Just b/c you don't know something about the middle variable, then it can be independent
\end{warning}

\begin{example}
    $X_2$ Grass being wet, $X_1$ raining, and $X_3$ sprinkler being on.
    \begin{itemize}
        \item If you know the grass is wet, you know that either the sprinkler is on or it's raining.
        \begin{itemize}
            \item If it didn't have the sprinkler on, then it must have rained.
            \item If it didn't rain, then the sprinkler must have been on.
            \item So this means that $X_1$ and $X_3$ are dependent given $X_2$.
        \end{itemize}
        \item If you don't know the grass is wet, then $X_1$ and $X_3$ are independent b/c you don't know if it rained or the sprinkler was on.
    \end{itemize}
\end{example}

\begin{example}
    \begin{enumerate}
        \item \textbf{Given:} Caveman is deciding whether to go hunt for meat. He must take into account several factors:
        \begin{itemize}
            \item Weather
            \item Possibility of over-exertion
            \item Possibility encountering lion
        \end{itemize}

        These factors can result in Cavemen's death. His decision will ultimately depend on the \textbf{chances} of his death.
        \item \textbf{Binary Variables:}
        \begin{itemize}
            \item $W = \{\text{Sun}, \text{Rainy}\}$: Weather
            \item $H$: Whether the Cavemen goes hunting or not.
            \item $L$: Whether the Cavemen encounters a lion or not.
            \item $T$: Whether the Cavement is tired or not.
            \item $D$: Whether the Cavemen dies or not
        \end{itemize}
        \item \textbf{Problem:} Cavemen must decide whether to go hunting or not. 
        \begin{itemize}
            \item He must consider the conditional probabilities (i.e. dependence) of each event.
        \end{itemize}
    \end{enumerate}
\end{example}

\begin{warning}
    Have to be discrete. 
\end{warning}
\newpage

\begin{example}
    \begin{enumerate}
        \item \textbf{Given:} Bayesian network.
        \customFigure[0.25]{../Images/L6_7.png}{}
        \item \textbf{Problem:} $A$ and $E$ are 
        \begin{itemize}
            \item independent if $\mathcal{K}=$
            \item not necessarily independent for $\mathcal{K}=$
        \end{itemize}
    \end{enumerate}
\end{example}

\begin{process}
    \begin{enumerate}
        \item 
    \end{enumerate}
\end{process}

\begin{example} \textbf{Bayesian Inference}
    \begin{enumerate}
        \item \textbf{Given:}
        \item \textbf{Problem:}
    \end{enumerate}
\end{example}

\begin{process}
    \begin{enumerate}
        \item \textbf{Given:} 
        \customFigure[0.5]{../Images/L6_9.png}{}
    \end{enumerate}
\end{process}

\begin{example} \textbf{Inference via Sampling}
    \begin{enumerate}
        \item \textbf{Given:}
        \item \textbf{Problem:}
    \end{enumerate}
\end{example}

