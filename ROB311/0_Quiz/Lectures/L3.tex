\begin{summary}
    
\end{summary}

\begin{example}
    Different ways to formulate the CSP problem. 
    \begin{itemize}
        \item How can you formualte the CSP problem in a different way? Can I get a specific example?
        \begin{itemize}
            \item The domain could be set to everything, then set the constraints later.
        \end{itemize}
        \item What is the constraint graph showing? Grouping the variables
        \item How do you check consistency in a CSP?
        \item Why can you use any search algorithm when you formulate this as a search problem? 
        \item What does a node contain? A node contans a path. 
        \begin{itemize}
            \item How does that match the example on slide 10. It does. 
        \end{itemize}
        \item Why is formulalting a CSP problem as a search problem a bad idea? B/c you have to search through all possible combinations, but if you find a constraint then you can prune the search space.
        \begin{itemize}
            \item A lot easier to see if there is a solution or not. But in a search problem, you see if there's a solution and how to get to it. 
        \end{itemize}
    \end{itemize}
\end{example}

\begin{summary}
    Want a way to learn heuristics.
\end{summary}