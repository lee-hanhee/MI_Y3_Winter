\documentclass[5pt]{extarticle} % Note the extarticle document class
\usepackage[margin=0.05in]{geometry} % Set 0.3-inch margins
\usepackage{multicol}              % For multi-column support
\usepackage{lipsum}                % Dummy text generator (optional)
\usepackage{amssymb} % math symbols
\setlength{\parskip}{0ex}
\setlength\parindent{0pt} % no indent
%% optional packages -- documentation at ctan.org
\usepackage{graphicx}  % image handline
\usepackage{amsmath}   % enhanced equation environments
\usepackage{tikz}      % block diagrams
\usetikzlibrary{positioning}  % allow relative positioning of tikz elements
\usepackage{pgfplots}  % package for plots, based on tikz
\usepackage{hyperref}
\usepackage{paracol}             % Import paracol package
\usepackage{float}

\newcommand{\customFigure}[3][]{%
    \vspace{-1.5em}
    \begin{figure}[H]
        \centering
        \includegraphics[width=#1\textwidth]{#2}
    \end{figure}
    \vspace{-1.5em}
}

\begin{document}
\begin{paracol}{3}
    {\tiny

    \textcolor{red}{\textbf{Modelling CS}} $u$: control input, $y$: plant output

    \textcolor{blue}{\textbf{State variable}} CS is in state variable form if \\ 
    $\overset{\cdot}{x}_1 = f_1 (t,x_1,\ldots,x_n,u), \ldots, \overset{\cdot}{x}_n = f_n (t,x_1,\ldots,x_n,u)$ \\
    $y=h(t,x_1,\ldots,x_n,u)$ is a collection of $n$ 1st order ODEs. 

    \textcolor{blue}{\textbf{Time-Invariant (TI)}} CS is TI if $f_i(\cdot)$ does not depend on $t$.

    \textcolor{blue}{\textbf{State space (SS)}} TI CS is in SS form if $ \dot{x} = f(x, u), y = h(x, u)$ where $x(t) \in \mathbb{R}^n$ is called the state.

    \textcolor{blue}{\textbf{Single-input-single-output (SISO)}} CS is SISO if $u(t), y(t) \in \mathbb{R}$.

    \textcolor{blue}{\textbf{LTI}} CS in SS form is LTI if $ \dot{x} = Ax + Bu, \; y = Cx + Du$ \\
    *$A \in \mathbb{R}^{n \times n}, \; B \in \mathbb{R}^{n \times m}, \; C \in \mathbb{R}^{p \times n}, \; D \in \mathbb{R}^{p \times m}$ \\ 
    *SISO: $p=1, \; m=1$ \\
    *$ x(t) \in \mathbb{R}^n, \; u(t) \in \mathbb{R}^m, \; y(t) \in \mathbb{R}^p. $

    \textcolor{blue}{\textbf{Input-Output (IO)}} LTI CS is in IO form if \\
    $ \frac{d^n y}{dt^n} + a_{n-1} \frac{d^{n-1} y}{dt^{n-1}} + \cdot + a_1 \frac{dy}{dt} + a_0 y = b_m \frac{d^m u}{dt^m} + \cdot + b_1 \frac{du}{dt} + b_0 u $ \\
    *$ m \leq n$ (causality)

    \textcolor{orange}{\textbf{IO to SS Model}} 1.  Define $x$ s.t. highest order derivative in $\overset{\cdot}{x}$ \\
    2.1 If LTI, then \\
    *Write $\overset{\cdot}{x} = Ax + Bu = f(x,u)$ by isolating for components of $\overset{\cdot}{x}$\\ 
    *Write $y = Cx + Du = h(x,u)$ by setting measurement output $y$ to component of $x$ \\
    2.2 If not LTI, then \\
    *Write $\overset{\cdot}{x} = f(x,u)$ by isolating for components of $\overset{\cdot}{x}$ \\
    *Write $y = h(x,u)$ by setting measurement output $y$ to component of $x$

    \textcolor{red}{\textbf{Equilibria}} $y_d$ (steady state) b/c if $y(0) = y_d$ at $t=0$, then $y(t) = y_d \; \forall t \geq 0$.

    \textcolor{blue}{\textbf{Equilibrium pair}} Consider the system $\overset{\cdot}{x} = f(x,u)$. The pair $(\bar{x}, \bar{u})$ is an equilibrium pair if $f(\bar{x}, \bar{u}) = 0$.

    \textcolor{blue}{\textbf{Equilibrium point}} $\bar{x}$ is an equilibrium point w/ control $u=\bar{u}$. \\
    *If $u=\bar{u}$ and $x(0) = \bar{x}$ then $x(t) = \bar{x} \; \forall t \geq 0$ (i.e. a system that starts at equilibrium remains at equilibrium).

    \textcolor{orange}{\textbf{Find Equilibrium Pair/Point}} 1. Set $f(x,u) = 0$ \\
    2. Solve $f(x,u) = 0$ to find $(x,u) = (\bar{x}, \bar{u})$. \\
    3. If specific $u = \bar{u}$, then find $x = \bar{x}$ by solving $f(x,\bar{u}) = 0$.

    \textcolor{red}{\textbf{Linearization of Nonlinear System}} Consider system $\overset{\cdot}{x} = f(x,u)$ w/ equ. pair $(\bar{x}, \bar{u})$, then error coordinates around equ. pair $\delta x = x - \bar{x}$, $\delta u \text{=} u - \bar{u}$, $\delta y \text{=} y - h(\bar{x},\bar{u})$ $\delta \overset{\cdot}{x} \text{=} \overset{\cdot}{x} - f(\bar{x},\bar{u})$ w/ \\
    $\delta \overset{\cdot}{x} = A \delta x + B \delta u$, $A = \frac{\partial f (\bar{x}, \bar{u})}{\partial \underline{x}}  \in \mathbb{R}^{n_1 \times n_1}$, $B = \frac{\partial f (\bar{x}, \bar{u})}{\partial u} \in \mathbb{R}^{n_1}$, \\
    $\delta y = C \delta x + D \delta u$, $C = \frac{\partial h}{\partial \underline{x}} (\bar{x}, \bar{u}) \in \mathbb{R}^{1 \times n_1}$, $D = \frac{\partial h(\bar{x}, \bar{u})}{\partial u} \in \mathbb{R}$ \\
    *Only valid at equ. pairs.
    \customFigure[0.25]{../Images/L3_0.png}

    \textcolor{blue}{\textbf{Linear Approx.}} Given a diff. fcn. $f: \mathbb{R} \rightarrow \mathbb{R}$, its linear approx. at $\bar{x}$ is $f_{\text{lin}} = f(\bar{x}) + f'(\bar{x})(x-\bar{x})$. \\
    *Remainder Thm: $f(x) = f_{\text{lin}} + r(x)$ where $\lim_{x \rightarrow \bar{x}} \frac{r(x)}{x-\bar{x}} = 0$. 
    \customFigure[0.1]{../Images/L3_1.png}
    
    *Note: Can provide a good approx. near $\bar{x}$ but not globally. 
    
    *Gen. $f: \mathbb{R}^{n_1} \rightarrow \mathbb{R}^{n_2}$, $f(x) = f(\bar{x}) + \frac{\partial f}{\partial x}(\bar{x})(x-\bar{x}) + R(x)$ \\
    *Jacobian: $\frac{\partial f}{\partial x} (\bar{x})  = \begin{bmatrix} \frac{\partial f}{\partial x_1} (\bar{x})  & \ldots & \frac{\partial f}{\partial x_{n_1}} (\bar{x}) \end{bmatrix} \in \mathbb{R}^{n_2 \times n_1}$ 

    \textcolor{orange}{\textbf{Linearization Steps}} 1. Find equ. pair $(\bar{x}, \bar{u})$ \\
    2. Derive $A,B,C,D$ and then evaluate at $(\bar{x}, \bar{u})$ \\
    3. Write $\delta \overset{\cdot}{x} = A \delta x + B \delta u$ and $\delta y = C \delta x + D \delta u$ \\

    \textcolor{red}{\textbf{Laplace Transform}} Given a fcn $f: \mathbb{R}_+ = [0,\infty) \to \mathbb{R}^n$, its Laplace transform is 
    $F(s) = \mathcal{L} \{ f(t) \} := \int_{0^-}^{\infty} f(t) e^{-st} \, dt$, $s \in \mathbb{C}$. \\ 
    *$\mathcal{L:} f(t) \mapsto F(s)$, $t \in \mathbb{R}_+$ (time dom.) \& $s \in \mathbb{C}$ (Laplace dom.).
    
    \textcolor{blue}{\textbf{P.W. CTS:}} A fcn $f: \mathbb{R}_+ \to \mathbb{R}^n$ is \textbf{p.w. cts} if on every finite interval of $\mathbb{R}$, $f(t)$ has at most a finite \# of discontinuity points ($t_i$) and the limits 
    $\lim_{t \to t_i^+} f(t)$, $\lim_{t \to t_i^-} f(t)$ are finite.

    \customFigure[0.1]{../Images/L4_0.png}

    \textcolor{blue}{\textbf{Exp. Order}} A function $f: \mathbb{R}_+ \to \mathbb{R}^n$ is of \textbf{exp. order} if $\exists$ constants $K, \rho, T > 0$ s.t. $\|f(t)\| \leq K e^{\rho t}$, $\forall t \geq T$.

    \textcolor{blue}{\textbf{Existence of LT Thm}} If $f(t)$ is p.w. cts and of exp. order w/ constants $K, \rho, T > 0$, then $F(\cdot)$ exists and is defined $\forall \, s \in D := \{s \in \mathbb{C} : \text{Re}(s) > \rho\}$ and $F(\cdot)$ is analytic on $D$. \\
    *Analytic fcn iff differentiable fcn. \\
    *$D$: Region of convergence (ROC), open half plane. \\
    \customFigure[0.1]{../Images/L4_1.png}

    \textcolor{blue}{\textbf{Unit Step}} $\mathbf{1}(t) := 
        \begin{cases} 
        1, & \text{if } t \geq 0 \\ 
        0, & \text{otherwise}
        \end{cases}$

    \textcolor{blue}{\textbf{Table of Common Laplace Transforms:}} $f(t) \; \mid \; F(s)$ \\ 
    $\mathbf{1}(t) \mapsto \frac{1}{s} \, \quad t \mathbf{1}(t) \mapsto \frac{1}{s^2} \quad t^k \, \mathbf{1}(t) \mapsto \frac{k!}{s^{k+1}} \quad e^{at} \, \mathbf{1}(t) \mapsto \frac{1}{s-a}$ \\ 
    $t^n e^{at} \, \mathbf{1}(t) \mapsto \frac{n!}{(s-a)^{n+1}} \quad \sin(at) \, \mathbf{1}(t) \mapsto \frac{a}{s^2 + a^2} \quad $ \\
    $ \cos(at) \, \mathbf{1}(t) \mapsto \frac{s}{s^2 + a^2} \quad \frac{1}{2\omega^3} [\sin(\omega t) - \omega t \cos(\omega t)] \, \mathbf{1}(t) \mapsto \frac{1}{(s^2 + \omega^2)^2}$ \\

    \textcolor{blue}{\textbf{Prop. of Laplace Transform}} \textbf{Linearity:} \\ 
    $\mathcal{L} \{c f(t) + g(t)\} = c \mathcal{L} \{f(t)\} + \mathcal{L} \{g(t)\}, c \sim \text{constant}.$  
    
    \textbf{Differentiation:} If the Laplace transform of $f'(t)$ exists, then $\mathcal{L} \{f'(t)\} = s \mathcal{L} \{f(t)\} - f(0^-).$  \\
    If the Laplace transform of $f^{(n)}(t) := \frac{d^n f}{dt^n}(t)$ exists, then \\ 
    $\mathcal{L} \{f^{(n)}(t)\} = s^n \mathcal{L} \{f(t)\} - \sum_{i=1}^n s^{n-i} f^{(i-1)}(0^-).$  
    
    \textbf{Integration:}  
    $\mathcal{L} \left\{\int_0^t f(\tau) d\tau \right\} = \frac{1}{s} \mathcal{L} \{f(t)\}.$  
    
    \textbf{Convolution:} Let $(f * g)(t) := \int_0^t f(\tau) g(t-\tau) d\tau = \int_0^t f(t-\tau) g(\tau) d\tau$, then
    $\mathcal{L} \{(f * g)(t)\} = \mathcal{L} \{f(t)\} \mathcal{L} \{g(t)\}.$  
    
    \textbf{Time Delay:} $\mathcal{L} \{f(t-T) \mathbf{1}(t-T)\} = e^{-Ts} \mathcal{L} \{f(t)\}, T \geq 0.$  
    
    \textbf{Multiplication by $t$:} $\mathcal{L} \{t f(t)\} = -\frac{d}{ds} [\mathcal{L} \{f(t)\}].$  
    
    \textbf{Shift in $s$:} $\mathcal{L} \{e^{at} f(t)\} = \mathcal{L} \{f(t)\}\big|_{s \to s-a} = F(s-a)$, where $F(s) = \mathcal{L} \{f(t)\}$ \& $a ~ \text{const}$. \\   

    \textcolor{blue}{\textbf{Trig. Id.}} $2 \sin(2t) = 2 \sin(t) \cos(t)$, $\sin(a-b) = \sin(a) \cos(b) - \cos(a) \sin(b)$, $\cos(a-b) = \cos(a) \cos(b) + \sin(a) \sin(b)$

    \textcolor{blue}{\textbf{Complete the Square:}} $a x^2 + b x + c = a (x + \frac{b}{2a})^2 - \frac{b^2}{4a} + c$
    
    \textcolor{orange}{\textbf{LT Steps:}} 1. Write $f(t)$ as a sum and use linearity \\
    *Trig. id. may be useful. \\
    2. Use prop. of LT and common LT to find $F(s)$  

    \switchcolumn

    \textcolor{red}{\textbf{Inverse Laplace Transform}} Given $F(s)$, its inverse LT is  
    $f(t) = \mathcal{L}^{-1} \{ F(s) \} := \frac{1}{2\pi j} \int_{c-j\infty}^{c+j\infty} F(s) e^{st} ds$ \\ 
    $= \lim_{w \to \infty} \frac{1}{2\pi j} \int_{c-jw}^{c+jw} F(s) e^{st} ds$, $c \in \mathbb{C}$ is selected s.t. the line $L := \{s \in \mathbb{C} : s = c + j\omega, \omega \in \mathbb{R}\}$ is inside the ROC of $F(s)$.  
    
    \textbf{Zero:} $z \in \mathbb{C}$ is a zero of $F(s)$ if $F(z) = 0$.  

    \textbf{Pole:} $p \in \mathbb{C}$ is a pole of $F(s)$ if $\frac{1}{F(p)} = 0$.  

    \textcolor{blue}{\textbf{Cauchy's Residue THM}} If $F(s)$ is analytic (complex diff.) everywhere except at isolated poles $\{p_1, \dots, p_N\}$, then \\
    $\mathcal{L}^{-1} \{ F(s) \} = \sum_{i=1}^N \text{Res} \left[ F(s) e^{st}, s = p_i \right] \mathbf{1}(t)$,  \\
    *$\text{Res}[F(s)e^{st}, s = p_i]$: Residue of $F(s)e^{st}$ at $s = p_i$.  
    
    \textcolor{blue}{\textbf{Residue Computation}} Let $G(s)$ be a complex analytic fcn w/ a pole at $s = p$, $r$ be the multiplicity of the pole $p$. Then  
    $\text{Res}[G(s), s = p] = \frac{1}{(r-1)!} \lim_{s \to p} \frac{d^{r-1}}{ds^{r-1}} \left[ G(s) (s-p)^r \right]$.  

    \textcolor{orange}{\textbf{Inv. LT Partial Frac.:}} 1. Factorize $F(s)$ into partial fractions. \\
    2. Find coefficients and use LT table to find inverse LT. \\
    *Complete the square. 

    \textcolor{orange}{\textbf{Inv. LT Residue:}} 1. Find poles of $F(s)$ and their residues. \\
    2. Use Cauchy's Residue THM to find inverse LT. \\
    *Note: Complex Conjugate (CC) poles $\rightarrow$ CC residues (use Euler).
    *$\cos(x) = \frac{e^{jx} + e^{-jx}}{2}$, $\sin(x) = \frac{e^{jx} - e^{-jx}}{2j}$

    \textcolor{red}{\textbf{Transfer Function:}} Consider a CS in IO form. Assume zero initial conds. $y(0) = \cdots = \frac{d^{(n-1)}y}{dt^{(n-1)}}(0) = 0$ and \\ 
    $u(0) = \cdots = \frac{d^{(m-1)} u}{dt^{(m-1)}}(0) = 0$. Then the TF from $u$ to $y$ is \\
    $G(s) := \frac{y(s)}{U(s)} = \frac{b_m s^m + \cdots + b_0}{s^n + a_{n-1} s^{n-1} + \cdots + a_0}$. \\
    *\textbf{0 Ini. Conds.:} $y_0(s) = G(s) u(s)$ \\
    *\textbf{$\emptyset$ Ini. Conds.:} $y_{\emptyset}(s) = y_0(s) + \frac{\text{poly. based on initial conds.}}{s^n + a_{n-1} s^{n-1} + \cdots + a_0}$ \\

    \textcolor{orange}{\textbf{TF Steps (IO to TF):}} 1. Given IO form of CS, assume zero initial conds. \\
    2. Find $G(s)$ by taking LT of IO form and forming $Y(s)/U(s)$. \\
    *Careful: $Y(s)/U(s) = G(s)$ not $U(s)/Y(s) = G(s)$.

    \textcolor{blue}{\textbf{Impulse Response:}} Given CS modeled by TF $G(s)$, its IR is  
    $g(t) := \mathcal{L}^{-1} \{ G(s) \}$. \\
    *$\mathcal{L} \{\delta(t) \} = 1$, then if $u(t) \text{=} \delta(t)$, then $Y(s) = U(s) G(s) = G(s)$.

    \textcolor{red}{\textbf{SS to TF:}} $G(s) = C (sI - A)^{-1} B + D$ s.t. $y(s) = G(s) U(s)$. \\
    *Assume $x(0) = 0 \in \mathbb{R}^n$ (zero initial conds.). \\
    *\textbf{LTI:} $G(s)$ of an LTI system is always a rational fcn. \\ 
    *\textbf{Not Invertible:} Values of $s$ s.t. $sI - A$ not invertible can correspond to poles of $G(s)$.

    \textcolor{orange}{\textbf{Inverse:}} 1. For $A \in \mathbb{R}^{n \times n}$, find $[\text{cof}(A)]_{(i,j)} \text{=} (-1)^{i+j} \text{det}(A_{(i,j)})$. \\ 
    *$A_{(i,j)}$: $A$ w/ row $i$ and col. $j$ removed. \\
    2. Assemble $\text{cof}(A)$ and find $\text{det}(A) = \sum_{j=1}^n a_{ij} [\text{cof}(A)]_{(i,j)}$ \\
    w/ fixed $i$ or $\text{det}(A) = \sum_{i=1}^n a_{ij} [\text{cof}(A)]_{(i,j)}$ w/ fixed $j$. \\
    3. Find $A^{-1} = \frac{1}{\text{det}(A)} \text{adj}(A) = \frac{1}{\text{det}(A)} [\text{cof}(A)]^T$. \\
    *$2\times2: A^{-1} = \frac{1}{\text{det}(A)} \begin{bmatrix} d & -b \\ -c & a \end{bmatrix}$

    \textcolor{orange}{\textbf{TF (SS to TF):}} 1. Given SS form, assume zero initial conds. \\
    2. Solve $G(s) = C (sI - A)^{-1} B + D$. \\
    *If $C \text{=} \begin{bmatrix} 0 \; \cdot \; 1_i \; \cdot \; 0 \end{bmatrix}$ \& $B \text{=} \begin{bmatrix} 0 \; \cdot \; 1_j \; \cdot \; 0 \end{bmatrix}$, then only need ith row \& jth col. of $\text{adj}(sI - A)$ s.t. $G(s) \text{=} \frac{[\text{adj}(sI - A)]_{(i,j)}}{\text{det}(sI - A)} + D$. \\
    *Multiple $i$, $j$ non-zero entries: Work it out using MM.

    \textcolor{red}{\textbf{TF to SS:}} Consider $G(s) = \frac{b_m s^m + \cdots + b_0}{s^n + a_{n-1} s^{n-1} + \cdots + a_0} = \frac{N(s)}{D(s)}$, where $m < n$ (i.e. $G(s)$ is strictly proper). Then the SS form is \\
    *$A = \begin{bmatrix} 0 & 1 & 0 & \cdots & 0 \\ 0 & 0 & 1 & \cdots & 0 \\ \vdots & \vdots & \vdots & \ddots & \vdots \\ 0 & 0 & 0 & 0 & 1 \\ -a_0 & -a_1 & -a_2 & \cdots & -a_{n-1} \end{bmatrix}$, $B = \begin{bmatrix} 0 \\ \vdots \\ 0 \\ 1 \end{bmatrix}$, \\
    $C = \begin{bmatrix} b_0 & \cdots & b_m & \mid & 0 & \cdots & 0 \end{bmatrix}$, $D = 0$. \\
    *\textbf{Unique:} State space of a TF is not unique. 

    \textcolor{red}{\textbf{Summary:}} \customFigure[0.25]{../Images/L8_0.png}

    \textcolor{red}{\textbf{Block Diagram}} \textcolor{blue}{\textbf{Types of Blocks:}} \\
    \textcolor{black}{\textbf{Cascade:}} $y_2 = (G_1(s) G_2(s)) U \overset{\text{SISO}}{=} y_2 = (G_2(s)G_1(s)) U$ \customFigure[0.25]{../Images/L9_0.png} \\
    \textcolor{black}{\textbf{Parallel}} $y = (G_1(s) + G_2(s)) U$ \customFigure[0.25]{../Images/L9_1.png} \\
    \textcolor{black}{\textbf{Feedback}} $y = \left(\frac{G_1(s)}{1 + G_1(s) G_2(s)}\right) R$ \customFigure[0.25]{../Images/L9_2.png} \\
    *\textbf{SC:} Unity Feedback Loop (UFL) if $G_2(s) = 1$.

    \textcolor{blue}{\textbf{Manipulations:}} 1. $y = G(U_1 - U_2) = GU_1 + GU_2$ \\
    2. $y_1 = G U \quad y_2 = U \; \mid \; y_1 = G U \quad y_2 = G \frac{1}{G} U$ \\ 
    3. From feedback loop to UFL. 
    \customFigure[0.3]{../Images/L9_3.png} 

    \textcolor{orange}{\textbf{Find TF from Block Diagram:}} 1. Start from in $\rightarrow$ out, making simplifications using block diagram rules. \\
    2. Simplify until you get the form $U(s) \rightarrow \boxed{G(s)} \rightarrow Y(s)$.

    \textcolor{red}{\textbf{Time Response of Elementary Terms:}} $\mathbf{1}(t) \leftarrow \text{pole @ 0}$ \\
    $t^n \mathbf{1}(t) \leftarrow \text{pole @ $0$ w/ mult. $n$} \; \mid \; e^{at} \mathbf{1}(t) \leftarrow \text{pole @ } a$ \\
    $\sin(\omega t + \phi) \mathbf{1}(t) \leftarrow \text{pole @ } \pm j \omega \; \mid \; \cos(\omega t + \phi) \mathbf{1}(t) \leftarrow \text{pole @ } \pm j \omega$ 

    \switchcolumn

    \textcolor{blue}{\textbf{Real Pole:}} $y(s) \text{=} \frac{1}{s+a}$, real pole at $s = -a$, then $y(t) \text{=} e^{-at} \mathbf{1}(t)$ \\
    1. $a>0 \implies \lim_{t \rightarrow \infty} y(t) \text{=} 0 \mid$ 2. $a<0 \implies \lim_{t \rightarrow \infty} y(t) \text{=} \infty$. \\
    3. $a=0 \implies y(t) = \mathbf{1}(t)$ is constant.
    \customFigure[0.1]{../Images/L10_0.png}

    \textcolor{green}{\textbf{Time Constant:}} $\tau = \frac{1}{a}$ of the pole $s=-a$ for $a>0$

    \textcolor{blue}{\textbf{Pair of Comp. Conj. Poles:}} \\
    $y(s) = \frac{\omega_n^2}{s^2 + 2\zeta\omega_n s + \omega_n^2} = \frac{\omega_n^2}{(s + \sigma)^2 + \omega_d^2}$, $|\zeta| < 1$, then \\
    $y(t) = \frac{\omega_n}{\sqrt{1 - \zeta^2}} e^{-\sigma t} \sin(\omega_d t) \mathbf{1}(t)$ \\
    *Poles: $s_{1,2} = - \zeta \omega_n \pm j \omega_n \sqrt{1 - \zeta^2} = -\sigma \pm j \omega_d$ \\
    *$\zeta = \frac{\sigma}{\omega_n}$: Damping ratio (or damping coefficient) \\
    *$\sigma = \zeta \omega_n$: Decay/growth rate $\mid$ $\omega_d$: Freq. of oscillation \\
    *$\omega_n = \sqrt{\sigma^2 + \omega_d^2} \left[\frac{\text{radians}}{\text{seconds}} \right]$: Undamped natural freq. \\
    *$\omega_d = \omega_n \sqrt{1 - \zeta^2} \left[\frac{\text{radians}}{\text{seconds}} \right]$: Damped natural freq. \\
    *$|s_{1,2}|^2 = \omega_n^2$: Mag. of poles is $\omega_n$. \\
    *$\cos^{-1}(\zeta)$: Angle of $s_1$ on complex plane CW from -ve $\text{Re}$ axis. \\

    \customFigure[0.2]{../Images/L10_5.png}

    \textcolor{green}{\textbf{Damping Ratio Effect:}} $0 < \zeta_1 < \zeta_2 < 1$, then \\
    \customFigure[0.2]{../Images/L10_3.png}

    $-1 < \zeta_4 < \zeta_3 < 0$, then $\sigma = \zeta \omega_n <0$, (exp. envelop $\uparrow$) \\
    \customFigure[0.2]{../Images/L10_4.png}

    \textcolor{blue}{\textbf{Class. of 2nd Order Sys.:}} $y(s) = \frac{\omega_n^2}{s^2 + 2\zeta\omega_n s + \omega_n^2}$, w/ $|\zeta| < 1$ \\
    \customFigure[0.2]{../Images/L10_1.png}

    \textcolor{blue}{\textbf{Loc. of Poles and Behavior:}} 
    \customFigure[0.2]{../Images/L10_2.png}

    \textcolor{red}{\textbf{Control Spec. of 2nd Order Sys.:}} \textcolor{blue}{\textbf{Step Response:}} Given a TF $G(s)$, its SR is $y(t)$ resulting from applying the input $u(t) = \mathbf{1}(t)$, i.e. $\mathcal{L}^{-1} \left\{G(s) \frac{1}{s} \right\}$.

    \textcolor{blue}{\textbf{Control Spec.}} A control spec. is a criterion specifiying how we would like a CS to behave. 

    % \textcolor{blue}{\textbf{Metrics:}} Used to quantify the transient performance of 2nd order systems w/ $0 < \zeta < 1$.

    \textcolor{blue}{\textbf{2nd Order Sys. Metrics:}} $G(s) \text{=} \frac{\omega_n^2}{s^2 + 2\zeta\omega_n s + \omega_n^2}$ w/ $U(s) \text{=} \frac{1}{s}$ \\
    *$0 < \zeta < 1$ (i.e. 2 comp. conj. poles w/ $\text{Re(pole)}<0$). 
    \customFigure[0.2]{../Images/L10_6.png}

    \textcolor{green}{\textbf{Rise Time (RT)}:} $T_r$ is the time it takes $y(t)$ to go from 10\% to 90\% of its steady-state value.

    \textcolor{orange}{\textbf{RT:}} 1. Find $t_1 > 0$ s.t. $y(t_1) = 0.1$, $t_2 > 0$ s.t. $y(t_2) = 0.9$. \\
    3. Compute $T_r = t_2 - t_1$. $\boxed{T_r \approx \frac{1.8}{\omega_n}}$. 

    \textcolor{green}{\textbf{Settling Time (ST)}:} $T_s$ is the time required to reach and stay w/in 2\% of the steady-state value.

    \textcolor{orange}{\textbf{ST:}} 1. Find when it's first that $|y(t) - 1| \leq 0.02$. $\boxed{T_s \approx \frac{4}{\zeta \omega_n}}$.

    \textcolor{green}{\textbf{Peak Time}:} $T_p$ is time req'd to reach the max (peak) value.

    \textcolor{orange}{\textbf{Peak Time:}} 1. Find the first time when $\overset{\cdot}{y}(t) = 0$. \\
    *$\boxed{T_p = \frac{\pi}{\omega_d} = \frac{\pi}{\omega_n \sqrt{1 - \zeta^2}}}$.

    \textcolor{green}{\textbf{\% Overshoot}:} $\% \text{OS} = \frac{\text{[peak value]} - \text{[steady-state value]}}{\text{[steady-state value]}} \times 100\%$ \\
    *$\text{\% OS} = \text{OS} \times 100 \%$. \\
    *$\boxed{\% \text{OS} = \text{exp}\left(-\frac{\pi \zeta}{\sqrt{1-\zeta^2}}\right) \iff \zeta = \frac{-\ln(\text{OS})}{\sqrt{\pi^2 + (\ln(\text{OS}))^2}}}$.

    \switchcolumn
    \newpage

    \textcolor{red}{\textbf{Transient Performance Sat.:}} Given performance spec. $T_r \leq T_r^d$, $T_s \leq T_s^d$, $\text{OS} \leq \text{OS}^d$, find loc. of poles of $G(s)$. \\
    *Admissible region for the poles of $G(s)$ s.t. the step response meets all three spec. is the intersection of the above three regions. \\
    \textcolor{blue}{\textbf{Rise Time:}} $T_r \approx \frac{1.8}{\omega_n} \leq T_r^d \overset{\text{app.}}{\iff} \omega_n \geq \frac{1.8}{T_r^d} \equiv \omega_n^d$ 

    \customFigure[0.1]{../Images/L12_0.png}

    \textcolor{blue}{\textbf{Settling Time:}} $T_s \approx \frac{4}{\zeta \omega_n} = \frac{4}{\sigma} \leq T_s^d \overset{\text{app.}}{\iff} \sigma \geq \frac{4}{T_s^d} \equiv \sigma^d$ 
    \customFigure[0.1]{../Images/L12_1.png}

    \textcolor{blue}{\textbf{OS:}} $\text{exp}\left(\frac{-\pi \zeta}{\sqrt{1\text{-}\zeta^2}}\right) \leq \text{OS}^d \overset{\text{app.}}{\iff} \zeta \geq \frac{-\ln(\text{OS}^d)}{\sqrt{\pi^2 + (\ln(\text{OS}^d))^2}} \equiv \zeta^d$ 
    \customFigure[0.1]{../Images/L12_2.png}

    \textcolor{blue}{\textbf{Add. Poles \& Zeros:}} The analysis remains approx. correct under the following assumptions: \\
    1. Any add. poles of $G(s)$ have much more -ve real part (5-10 times) than the real part of the dom. complex conjugate poles. 
    \customFigure[0.2]{../Images/L12_3.png} \\
    *\textcolor{red}{dominant poles}, \textcolor{blue}{additional poles}.

    2. Real part of zeros are -ve \& very diff. from the real part of the two dom. poles. 

    \textcolor{red}{\textbf{Internal Stablity:}} $\overset{\cdot}{x} = Ax$ is \\
    1. \textbf{Stable} if $\forall x(0) \in \mathbb{R}^n$, the soln. $x(t)$ is bdd; that is, $\exists M > 0$ s.t. $\|x(t)\| \leq M$ $\forall t \geq 0$. \\
    2. \textbf{Asymp. Stable} if it's stable \& $\forall x(0) \in \mathbb{R}^n$, the soln. $x(t)$ converges to the origin; that is, $\lim_{t \to \infty} x(t) = 0$. \\
    3. \textbf{Unstable} if it's not stable; that is, $\exists x(0) \in \mathbb{R}^n$ s.t. $x(t)$ is not bdd. 

    \textcolor{blue}{\textbf{Asymptotic Stablity Thm.}} $\overset{\cdot}{x} = Ax$ is A.S. iff $\text{eig}(A) \subseteq \mathbb{C}^- \equiv \{s \in \mathbb{C} \mid \text{Re}(s) < 0\}$, i.e. open left half plane (OLHP).

    \textcolor{blue}{\textbf{Instability Thm.}} If $\exists$ an eigenvalue $\lambda$ of $A$ w/ $\text{Re}(\lambda) > 0$, then $\overset{\cdot}{x} = Ax$ is unstable.

    \textcolor{blue}{\textbf{Fact:}} Zeros of $s^2 + a_1 s + a_0$ are in $\mathbb{C}^-$ iff $a_1, a_0 > 0$.

    \textcolor{orange}{\textbf{Internal Stability}} 1. Linearize around $(\bar{x}, \bar{u})$ w/ $\bar{u} = 0$. \\
    2. Find $A$ and determine $\text{eig}(A) = \lambda$ s.t. $\det(s I - A) = 0$. \\ 
    3. Check if $\text{eig}(A) \subseteq \mathbb{C}^-$ for asymptotic stability \\ 
    4. Check if $\text{Re}(\text{eig}(A)) > 0$ for instability.

    \textcolor{red}{\textbf{BIBO Stability:}} An LTI system w/ 0 i.c. is Bounded Input Bounded Output (BIBO) stable if for any bdd input $u(t)$, the output $y(t)$ is also bdd.

    \textcolor{blue}{\textbf{BIBO Unstable:}} An LTI system w/ 0 i.c. is BIBO unstable if it's not BIBO stable; that is, $\exists$ a bdd $u(t)$ s.t. $y(t)$ is not bdd.

    \textcolor{blue}{\textbf{BIBO Stable Thm.}} A system $y(s) = G(s) U(s)$ is BIBO stable iff poles($G(s)) \subseteq \mathbb{C}^{-}.$

    \textcolor{blue}{\textbf{Lemma:}} If $p$ is a pole of $G(s)$, then $p$ is an $\text{eig}(A)$. I.e. $\text{poles}(G(s)) := \{ p \in \mathbb{C} \mid p \text{ is a pole of } G(s) \} \subseteq \text{eig}(A)$.  \\
    *\textbf{Pole-0 Cancellation:} $\text{eig}(A)$ need not be a pole of $G(s)$.

    \textcolor{blue}{\textbf{Thm.}} If $\text{eig}(A) \subseteq \mathbb{C}^{-}$, then $\forall B, C, D$ the TF $G(s)$ is BIBO stable. That is, internal asymptotic stability $\Rightarrow$ BIBO stability.

    \textcolor{orange}{\textbf{BIBO Stability}} 1. Find $G(s)$ from SS form and determine poles. \\
    2. Check if $\text{poles}(G(s)) \subseteq \mathbb{C}^{-}$.
    1. Check if $\text{eig}(A) \subseteq \mathbb{C}^{-}$ since internal asymptotic stability $\Rightarrow$ BIBO stability.
    
    \textcolor{red}{\textbf{Routh-Hurwitz:}} Consider $a(s) = s^n + a_{n-1} s^{n-1} + \cdots + a_0$. \\
    *$s^n \quad \; \, \, \mid \; 1 \quad a_{n-2} \quad a_{n-4} \quad a_{n-6} \quad \cdots \quad 0$ \\ 
    *$s^{n-1} \; \mid \; a_{n-1} \quad a_{n-3} \quad a_{n-5} \quad a_{n-7} \quad \cdots \quad 0$ \\
    *$s^{n-2} \; \mid \; b_1 \quad b_2 \quad b_3 \quad \cdots$ \\
    *$s^{n-3} \; \mid \; c_1 \quad c_2 \quad \cdots$ \\
    $\vdots$ \\
    *$s \quad \quad \; \mid \; * \quad 0$ \\
    *$1 \quad \quad \; \mid \; * \quad 0$ \\
    \(
    \textcolor{blue}{
    b_1 = -\frac{1}{a_{n-1}} \det 
    \begin{bmatrix} 
    1 & a_{n-2} \\ 
    a_{n-1} & a_{n-3} 
    \end{bmatrix}
    }
    \)
    \(
    \textcolor{red}{b_2 = -\frac{1}{a_{n-1}} \det 
    \begin{bmatrix} 
    1 & a_{n-4} \\ 
    a_{n-1} & a_{n-5} 
    \end{bmatrix}}
    \)
    \(
    \textcolor{green}{b_3 = -\frac{1}{a_{n-1}} \det 
    \begin{bmatrix} 
    1 & a_{n-6} \\ 
    a_{n-1} & a_{n-7} 
    \end{bmatrix}}
    \)
    \(
    c_1 = -\frac{1}{b_1} \det 
    \begin{bmatrix} 
    a_{n-1} & a_{n-3} \\ 
    b_1 & b_2 
    \end{bmatrix}
    \)
    \(
    c_2 = -\frac{1}{b_1} \det 
    \begin{bmatrix} 
    a_{n-1} & a_{n-5} \\ 
    b_1 & b_3 
    \end{bmatrix}
    \)

    \textcolor{blue}{\textbf{Routh-Hurwitz Stability Criterion:}} The roots of \(a(s)\) are in \(\mathbb{C}^{-}\) iff the 1st col of Routh array has no sign changes. The \# of sign changes is equal to the \# of roots of \(a(s) \in \mathbb{C}^{+} := \{ s \in \mathbb{C} : \operatorname{Re}(s) > 0 \}\). \\
    *If 1st element of a row is 0, Rooth array cannot be completed.
    
    \textcolor{blue}{\textbf{FVT v1:}} Suppose \( Y(s) = \mathcal{L}\{ y(t) \} \) is a proper rational fcn. If $y(\infty) := \lim_{t \to \infty} y(t)$ exists and is finite, then $y(\infty) = \lim_{s \to 0} s Y(s)$

    \textcolor{blue}{\textbf{FVT v2:}} Suppose \(Y(s) = \mathcal{L} \{ y(t) \}\) is a proper rational fcn. Moreover, suppose either: \\
    1. \(\text{poles}(Y(s)) \subseteq \mathbb{C}^{-}\) \\
    2. \( Y(s) \) has only one pole at \( s = 0 \) and all other poles are in \( \mathbb{C}^{-} \).

    Then \( y(\infty) := \lim_{t \to \infty} y(t) \) exists and is finite and satisfies $y(\infty) = \lim_{s \to 0} s Y(s).$
    
    \textcolor{orange}{\textbf{FVT}} 1. Does $y(\infty)$ exist? Check if pole at $s=0$, then compute Rooth Array to see if poles are in $\mathbb{C}^{-}$. \\
    2. Compute $\lim_{s \to 0} s Y(s)$ if it exists.
    
    \switchcolumn 
    
        }
\end{paracol}

\end{document}
