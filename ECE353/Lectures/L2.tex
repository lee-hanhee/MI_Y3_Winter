\begin{summary}
    \begin{itemize}
        \item The kernel is the part of the operating system (OS) that interacts with hardware (it runs in kernel mode).
        \item System calls are the interface between user and kernel mode:
        \begin{itemize}
            \item Every program must use this interface!
        \end{itemize}
        \item File format and instructions to define a simple “Hello world” (in 168 bytes):
        \begin{itemize}
            \item Difference between API and ABI.
            \item How to explore system calls.
        \end{itemize}
        \item Different kernel architectures shift how much code runs in kernel mode.
    \end{itemize}    
\end{summary}

\begin{faq}
    \begin{itemize}
        \item What is difference b/w printf and write?
    \end{itemize}
\end{faq}
\subsection{File Descriptor (Abstraction)}
\begin{motivation}
    Since our processes are independent, we need an explicit way to transfer data.
\end{motivation}

\begin{definition}
    \begin{enumerate}
        \item \textbf{IPC:} Inter-process communication is transferring data b/w two processes.
        \item \textbf{File Descriptor:} A resource that users may either read bytes from or write bytes to (identified by an index stored in a process).
        \begin{itemize}
            \item e.g. File or terminal.
        \end{itemize}
    \end{enumerate}
\end{definition}

\subsubsection{System Calls Make Requests to the Operating System}
\begin{definition}
\begin{verbatim}
ssize_t write(int fd, const void *buf, size_t count);
Description: writes bytes from a byte array to a file descriptor
    fd    - the file descriptor
    buf   - the address of the start of the byte array (called a buffer)
    count - how many bytes to write from the buffer

void exit_group(int status);
Description: exits the current process and sets an exit status code
    status - the exit status code (0–255)
\end{verbatim}    
\end{definition}

\begin{example}
    \textbf{Hypothetical "Hello World" Program}
\begin{lstlisting}[language=C]
void _start(void) {
    write(1, "Hello world\n", 12);
    exit_group(0);
}
\end{lstlisting}
\end{example}

\begin{warning}
    System calls uses registers, while C is stack based.
\end{warning}

\subsubsection{API Tells You What and ABI Tells You How}
\begin{definition}
    \begin{itemize}
        \item Application Programming Interface (API) abstracts the details and describes the arguments and return value of a function.
        \begin{itemize}
            \item e.g. A function takes 2 integer arguments
        \end{itemize}
        \item Application Binary Interface (ABI) specifies the details, specifically how to pass arguments and where the return value is.
        \begin{itemize}
            \item e.g. The same function using the C calling convention (arguments on the stack)
        \end{itemize}
    \end{itemize}
\end{definition}

\subsection{Programs on Linux Use the ELF Filer Format}
\begin{definition}
    Executable and Linkable Format (ELF) specifies both executables and libraries
    \begin{itemize}
        \item Always starts with the 4 bytes: 0x7F 0x45 0x4C 0x46 or with ASCII encoding: DEL 'E' 'L' 'F'
        \item These 4 bytes are called “magic”, and that’s how you know what kind of file this is (other file formats may have a different number of bytes)
    \end{itemize}
\end{definition}

\subsubsection{Bytes Represent an ELF File}
\begin{definition}
    
\end{definition}

\subsection{Kernels}
\begin{definition}
    \begin{itemize}
        \item \textbf{Kernel mode} is a privilege level on your CPU that gives access to more instructions.
        \item The kernel is the part of your operating system that runs in kernel mode.
        \item These instructions allow only trusted software to interact with hardware:
        \begin{itemize}
            \item e.g., only the kernel can manage virtual memory for processes.
        \end{itemize}
    \end{itemize}
\end{definition}

\subsection{System Calls Transition Between User and Kernel Mode}
\begin{definition}
    
\end{definition}

\subsection{System Calls Are Traceable}