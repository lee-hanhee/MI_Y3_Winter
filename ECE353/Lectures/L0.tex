\subsection{Converting Between Binary, Hexadecimal, and Decimal}
\begin{process}
\begin{enumerate}
    \item \textbf{Binary to Decimal:}
    \begin{enumerate}
        \item Write down the binary number.
        \item Assign place values, starting from $2^0$ on the rightmost digit.
        \item Multiply each binary digit by its corresponding power of 2.
        \item Add all the results together to get the decimal equivalent.
    \end{enumerate}
    
    \item \textbf{Decimal to Binary:}
    \begin{enumerate}
        \item Divide the decimal number by 2.
        \item Record the remainder (0 or 1).
        \item Repeat the division process with the quotient until the quotient is 0.
        \item Write the remainders in reverse order to obtain the binary equivalent.
    \end{enumerate}
    
    \item \textbf{Binary to Hexadecimal:}
    \begin{enumerate}
        \item Group the binary number into groups of 4 digits, starting from the right. Add leading zeros if necessary.
        \item Convert each 4-digit binary group to its hexadecimal equivalent using the binary-to-hex mapping (e.g., 0000 = 0, 0001 = 1, 1110 = E).
        \item Combine the hexadecimal digits to get the hexadecimal equivalent.
    \end{enumerate}
    
    \item \textbf{Hexadecimal to Binary:}
    \begin{enumerate}
        \item Write down each hexadecimal digit.
        \item Replace each hexadecimal digit with its 4-bit binary equivalent.
        \item Combine the binary groups to get the binary equivalent.
    \end{enumerate}
    
    \item \textbf{Decimal to Hexadecimal:}
    \begin{enumerate}
        \item Divide the decimal number by 16.
        \item Record the remainder as a hexadecimal digit (0–9 or A–F).
        \item Repeat the division process with the quotient until the quotient is 0.
        \item Write the remainders in reverse order to obtain the hexadecimal equivalent.
    \end{enumerate}
    
    \item \textbf{Hexadecimal to Decimal:}
    \begin{enumerate}
        \item Write down the hexadecimal number.
        \item Assign place values, starting from $16^0$ on the rightmost digit.
        \item Multiply each hexadecimal digit by its corresponding power of 16, converting any letters (A–F) to decimal values (A=10, B=11, etc.).
        \item Add all the results together to get the decimal equivalent.
    \end{enumerate}
\end{enumerate}
\end{process}

\subsection{Little-endian and Big-endian}
\begin{definition}
\begin{itemize}
    \item \textbf{Little-endian:} In the little-endian format, the least significant byte (LSB) of a multi-byte data value is stored at the lowest memory address, and the most significant byte (MSB) is stored at the highest memory address.

    \item \textbf{Big-endian:} In the big-endian format, the most significant byte (MSB) of a multi-byte data value is stored at the lowest memory address, and the least significant byte (LSB) is stored at the highest memory address.
\end{itemize}
\end{definition}

\begin{example}
    \begin{itemize}
        \item For example, the hexadecimal value \texttt{0x12345678} would be stored in memory as:
        \[
        \texttt{78 56 34 12}
        \]
        \item For example, the hexadecimal value \texttt{0x12345678} would be stored in memory as:
        \[
        \texttt{12 34 56 78}
        \]
    \end{itemize}
\end{example}

\subsection{Memory}
\begin{summary}
    Table, int*, \&a, int**a, *a, int[5], etc. 
\end{summary}