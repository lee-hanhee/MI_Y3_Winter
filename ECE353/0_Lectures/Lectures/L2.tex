\begin{summary}
    \begin{itemize}
        \item \texttt{ldd <executable>}: Shows which dynamic libraries are used by the executable.
        \item \texttt{-Db\_sanitize=address}: Add the flag to Meson to detect memory leaks, which was built into the compiler, but you have to recompile.
        \item \texttt{valgrind <executable>}: Detect memory leaks from malloc and free. 
        \begin{itemize}
            \item \textbf{Warning:} GNU C library (libc.so) may allocate memroy for its own uses, so it doesn't bother to free.
        \end{itemize}
    \end{itemize}
    \vspace{1em}

    \begin{itemize}
        \item Know the high-level rules of ABI changes without API changes.
    \end{itemize}
\end{summary}

\begin{faq}

\end{faq}

\subsection{What is an Operating System?}
\begin{definition}
    An operating system consists of a kernel and libraries required for your application.
\end{definition}

\begin{warning}
    OS have different libraries for different applications.
\end{warning}

\subsection{Normal Compilation in C}
\begin{definition}
    \customFigure[0.5]{../Images/L3_1.png}{}
\end{definition}

\begin{notes}
    Object files (.o) are ELF files with code for functions.
\end{notes}

\subsection{Static Libraries and Dynamic Libraries}
\subsubsection{Static Libraries}
\begin{definition}
    \customFigure[0.5]{../Images/L3_2.png}{}
    \begin{itemize}
        \item Static libraries are included at link time.
    \end{itemize}
\end{definition}

\begin{notes}
    \begin{itemize}
        \item Put all .o files into a single .a file (i.e. library), then link the library with the application.
    \end{itemize}
\end{notes}

\subsubsection{Dynamic Libraries}
\begin{motivation}
    C standard library (.so) is a dynamic library that is a collection of .o files containing funciton defintiions.
\end{motivation}

\begin{definition}
    \customFigure[0.5]{../Images/L3_4.png}{}
    \begin{itemize}
        \item Dynamic libraries are included at runtime.
        \item Dynamic libraries are for reusable code. Multiple applications can use the same library.
        \customFigure[0.5]{../Images/L3_3.png}{}
        \begin{itemize}
            \item  The operating system only loads \texttt{libc.so} in memory once, and shares it.
        \end{itemize}
        \item Dynamic libraries allow easier debugging. 
        \begin{itemize}
            \item Control dynamic linking with environment variables \texttt{LD\_LIBRARY\_PATH} and \texttt{LD\_PRELOAD}.
        \end{itemize}
    \end{itemize}
\end{definition}

\subsubsection{Comparison of Static and Dynamic Libraries}
\begin{notes}
    \begin{itemize}
        \item \textbf{Static:} Statically linking basically copies the \texttt{.o} files directly into the executable
        \begin{itemize}
            \item Statically linking prevents re-using libraries (commonly used libraries have many duplicates)
            \item Any updates to a static library requires the executable to be \textbf{recompiled}
        \end{itemize}
        \item \textbf{Dynamic:} The executable has a reference to the dynamic library
        \begin{itemize}
            \item Dynamic libraries updates can break executables.
            \begin{itemize}
                \item A dynamic library update may subtly change the ABI causing a crash.
            \end{itemize}
        \end{itemize}
    \end{itemize}
\end{notes}

\begin{example}
    \begin{enumerate}
        \item \textbf{Dynamic Library ABI Changes:}
        \begin{itemize}
            \item \textbf{Given:} Consider the following in a dynamic library:
            \begin{itemize}
                \item A \texttt{struct} with multiple fields corresponding to a specific data layout (C ABI).
                \item An executable accesses the fields of the \texttt{struct} used by a dynamic library.
            \end{itemize}
            \item \textbf{Problem:} Now, if a dynamic library reorders the fields:
            \begin{itemize}
                \item The executable uses the old offsets and is now wrong.
            \end{itemize}
            \item \textbf{Note:} This is OK if the dynamic library never exposes the fields of a \texttt{struct}.
        \end{itemize}
    
        \item \textbf{C Uses a Consistent ABI for Structs:}
        \begin{itemize}
            \item \texttt{struct}s are laid out in memory with the fields matching the declaration order.
            \item C compilers ensure the ABI (Application Binary Interface) of \texttt{struct}s is consistent for an architecture.
            \item Consider the following structures:
            \begin{itemize}
                \item Library v1:
        \begin{lstlisting}[language=C]
struct point {
    int y;
    int x;
};
        \end{lstlisting}
        \item Library v2:
        \begin{lstlisting}[language=C]
struct point {
    int x;
    int y;
};
            \end{lstlisting}
            \end{itemize}
            \item For v1, the \texttt{x} field is offset by 4 bytes from the start of \texttt{struct point}'s base.
            \item For v2, it is offset by 0 bytes, and this difference will cause problems.
        \end{itemize}
    
        \item \textbf{After Compilation the Translation Differs for Each Version:}
        \customFigure[0.5]{../Images/L3_6.png}{}
    
        \item \textbf{Point API Has 4 Functions:}
        \customFigure[0.5]{../Images/L3_7.png}{}
        
        \item \textbf{ABI Stable Code Should Always Print "1,2" for Both Lines:}
        \customFigure[0.5]{../Images/L3_8.png}{}

        \item \textbf{Mismatched Versions Don't Agree on the Location of X and Y:}
        \customFigure[0.5]{../Images/L3_9.png}{}
        
        \item \textbf{Takeaways:} The definition of struct point in both libraries is different. Order of x and y change. 
        \begin{itemize}
            \item Our code works correctly if the compiled and linked versions match. If you expose a struct it becomes part of your ABI.
            \item A proper stable ABI would hide the struct from point.h
        \end{itemize}
    \end{enumerate}
\end{example}

\subsection{Semantic Versioning}
\begin{definition}
    Given a version number \texttt{MAJOR.MINOR.PATCH}, increment the:
    \begin{itemize}
        \item \textbf{MAJOR} version when you make incompatible API/\textbf{ABI} changes.
        \item \textbf{MINOR} version when you add functionality in a backwards-compatible manner.
        \item \textbf{PATCH} version when you make backwards-compatible bug fixes.
    \end{itemize}
\end{definition}

\subsection{System Calls are Rare in C}
\begin{definition}
    Most system calls have corresponding function calls in C, but may:
    \begin{itemize}
        \item Set \texttt{errno}
        \item Buffer reads and writes (reduce the number of system calls)
        \item Simplify interfaces (function combines two system calls)
        \item Add new features
    \end{itemize}
\end{definition}

\subsubsection{C exit}
\begin{definition}
    \begin{itemize}
        \item \textbf{System Call Exit (or exit\_group):} Program stops at that point.
        \item \textbf{C Exit:} Feature to register functions to call on program exit (e.g. \texttt{atexit}).
        \begin{itemize}
            \item i.e. Runs the function when the program exits.
        \end{itemize}
    \end{itemize}
\end{definition}
\begin{example}
\begin{lstlisting}[language=C]
    #include <stdio.h>
    #include <stdlib.h>
    
    void fini(void) {
        puts("Do fini");
    }
    
    int main(void) {
        atexit(fini);
        puts("Do main");
        return 0;
    }
\end{lstlisting}
    \begin{itemize}
        \item Return 0 is the same as calling \texttt{exit(0)}.
    \end{itemize}
\end{example}

