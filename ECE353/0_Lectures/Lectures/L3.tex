\begin{summary}
    \begin{itemize}
        \item User space (applications, libraries) vs. kernel space (OS)
        \begin{itemize}
            \item User space to kernel space: system calls
            \item Libraries: Performs the system calls in this course.
        \end{itemize}
        \item Open file descriptors: 0,1,2 go to Terminal (stdin, stdout, stderr)
        \item Process: Virtual registers, virtual memory (stacks, heap), open file descriptors.
    \end{itemize}
\end{summary}

\begin{faq}
    \begin{itemize}
        \item What does the terminal do with the open file descriptors?
        \item What does the file 1, file 2, and terminal do in the initial summary? 
        \item Slide 3: What is PCB? Contains all the info about the process.
        \item Slide 3: What is a PID? Keeps track of a running process.
        \item Slide 4: What do the different componetns of the process state diagram mean? REWATCH
        \begin{itemize}
            \item Created: Process is created.
            \item Ready: Process is ready to run.
            \item Running: Process is running, but it goes back to ready because it is waiting for something.
            \item Blocked: Process is blocked, which goes to 
        \end{itemize}
        \item Slide 5: What is the proc filesystem?
        \begin{itemize}
            \item The "proc" filesystem is a special filesystem in Unix-like operating systems that presents information about processes and other system information in a hierarchical file-like structure.
            \item It is commonly mounted at `/proc`.
            \item It allows users and applications to access kernel information in a structured and readable way.
        \end{itemize}
        \item Slide 7: What is parent and child process?
        \begin{itemize}
            \item Parent process: The original process that creates a new process.
            \item Child process: The new process created by the parent process.
            \item The child process is a copy of the parent process, but with a unique process ID.
        \end{itemize}
        \item Slide 8: How to distinguish between parent and child process? 0 is returned in the child process, and the process ID of the child process is returned in the parent process.
        \item Slide 9: What is POSIX system? POSIX (Portable Operating System Interface) is a family of standards specified by the IEEE for maintaining compatibility between operating systems.
        \item Slide 9: What is man? is a command to display documentation, and use number. 
        \item Slide 8: What happens when you call fork? Only the parent process calls the fork, and the child process is created. 
        \item Slide 8: Will parent run first then child? No, since there two independent processes, the order of execution is not guaranteed.
    \end{itemize}
\end{faq}

\subsection{Process: Adding onto the Process}
\begin{definition}
    \customFigure[0.5]{../Images/L3_10.png}{Process}
\end{definition}

\subsection{Process Control Block (PCB)}
\begin{definition}
    Contains all information.
    \begin{itemize}
        \item Process state
        \item CPU registers
        \item Scheduling information
        \item Memory management information
        \item I/O status information
        \item Any other type of accounting information
    \end{itemize}    
\end{definition}

\begin{warning}
    Each process gets a unique process ID (pid) to keep track of it.     
\end{warning}

\begin{example}
    In Linux, this is the \texttt{task\_struct} structure.
\end{example}

\subsection{Process State Diagram}
\begin{definition}
    \customFigure[0.5]{../Images/L3_11.png}{Process State Diagram}
\end{definition}

\begin{notes}
    Waiting $\iff$ Ready.
\end{notes}

\subsection{proc Filesystem}
\begin{definition}
    Read process state using the "proc" file system.
\end{definition}

\subsection{Create processes from scratch}

\subsubsection{Fork}
\begin{definition}
    \texttt{fork()} creates a new process, a copy of the current one.
\begin{lstlisting}[language=C]
int fork(void);
\end{lstlisting}
        
    \begin{itemize}
        \item Returns the process ID of the newly created child process:
        \begin{itemize}
            \item -1: on failure
            \item 0: in the child process
            \item >0: in the parent process
        \end{itemize}
    \end{itemize}
        
\end{definition}

\subsection{Clone processes}


