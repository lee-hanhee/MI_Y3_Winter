\begin{summary}
    
\end{summary}

\subsection{Useful Terminal Commands}
\begin{summary}
    \begin{itemize}
        \item \texttt{./hello-world-linux-aarch64} to run hello world.
        \item \texttt{readelf -a <FILE>} to see the ELF header.
        \item \texttt{strace <PROGRAM>} to trace all the system calls a process makes on Linux.
    \end{itemize}
\end{summary}

\subsection{Three OS Concepts}
\begin{definition}
    \begin{enumerate}
        \item \textbf{Virtualization:} Share one resource by mimicking multiple independent copies.
        \item \textbf{Concurrency:} Handle multiple things happening at the same time.
        \item \textbf{Persistence:} Retain data consistency even without power. 
    \end{enumerate}
\end{definition}

\subsection{OS Manages Resources}
\begin{definition}
    Insert picture. 
\end{definition}

\subsection{Program}
\begin{definition}
    A file containing all the instructions and data required to run. 
\end{definition}

\subsection{Process:}
\begin{definition}
    An instance of running a program.
    \customFigure[0.5]{../Images/L1_0.png}{Process}
\end{definition}

\subsubsection{Basic Requirements for a Process}
\begin{definition}
    Insert picture w/ virtual memory. 
\end{definition}

\subsection{Process (Abstraction)}

\subsubsection{Static}
\begin{definition}
    Only able to use the global variable in the current C file.     
\end{definition}

\subsubsection{Motivation for Virtualization}
\begin{motivation}
    How to run two different programs at the same time?
        Insert code. 
        \begin{itemize}
            \item Was the address of local the same b/w 2 processes? Different address in physical memory b/w different processes.
            \item Was the address of global the same b/w 2 processes? Same address in physical memory b/w different processes, but uses virtual memory.
            \item What else may be needed for a process?  
        \end{itemize}
\end{motivation}

\begin{warning}
    Local variables are stored on the stack. 
\end{warning}

\subsubsection{Does the OS allocate different stacks for each process?}
\begin{definition}
    The stacks for each process need to be in physical memory. One option is the operating system just allocates any unused memory for the stack. 
    \begin{itemize}
        \item 
    \end{itemize}
\end{definition}

\subsubsection{What about global variables?}
\begin{definition}
    The compiler needs to pick an address (random) for each variable when you compile.
    \begin{itemize}
        \item What if we had a global registry of addresses? Impossible (too much space and know memory addresses ahead of time).
    \end{itemize}
\end{definition}