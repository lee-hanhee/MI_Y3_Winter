\documentclass[twoside]{article}
\usepackage{style}
\title{ECE353 Cheatsheet}
\author{Hanhee Lee}
\lhead{ECE353}
\rhead{Hanhee Lee}

\begin{document}

\section{Terminal Commands}
\begin{summary}
    \begin{center}
        \begin{tabular}{l}
        \toprule
        \textbf{Terminal Command} \\
        \midrule
        \texttt{strace <PROGRAM>} \\
        \multicolumn{1}{p{\linewidth}}{
        \begin{itemize}
            \item Trace all the system calls a process makes on Linux. 
        \end{itemize}} \\
        \midrule
        \texttt{valgrind <executable>} \\
        \multicolumn{1}{p{\linewidth}}{
        \begin{itemize}
            \item Detect memory leaks from \texttt{malloc} and \texttt{free}.
        \end{itemize}} \\
        \midrule
        \texttt{-Db\_sanitize=address} \\
        \multicolumn{1}{p{\linewidth}}{
        \begin{itemize}
            \item Detect memory leaks by adding this flag to Meson.
        \end{itemize}} \\
        \midrule
        \texttt{htop} \\
        \multicolumn{1}{p{\linewidth}}{
        \begin{itemize}
            \item Process tree. Use F5 to swtich b/w tree and list view.
        \end{itemize}} \\
        \bottomrule
        \end{tabular}
    \end{center}
\end{summary}
\newpage

\section{Functions}
\begin{summary}
\begin{lstlisting}[language=C]
int fork();
\end{lstlisting}
    \begin{itemize}
        \item Creates a new process that's a clone of the currently running process. In the original process, it returns the process ID (pid) of the newly created child process. In the child process, it returns 0.
    \end{itemize}
    
\begin{lstlisting}[language=C]
int execlp(const char *file, const char *arg, ...);
\end{lstlisting}
    \begin{itemize}
        \item Replaces the current process with a new program specified by \texttt{file}. The new process is given the command-line arguments specified by \texttt{arg} and any additional arguments. Returns only if there is an error.
    \end{itemize}
    
\begin{lstlisting}[language=C]
int dup2(int oldfd, int newfd);
\end{lstlisting}
    \begin{itemize}
        \item Duplicates the file descriptor \texttt{oldfd} to \texttt{newfd}. If \texttt{newfd} is already open, it is first closed. Returns the new file descriptor on success.
    \end{itemize}
    
\begin{lstlisting}[language=C]
int waitpid(pid_t pid, int *status, int options);
\end{lstlisting}
    \begin{itemize}
        \item Waits for a specific child process (\texttt{pid}) to change state. The state change is stored in \texttt{status}. The \texttt{options} argument can modify the behavior of \texttt{waitpid}, use 0 for the defaults (blocking). Returns the pid of the child process on success.
    \end{itemize}
    
\begin{lstlisting}[language=C]
int pipe(int pipefd[2]);
\end{lstlisting}
    \begin{itemize}
        \item Creates a unidirectional data channel. \texttt{pipefd[0]} is set up for reading, and \texttt{pipefd[1]} is set up for writing. Returns 0 on success.
    \end{itemize}
    
\begin{lstlisting}[language=C]
void exit(int status);
\end{lstlisting}
    \begin{itemize}
        \item Terminates the calling process with an exit status of \texttt{status}.
    \end{itemize}
    
\begin{lstlisting}[language=C]
ssize_t write(int fd, const void *buf, size_t count);
\end{lstlisting}
    \begin{itemize}
        \item Writes \texttt{count} bytes from \texttt{buf} to the file or device associated with \texttt{fd}. Returns the number of bytes written.
    \end{itemize}
    
\begin{lstlisting}[language=C]
ssize_t read(int fd, void *buf, size_t count);
\end{lstlisting}
    \begin{itemize}
        \item Reads up to \texttt{count} bytes from the file or socket associated with \texttt{fd} into \texttt{buf}. Returns the number of bytes read.
    \end{itemize}
    
\begin{lstlisting}[language=C]
int pthread_create(pthread_t *thread, const pthread_attr_t *attr, 
                    void *(*start_routine)(void *), void *arg);
\end{lstlisting}
    \begin{itemize}
        \item Creates a new thread with attributes specified by \texttt{attr}. The new thread starts execution by invoking \texttt{start\_routine} with \texttt{arg} as its argument. Returns 0 on success.
    \end{itemize}
    
\begin{lstlisting}[language=C]
void pthread_exit(void *retval);
\end{lstlisting}
    \begin{itemize}
        \item Terminates the calling thread, returning \texttt{retval} to any joining thread. 
    \end{itemize}
    
\begin{lstlisting}[language=C]
int pthread_join(pthread_t thread, void **retval);
\end{lstlisting}
    \begin{itemize}
        \item Waits for the thread specified by \texttt{thread} to terminate. The thread's return value is stored in \texttt{retval}. Returns 0 on success.
    \end{itemize}
    
\begin{lstlisting}[language=C]
int pthread_detach(pthread_t thread);
\end{lstlisting}
    \begin{itemize}
        \item Detaches the specified thread, so its resources can be reclaimed immediately upon termination. Returns 0 on success.
    \end{itemize}  
    
\begin{lstlisting}[language=C]
atexit(void (*function)(void));
\end{lstlisting}
    \begin{itemize}
        \item Register functions to call on program exit.
    \end{itemize}

\begin{lstlisting}[language=C]
int execve(const char *pathname, char *const argv[], char *const envp[]);
\end{lstlisting}
    \begin{itemize}
        \item Replaces the current process with a new program and resets. 
        \begin{itemize}
            \item \texttt{pathname:} Full path of the program to load. 
            \item \texttt{argv:} Array of strings (array of characters), terminated by a null pointer. Represents arguments to the process.
            \item \texttt{envp:} Array of strings (array of characters), terminated by a null pointer. Represents the environment variables of the process.
            \item Returns only if there is an error.
        \end{itemize}
    \end{itemize}

\begin{lstlisting}[language=C]
int open(const char *pathname, int flags);
\end{lstlisting}
    \begin{itemize}
        \item Opens a file specified by \texttt{pathname} with the specified \texttt{flags}. Returns a file descriptor on success.
    \end{itemize}

\begin{lstlisting}[language=C]
int close(int fd);
\end{lstlisting}
    \begin{itemize}
        \item Closes the file descriptor \texttt{fd}. Returns 0 on success.
    \end{itemize}

\begin{lstlisting}[language=C]
DIR *opendir(const char *name);
\end{lstlisting}
    \begin{itemize}
        \item Opens the directory specified by \texttt{name} for reading. Returns a pointer to a \texttt{DIR} structure on success.
    \end{itemize}

\begin{lstlisting}[language=C]
int closedir(DIR *dirp);
\end{lstlisting}
    \begin{itemize}
        \item Closes the directory stream pointed to by \texttt{dirp}. Returns 0 on success.
    \end{itemize}

\begin{lstlisting}[language=C]
ssize_t read(int fd, void *buf, size_t count);
\end{lstlisting}
    \begin{itemize}
        \item Reads up to \texttt{count} bytes from the file descriptor \texttt{fd} into the buffer \texttt{buf}. Returns the number of bytes read on success or \texttt{-1} on error.
    \end{itemize}

\begin{lstlisting}[language=C]
void perror(const char *s);
\end{lstlisting}
    \begin{itemize}
        \item Prints a descriptive error message to \texttt{stderr}, prefixed by the string \texttt{s}, based on the current value of \texttt{errno}.
    \end{itemize}

\begin{lstlisting}[language=C]
void exit(int status);
\end{lstlisting}
    \begin{itemize}
        \item Terminates the calling process with the specified \texttt{status}. Use \texttt{EXIT\_SUCCESS} or \texttt{EXIT\_FAILURE} for standard status codes.
    \end{itemize}
\end{summary}
\cleardoublepage

\section{Operating System Structure}

\subsection{Different Types of Kernels}

\section{Processes}
\begin{definition}
    An instance of running a program.
\end{definition}

\section{Threads}

\section{Synchronization}

\section{CPU Scheduling}

\section{Memory Management}

\section{File Systems}

\section{I/O}




\end{document}